\documentclass[a4paper,12pt]{article}
\newcommand{\newpar}[1]{\bigskip \noindent \textbf{#1.}}
\newcommand{\subpar}[1]{\medskip \noindent (#1)}
\newcommand{\la}{\leftarrow}
\newcommand{\ra}{\rightarrow}

\begin{document}
\newpar{1} ``ENTA 1000; STA X''.

\newpar{2} Actually the address is set to the next instruction
preceding the jump to the subroutine in instruction $03$.

\newpar{3} Read $100$ words from tape unit 0 and store them in $X+1,
\ldots, X+100$.  Sort this array in increasing order and write back
the result to tape unit 1.

\newpar{4}
\begin{verbatim}
3000 00 00 00 18 35
3001  2050 00 05 09
3002  2051 00 05 10
3003     1 00 00 49
3004   499 01 05 26
3005  3016 00 01 41
3006     2 00 00 50
3007     2 00 02 50
3008 00 00 00 02 48
3009 00 00 02 02 55
3010    -1 03 05 04
3011  3006 00 01 47
3012    -1 03 05 56
3013     1 00 00 51
3014  3008 00 06 39
3015  3003 00 00 39
3016  1995 00 18 37
3017  2035 00 02 52
3018   -50 00 02 53
3019   501 00 00 53
3020    -1 05 05 08
3021 00 00 00 01 05
3022 00 00 04 12 31
3023 00 01 00 01 52
3024    50 00 01 53
3025  3020 00 02 45
3026 00 00 04 18 37
3027    24 04 05 12
3028  3019 00 00 45
3029 00 00 00 02 05
0000              2
1995 06 09 19 22 23
1996 00 06 09 25 05
1997 00 08 24 15 04
1998 19 05 04 00 17
1999 19 09 14 05 22
2024           2035
2029           2010
2050           -499
2051              3
\end{verbatim}

\newpar{5} Because the machine waits the unit is ready, if it's not
initially ready.

\newpar{6}
\subpar{a}  Suppose $n$ is not a prime.  Then there exits two integers
$a$ and $b$ such that $1 < a\le b <n$ and $n = ab$.  If $a > \sqrt{n}$,
we have:
\[ n = a b > a^2 = n.\]
Which is absurd so $1 \le a \le \sqrt{n}$.

\subpar{b} Note: $P = \mathrm{PRIME}[K]$.  We have
\[ N = Q P + R,\,\mbox{with $0 < R  < P$}.\]
If we have $Q \le P$, then
\[ N \le P^2 + R < P^2 + P \le (P+1)^2.\]
Thus
\[ \sqrt{N} < P+1.\]
If $N$ has a divisor $d$, we deduce from (a) that $d < P+1$.  That's
equivalent to $d \le P$.  We then deduce that $N$ has no divisor since
we have already tested that no prime number less than $P$ divide $N$.

\newpar{7}
\subpar{a} 4B here references the address 4H at line $29$.

\subpar{b} The jump at line $14$ would references line $15$ instead of
line $25$.

\newpar{8} The program writes $i$ zero(es) word(s) followed by the word
``AAAAA'' to the buffer BUF, where $i$ varies from $1$ to $75$ ---
which represents $2925$ words (100 lines).  It then writes $2400$ words from the
buffer to the line printer.

\newpar{11} By symmetry, we could assume that the saddle point in the
matrix is $a_{11}$.  Thus, the probability the matrix has a saddle
point is $m n$ times the probability $a_{11}$ is a saddle point.  The
later is equal to $1/(mn)!$ times the probability to have
\[ a_{2,1} < a_{1,1}, \ldots, a_{m,1} < a_{1,1}, a_{1,1} < a_{1,2},
\ldots, a_{1,1} < a_{1,n}.\]
Which is equal to
\[ C_{mn}^{m+n-1} (m-1)! (n-1)! (mn - (m+n-1))! =
\frac{(mn)!(m-1)!(n-1)!}{(m+n-1)!}.\]
Finally, the probability is equal to
\[ m n \times \frac{(m-1)!(n-1)!}{(m+n-1)!} = \frac{mn}{C_{m+n}^m}.\]

If the elements of the matrix are zeroes and ones, all we have to do
is finding a column of zeroes or a row of ones.  The probability that
there's no column of zeroes is
\[ \left(1 - \frac{1}{2^m}\right)^n\]
and the probability that there's no row of ones
\[ \left(1 - \frac{1}{2^n}\right)^m.\]
Thus the probability to have a saddle point is
\[ \left(1 - \left(1 - \frac{1}{2^m}\right)^n\right)
+  \left(1 - \left(1 - \frac{1}{2^n}\right)^m\right).\]

\newpar{17} For each positive integers $k$, $n$ and $M$, we have

\[ M = \lfloor 10^n/k + 1/2\rfloor \]

if and only if

\[ \left\lfloor \frac{2\times 10^n}{2M+1}\right\rfloor + 1 \le k \le
\left\lfloor \frac{2\times 10^n}{2M-1}\right\rfloor\mbox{(*)}.\]

If we note $N = \left\lfloor\frac{2\times10^n}{2m+1}\right\rfloor$, we
have

\begin{eqnarray*}
  S_n &=& \sum_{k=1}^{2\times10^n}r_n(1/k) \\
  &=& 10^{-n} \sum_{k=1}^N \lfloor 10^n/k + 1/2\rfloor + 10^{-n}
  \sum_{k=N+1}^{2\times10^n} \lfloor 10^n/k + 1/2\rfloor \\
  &=& 10^{-n} \sum_{k=1}^N (10^n/k + O(1)) +
  10^{-n} \sum_{k=N+1}^{2\times10^n} \lfloor 10^n/k + 1/2\rfloor \\
  &=& H_N + O(N/10^n) +  10^{-n} \sum_{k=N+1}^{2\times10^n} \lfloor 10^n/k +
  1/2\rfloor \\
  &=& H_N + O(m^{-1}) + 10^{-n} \sum_{k=1}^m
  \sum_{\left\lfloor\frac{2\times10^n}{2k+1}\right\rfloor + 1\le j \le
    \left\lfloor\frac{2\times 10^n}{2k-1}\right\rfloor} \lfloor 10^n/j
  + 1/2\rfloor \\
  &=& H_N + O(m^{-1}) + 10^{-n} \sum_{k=1}^m k
  \left(\left\lfloor\frac{2\times 10^n}{2k-1}\right\rfloor -
  \left\lfloor\frac{2\times10^n}{2k+1}\right\rfloor\right),\,\mbox{from
    (*)} \\
  &=& H_N + O(m^{-1}) + 10^{-n} \left( \sum_{k=0}^{m-1}
  \left\lfloor\frac{2\times10^n}{2k+1}\right\rfloor -
  m\left\lfloor\frac{2\times10^n}{2m+1}\right\rfloor\right) \\
  &=& H_N + O(m^{-1}) + 10^{-n} \left(
  \sum_{k=0}^{m-1} \frac{2\times 10^n}{2k+1} - 10^n (1 - O(m^{-1})) + O(m)
  \right) \\
  &=& H_N + O(m^{-1}) - 1 + O(m/10^n) + 2 \sum_{k=0}^{m-1}
  \frac{1}{2k+1} \\
  &=& H_N + O(m^{-1}) - 1 + O(m/10^n) + 2 \left(H_{2m} -
  \frac{H_m}{2}\right)
\end{eqnarray*}

Given that $H_n = \ln n + \gamma - o(n^{-1})$, we then deduce

\[  S_n = \ln N + 2\gamma + 2\ln 2 - 1 + o(N^{-1}) + O(m^{-1}) + \ln m
+ O(m/10^n).\]

Moreover, if we choose $m = \lfloor 10^{n/2}\rfloor$ we have

\[ N = \frac{2\times 10^n}{2m+1} + O(1).\]

And

\[ 0 \le \frac{1}{2m+1} - \frac{1}{2\times10^{n/2}+1} <
\frac{1}{2\times10^{n/2}-2} - \frac{1}{2\times10^{n/2}+1} =
O(10^{-n}).\]

So finally, we have $N = 10^{n/2} + O(10^{-n})$.  Thus, we deduce that

\[ S_n = 10 \ln n + 2\gamma + 2\ln 2 -1 + O(10^{-n/2}).\]

\newpar{19} \subpar{a} Let's show the equality by induction.

If $k=0$, we have

\[ x_1 y_0 - x_0 y_1 = 1.\]

Suppose that we have the equality for $k \ge 0$, then we have

\begin{eqnarray*}
  x_{k+2} y_{k+1} - x_{k+1} y_{k+2} &=& x_{k+1} y_k - x_k y_{k+1} \\
  &=&1\,\mbox{, by induction}
\end{eqnarray*}

\subpar{b}  Let $x, y$ be integers such that $0 \le x \le y \le n$.
And suppose there's a positive integer $k$ such that:

\[ \frac{x_n}{y_n} < \frac{x}{y} < \frac{x_{n+1}}{y_{n+1}}.\]

We then deduce

\begin{eqnarray*}
  y x_k y_{k+1} < x y_k y_{k+1} < y x_{k+1} y_k \\
  \max\left( y_k( y x_{k+1} - x y_{k+1} ), y_{k+1} (x y_k - y x_k)
  \right) < y (x_{k+1} y_k - x_k y_{k+1}) \\ 
  y_k + y_{k+1} \le 2 \max(y_k, y_{k+1}) <  y
\end{eqnarray*}

So finally, we have

\begin{eqnarray*}
  n \ge y &>& y_k + \left((y_{k-1} + n)/y_k - 1 \right) y_k - y_{k-1} \\
  &\ge& n
\end{eqnarray*}

Which is absurd.  We then deduce that there's no fraction between $0$
and $1$ whose denominator is less than $n$ and whose value lays
between two consecutive elements of the sequel $(x_n)$.  We then
deduce that $(x_n)$ is in fact the Farey series of order $n$.
\end{document}

