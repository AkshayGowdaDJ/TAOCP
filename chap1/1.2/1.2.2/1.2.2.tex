\documentclass[a4paper,12pt]{article}
\newcommand{\newpar}[1]{\bigskip \noindent \textbf{#1.}}
\newcommand{\la}{\leftarrow}
\begin{document}

\newpar{1} There's no smallest positive rational number.

\newpar{2} No. It's $1 + 0.24$.

\newpar{3} It's rational.

\newpar{4} $(0.125)^{-\frac{2}{3}} = (\frac{1000}{125})^{\frac{2}{3}} = 
((\frac{10}{5})^3)^{\frac{2}{3}} = 4$.  So it's an integer.

\newpar{5} A real number is a quantity $x$ who has a 
``\emph{binary expansion}'':
\[ x = n + 0.b_1b_2b_3\ldots, \mbox{ (1) }\]
where $n$ is an integer, each $b_i$ is a digit equal to zero or one and no
infinite sequence of $1$'s appears.  The representation (1) means that:
\[n + \frac{b_1}{2} + \frac{b_2}{4} + \cdots + \frac{b_k}{2^k} \le x <
n + \frac{b_1}{2} + \cdots + \frac{b_k}{2^k} + \frac{1}{2^k} \mbox{ (2) } \]

\newpar{6} Let $x = m + 0.d_1d_2\ldots$ and $y = n + 0.e_1e_2\ldots$ be real
numbers.\\
If $m < n$, then $x < y$.\\
If $m > n$, then $x > y$.\\
If $m = n$ and $d_i = e_i$ for all $i$ positive, then $x = y$.\\
Otherwise, there's an integer $k$ such that:
\[ d_i = e_i \mbox{ for all } i \le k \mbox{ and } d_k \not= e_k \]
If $d_k > e_k$ then $x > y$ otherwise $x < y$.

\newpar{7} Let $x$ and $y$ be integers.\\
Let's proove by induction relative to $y$ the relations (4).

\noindent
If $y=0$, then:
\[ b^{x+0} = b^x b^0,\]
Suppose we have:
\[ b^{x+y} = b^x b^y,\]
Then:
\begin{eqnarray*}
b^{x+y+1} & = & b^{x+y} b, \mbox{ by (3) }\\
& = & b^x b^y b, \mbox{ by induction }\\
& = & b^x b^{y+1}, \mbox{ by (3) }
\end{eqnarray*}

If $y = 0$, then:
\[ (b^x)^0 = b^{x.0} = 1, \]
Suppose we have:
\[ (b^x)^y = b^{xy} ,\]
Then:
\begin{eqnarray*}
(b^x)^{y+1} & = & (b^x)^y b^y, \mbox{ by (3) }\\
& = & b^{xy} b^x, \mbox{ by induction }\\
& = & b^{xy + x}, \mbox{ by the other relation we just proved}\\
& = & b^{x(y+1)}
\end{eqnarray*}

\newpar{8} Let $m$ be a positive integer.  If $x$ and $y$ are positive
integers, we could easily proove by induction that if $x > y$ then
$x^n > y^n$ for all positive integer $n$.  So if $u$ has a positive $m^{th}$
root, it's unique.

\medskip
Suppose $x^m = u$ and let $x_k$ be the decimal expansion of $x$.\\
Note: $x_k = d_0 + \frac{d_1}{10} + \cdots + \frac{d_k}{10^k}$, for $k \ge 0$.\\
$d_0$ is the integer verifying:
\[ d_0{}^m \le u < (d_0 + 1)^m \mbox{ and } x_0 = d_0.\]
Now suppose $d_0, d_1, \ldots, d_{k-1}$ are determined.\\
Thus $x_{k-1}$ is defined and we have:
\[ x_{k-1}{}^m \le u < \left(x_{k-1} + \frac{1}{10^{k-1}}\right)^m \]
If $u = x_{k-1}{}^m$, $x = x_{k-1}$. Otherwise:
\[ x_{k-1}{}^m < u < \left(x_{k-1} + \frac{1}{10^{k-1}}\right)^m \]
If $(x_{k-1} + \frac{9}{10^k})^m \le u$, we could take $d_k = 9$ and we have:
\begin{eqnarray*}
\left(x_{k-1} + \frac{d_k}{10^k}\right)^m \le u & < & 
\left(x_{k-1} + \frac{1}{10^{k-1}}\right)^m \\
& = & \left(x_{k-1} + \frac{d_k}{10^k} + \frac{1}{10^k}\right)^m
\end{eqnarray*}

Thus,
\[ x_k{}^m \le u < \left(x_k + \frac{1}{10^k}\right)^m \]
If $(x_{k-1} + \frac{9}{10^k})^m > u$, given that $x_{k-1}{}^m < u$, there 
exists an integer $d_k$ between $0$ and $8$ such that:
\[ \left(x_{k-1} + \frac{d_k}{10^k}\right)^m \le u < 
\left(x_{k-1} + \frac{d_k+1}{10^k}\right)^m \]
Thus,
\[ x_k{}^m \le u < \left(x_k + \frac{1}{10^k}\right)^m \]

\newpar{9} Suppose $x = \frac{p}{q}$ and $y = \frac{r}{s}$:
\begin{eqnarray*}
b^{x+y} & = & b^{\frac{p}{q} + \frac{r}{s}}\\
& = & b^{\frac{ps + qr}{qs}}
\end{eqnarray*}
Thus,
\begin{eqnarray*}
(b^{x+y})^{qs} & = & b^{ps+qr}\\
& = & b^{ps} b^{qr}\\
& = & (b^p)^s (b^r)^q\\
& = & ((b^{\frac{p}{q}})^q)^s ((b^{\frac{r}{s}})^s)^q\\
& = & (b^x)^{qs} (b^y)^{sq}\\
& = & (b^x b^y)^{qs}\\
\mbox{Thus, }b^{x+y} & = & b^x b^y
\end{eqnarray*}

We have,
\begin{eqnarray*}
(b^{xy})^{qs} & = & (b^{\frac{pr}{qs}})^{qs}\\
& = & b^{pr}\\
& = & (b^p)^r\\
& = & ((b^x)^q)^r\\
& = & (b^x)^{qr}\\
& = & ((b^x)^r)^q\\
& = & (((b^x)^y)^s)^q\\
& = & ((b^x)^y)^{qs}\\
\mbox{Thus, }b^{xy} & = & (b^x)^y
\end{eqnarray*}

\newpar{10} Suppose that $\log_{10}2$ is a rational number.\\
Note: $\log_{10}2 = \frac{p}{q}$, with $\gcd(p, q) = 1$.\\
We have:
\begin{eqnarray*}
p &=& q \log_{10}2\\
p &=& \log_{10}2^q\\
10^p &=& 2^q\\
2^q &=& 2^p 5^p\\
\end{eqnarray*}
We have a primary factor decompositon so, $p = q = 0$. Wich is absurd.
Thus $\log_{10}2$ is irrational.

\newpar{11} Infinitely many because $\log_{10}2$ is irrational.  We will
never know if $b^x$ is $1.99999\ldots$ or $2.00000\ldots$.

\newpar{12} We have: $10^{0.30102999} \le 2 < 10^{0.30103000}$.\\
Thus: $0.30102999 \le \log_{10}2 < 0.3103000$.

\newpar{13} (a) Let $x$ be a positive real number and $n$ a positive integer.\\
We have:
\begin{eqnarray*}
f(x) &=& \left(1+\frac{x}{n}\right)^n - 1 - x\\
f'(x) &=& \left(1+\frac{x}{n}\right)^{n-1} - 1 \ge 0\\
f(x) &\ge& f(0) = 0
\end{eqnarray*}

\medskip
(b)
\begin{eqnarray*}
b^{n+\frac{d_1}{10}+\cdots+\frac{d_k}{10^k}}(b^{\frac{1}{10^k}}-1) & \le &
b^{n+1}(b^{\frac{1}{10^k}} - 1)\\ &=&
b^{n+1}((1 + (b-1))^{\frac{1}{10^k}} - 1)\\ & \le &
b^{n+1}\frac{b-1}{10^k}
\end{eqnarray*}

\newpar{14} We have:
\begin{eqnarray*}
b^{\log_b(c^y)} & = & c^y, \mbox{ by definition}\\
&=& (b^{\log_bc})^y\\
&=& b^{y \log_b c}
\end{eqnarray*}
Thus, $\log_b(c^y) = y \log_b c$.

\newpar{15} Let $x$, $y$ be positive real numbers.\\
$\log_bx = \log_b(y \frac{x}{y}) = \log_by + \log_b(\frac{x}{y})$ by (10)

\newpar{16} $\log_{10} x = \frac{\ln x}{\ln 10}$.

\newpar{17} $\log_2 32 = \log_2 2^5 = 5$\\
$\log_{\pi}\pi = 1$\\
$\ln e =  1$\\
$\log_b 1 = 0$\\
$\log_b(-1)$ is not defined.

\newpar{18} $\log_8 x = \frac{\ln x}{\ln 8} = \frac{\ln x}{\ln 2^3} =
\frac{1}{3} \log_2 x$.

\newpar{19} No. The greatest $14$-digit integer is $n = 10^{15} - 1$.\\
Given that: $\log_2 n \simeq 49.83$, $n$ is a $49$-bit integer.  So it
doesn't fit.

\newpar{20} \[ \log_2 10 = \frac{\ln 10}{\ln 2} = 
\frac{1}{\frac{\ln 2}{\ln 10}} = \frac{1}{\log_{10}2}\]

\newpar{21} 
\begin{eqnarray*}
\log_b(\log_b x) &=& \log_b\left(\frac{\ln x}{\ln b}\right)\\&=&
\log_b \ln x - \log_b \ln b\\&=&
\frac{(\ln \ln x - \ln \ln b)}{\ln b}
\end{eqnarray*}

\newpar{22} $\log_2 x - \ln x - \log_{10} x = \ln x (\frac{1}{\ln 2} - 1 -
\frac{1}{\ln 10})$.\\
We have
\[\frac{1}{\ln 2} - 1 - \frac{1}{\ln 10} \simeq 0.008 \]
Thus
\[ \frac{|\log_2 x - \ln x - \log_{10} x|}{|\ln x|} < \frac{1}{100}\]

\newpar{24} We shift the decimal point of $x$ to the left or to the right 
so that we have $1 \le \frac{x}{2^n} < 2$; this determinates $n$.\\
Set $x_0 = \frac{x}{2^n}$ and for $k \ge 1$,\\
$b_k = 0, x_k = x_{k-1}{}^2,$ if $x_{k-1}{}^2 < 2$;\\
$b_k = 1, x_k = \frac{x_{k-1}{}^2}{2},$ if $x_{k-1}{}^2 \ge 2$.

\newpar{25} Note $n$ the number of precision used by the computer and suppose
$n > 1$.\\
Let's modify slightly the algorithm. We'll introduce the variable $l$ wich
counts the number of times \textbf{L4} is executed.
\begin{description}
\item[L1.]
[Initialize.] Set $y \la 0$, $z \la x \gg 1$,
$k \la 1$, $l \la 0$.
\item[L2.]
[Test for end.] If $x = 1$, stop.
\item[L3.]
[Compare.] If $x - z < 1$, goto \textbf{L5}.
\item[L4.]
[Reduce values.] Set $x \la x-z$, $z \la x \gg k$,
$l \la l+1$,
$y \la y + \log_b\left(\frac{2^k}{2^k-1}\right)$ and goto \textbf{L2}.
\item[L5.]
[Shift.] Set $z \la z \gg 1$, $k \la k+1$, and goto \textbf{L5}.
\end{description}

\noindent 
Let $x_0$ be a number veryfying $1 \le x_0 < 2$, and let $v_k$ be the value of 
$\log_b\left(\frac{2^k}{2^k-1}\right)$ on the machine and  $\eta > 0$ veryfying:
\[ \left|v_k - \log_b\left(\frac{2^k}{2^k-1}\right)\right| < \eta,
\mbox{ for } 1 \le k \le n-1 \]

\begin{description}
\item[A1.]
[\textbf{Start--L1}.] $x = x_0$.
\item[A2.]
[\textbf{L1--L2}.] $y=0$, $k=1$, $l=0$, $z = x \gg k$, $x \ge 1$, $k \le n-1$
\item[A3.]
[\textbf{L2--Stop}.] 
$|\log_bx_0-y| < (n-1)\left(\eta + 
\left|\log_b\left(1 - \frac{1}{2^{n-1}}\right)\right|\right)$
\item[A4.]
[\textbf{L2--L3}.]
$x>1$, $z = x \gg k$, $l \le k$, $k \le n-1$,\\
$|\log_bx_0 - \log_b x - y| < l\left(\eta+
\left|\log_b\left(1 - \frac{1}{2^{n-1}}\right)\right|\right)$
\item[A5.]
[\textbf{L3--L4}.]
$m = \lceil-\log_2(x-1)\rceil$, $1 \le x-z < 1+\frac{1}{2^m}$,\\
$z = x \gg k$, $k \le n-1$,\\
$k = m$ and $l < k$ or $k=m+1$ and $l < k-1$,\\
$|\log_bx_0 - \log_b x - y| < l\left(\eta+
\left|\log_b\left(1 - \frac{1}{2^{n-1}}\right)\right|\right)$
\item[A6.]
[\textbf{L3--L5}.]
$x>1$, $z = x \gg k$, $l \le k$, $k < n-1$,\\
$|\log_bx_0 - \log_b x - y| < l\left(\eta+
\left|\log_b\left(1 - \frac{1}{2^{n-1}}\right)\right|\right)$
\item[A7.]
[\textbf{L4--L2}.]
$z = x \gg k$, $k \le n-1$, $0 \le x-1 < \frac{1}{2^m}$,\\
$k = m$ and $l \le k$ or $k = m+1$ and $l \le k-1$,\\
$|\log_bx_0 - \log_b x - y| < l\left(\eta+
\left|\log_b\left(1 - \frac{1}{2^{n-1}}\right)\right|\right)$
\item[A8.]
[\textbf{L5--L2}.]
$z = x \gg k$, $x > 1$, $k \le n-1$, $l < k$,\\
$|\log_bx_0 - \log_b x - y| < l\left(\eta+
\left|\log_b\left(1 - \frac{1}{2^{n-1}}\right)\right|\right)$
\end{description}

\begin{description}
\item[A1.]
\item[A2.]
\item[A3.]
If we're from \textbf{L1} before \textbf{L2}, we have $x_0 = x = 1$, and $y=0$.
Thus $|\log_bx_0 - y| = 0$, so the inequality holds.\\
We couldn't come from \textbf{L5} because according to \textbf{A8}, $x > 1$.\\
If we're from \textbf{L4}, according to \textbf{A7}:
\[ |\log_bx_0 - \log_b x - y| < l \left(\eta +
\left|log_b\left(1 - \frac{1}{2^{n-1}}\right)\right|\right)\]
And because $x = 1$ and given that $l \le k \le n-1$, we have:
\[ |\log_bx_0 - y| < (n-1) \left(\eta +
\left|log_b\left(1 - \frac{1}{2^{n-1}}\right)\right|\right)\]
\item[A5.]
From \textbf{A4} we have, $z = x \gg k$, $k \le n-1$, and
\[ |\log_bx_0 - \log_bx - y| < l \left(\eta +
\left|log_b\left(1 - \frac{1}{2^{n-1}}\right) \right| \right)\]
And from \textbf{L3} $x - z \ge 1$.  And from \textbf{A2, A7, A8} we have
$l \le k$.\\
We have:
\[ m \ge -\log_2(x-1) > m-1, \]
\[ \frac{1}{2^m} \le x-1 < \frac{1}{2^{m-1}} \mbox{ (1)}\]and
\[ \frac{1}{2^k} \le (x \gg k) < \frac{1}{2^{k-1}}\mbox{ (2)}.\]

We then deduce that if $k < m$, \[
x-1-z = x-1-(x \gg k) < \frac{1}{2^{m-1}} - \frac{1}{2^k} \le 0,\]
thus $x-z < 1$. So we should have $k \ge m$ if we want to reach \textbf{L4}.\\
Suppose $k = m$.\\
From (1) and (2), we deduce:
\[ -\frac{1}{2^m} < x - 1 - z < \frac{1}{2^m} \]
If $x-1-z \ge 0$, we have $k=m$, and $0 \le x-1-z < \frac{1}{2^m}$.\\
From \textbf{A8} and \textbf{A2}, $l < k$ so if we have $l = k$, we're from
\textbf{A7}.\\
But according to \textbf{A7}:
\[ k = m, 0 \le x-1 < \frac{1}{2^k} \mbox{ and } z = x \gg k\]
thus, $\frac{1}{2^k} \le z$ so $x-1-z < 0$.\\
If $l = k$, we don't reach \textbf{L4} and increment $k$ at \textbf{L5}.  Thus
if $k = m$, $l < k$.

If $x-1-z < 0$, we go to \textbf{L5} and increment $k$, now $k = m+1$ and
$z = x \gg (m+1)$.  Thus, $l < k-1$.\\
From (1) and (2)
\[ x-1-(x \gg (m+1)) > \frac{1}{2^m} - \frac{1}{2^m} = 0 \] and
\begin{eqnarray*}
x-1-(x\gg(m+1)) &=& x-1 - \left(\frac{x \gg m}{2} - \frac{b}{2^n} \right),
\mbox{ with } b \in \{0, 1\}\\ &=&
x-1 - (x \gg m) + \left(\frac{x \gg m}{2} + \frac{b}{2^n}\right)
\end{eqnarray*}

We have $0 < (x \gg m) - (x-1)$. Given that the machine precision is $n$, the
smallest number we could have is $\frac{1}{2^{n-1}}$.\\
Thus, \[(x \gg m) - (x-1) \ge \frac{1}{2^{n-1}}.\]
Finally we have
\begin{eqnarray*}
x-1-(x\gg(m+1)) & \le & \frac{b}{2^n} - \frac{1}{2^{n-1}} + \frac{x \gg m}{2}\\
& < & \frac{1}{2^n} - \frac{1}{2^{n-1}} + \frac{\frac{1}{2^{m-1}}}{2},
\mbox{ from (2) } \\
& < & \frac{1}{2^m} - \frac{1}{2^n} \\
x-1-(x \gg (m+1)) & < & \frac{1}{2^m}
\end{eqnarray*}
And of course we have:
\[|\log_bx_0 - \log_bx - y| < l\left(\eta +
\left|log_b\left(1 - \frac{1}{2^{n-1}}\right) \right| \right)\]

\item[A6.]
From \textbf{A5} $x > 1$ and $k \le n-1$.  If $k = n-1$, 
$z = x \gg (n-1) = \frac{1}{2^{n-1}}$.  Given that $x > 1$, we have 
$x - 1 \ge \frac{1}{2^{n-1}}$ because the number of bits of the machine is $n$.
Thus we have: $x - z \ge 1$.\\
So if we reach \textbf{L5}, we must have $k < n-1$.

\item[A7.]
We have easily from \textbf{A5}: 
\[z = x \gg k, k \le n-1, 0 \le x-1 < \frac{1}{2^m},\]
\[ k = m \mbox{ and } l \le k \mbox{ or } k=m+1 \mbox{ and } l<k.\]
We have:
\[z = x \gg k = \frac{x}{2^k} - \frac{b_1}{2^n} - \frac{b_2}{2^{n+1}} -
\cdots - \frac{b_k}{2^{n-1+k}}\]
with $(b_1, b_2, \ldots, b_k) \in \{0, 1\}^k$.\\
Thus,
\begin{eqnarray*}
0 \le \frac{x}{2^k} - z & \le & 
\frac{1}{2^n} \left(1 + \frac{1}{2} + \cdots + \frac{1}{2^{k-1}}\right)\\&=&
\frac{1}{2^n} \frac{\left(1 - \frac{1}{2^k}\right)}{1-\frac{1}{2}}\\
0 \le \frac{x}{2^k} - z & < & \frac{1}{2^{n-1}} \mbox{ (1)}
\end{eqnarray*}
And
\begin{eqnarray*}
\log_b\left(x \left(1 - \frac{1}{2^k}\right) \right) &=&
\log_b\left((x-z) + \left(z - \frac{x}{2^k}\right) \right)\\ &=&
\log_b(x-z) + \log_b\left(1+ \frac{z-\frac{x}{2^k}}{x-z}\right) \mbox{ (*)}
\end{eqnarray*}

From (1) we have
\[0 \le \frac{\frac{x}{2^k}-z}{x-z} < \frac{1}{2^{n-1}(x-z)} 
\le \frac{1}{2^{n-1}} ,\]
Thus,
\begin{eqnarray*}
0 \le -\ln \left(1 - \frac{\frac{x}{2^k}-z}{x-z} \right)
&<& -\ln \left(1 - \frac{1}{2^{n-1}}\right)\\
\left|\ln \left(1 - \frac{\frac{x}{2^k}-z}{x-z} \right)\right|
&<& \left|\ln \left(1 - \frac{1}{2^{n-1}}\right)\right|\\
\left|\log_b \left(1 - \frac{\frac{x}{2^k}-z}{x-z} \right)\right|
&<& \left|\log_b \left(1 - \frac{1}{2^{n-1}}\right)\right|
\end{eqnarray*}

From (*) we have:
\[ \left| \log_b \left(x \left(1 - \frac{1}{2^k}\right)\right) -
\log_b(x-z) \right| < \left| \log_b\left(1 - \frac{1}{2^{n-1}}
\right) \right| \mbox{ (2)}\]

From \textbf{A5} we have:
\[ |\log_bx_0 - \log_bx - y| < l \left(\eta +
\left|\ln \left(1 - \frac{1}{2^{n-1}}\right) \right| \right) \]

Remembering that:
\[ \left|v_k - \log_b\left(\frac{2^k}{2^k-1}\right) \right| < \eta, 
\mbox{ for } 1 \le k \le n-1 \mbox{ (3)}\]

Finally, we have
\begin{eqnarray*}
|\log_bx_0 - \log_b(x-z) & - & (y+v_k)|  \le \\ &&
|\log_bx_0 - \log_bx - y| +\\ &&
\left|\log_b\left(\frac{2^k}{2^k-1}\right)-v_k\right|+\\&&
\left|\log_b\left[x \left(1 - \frac{1}{2^k}\right) \right] - \log_b(x-z)
\right|
\end{eqnarray*}
From (1), (2), (3) we deduce
\begin{eqnarray*}
|\log_bx_0 - \log_b(x-z) - (y+v_k)|  & \le &
l\left(\eta + \left|\log_b\left(1 - \frac{1}{2^{n-1}}\right)\right|\right)+\\&&
\eta + \left|\log_b\left(1 - \frac{1}{2^{n-1}}\right)\right|\\& \le & 
(l+1)\left(\eta + \left|\log_b\left(1 - \frac{1}{2^{n-1}}\right)\right|\right).
\end{eqnarray*}

\item[A8.]
\end{description}

Let's show that the algorithm terminates.  We have $k \le n-1$. From \textbf{A5}
we have $k \ge \lceil -\log_2(x-1)\rceil$.\\
Until $k$ reach $m = \lceil -\log_2(x-1)\rceil$, $k$ is incremented.\\
If $k = m$, and $x-z \ge 1$, from \textbf{A7} we have after \textbf{L4},
$x-z < 1 + \frac{1}{2^m} - \frac{1}{2^m} <1$.  Thus $k$ is incremented in 
the next iteration.\\
And if $x-z<1$, $k$ is incremented.

So $k$ is incremented at least each $2$ iterations.  And given that
$k \le n-1$, the algorithm terminates.

\newpar{26} From the demonstration of \textbf{A3} in \textbf{25}, we deduce
that the maximum value of error corresponds to the maximum value of $l$.

To simplify the notation, let's say the numbers we deal with are integers
between $0$ and $2^{n-1}$ instead of $0$ and $2 - \frac{1}{2^{n-1}}$, where
$n$ is the precision of the machine.


Since we're only dealing with addition, soustraction and shifting, we just
mutliply the numbers by $2^{n-1}$.  We then have $2^{n-1} \le x_0 < 2^n$.
So let's rewrite the algorithm like this:
\begin{description}
\item[L1.]
[Initialize.] Set $y \la 0$, $z \la x \gg 1$,
$k \la 1$, $l \la 0$.
\item[L2.]
[Test for end.] If $x = 2^{n-1}$, stop.
\item[L3.]
[Compare.] If $x - z < 2^{n-1}$, goto \textbf{L5}.
\item[L4.]
[Reduce values.] Set $x \la x-z$, $z \la x \gg k$,
$l \la l+1$,
$y \la y + \log_b\left(\frac{2^k}{2^k-1}\right)$ and goto \textbf{L2}.
\item[L5.]
[Shift.] Set $z \la z \gg 1$, $k \la k+1$, and goto \textbf{L5}.
\end{description}
Note: $m = \lfloor \frac{n}{2}\rfloor$.\\
Let $k$ be an integer such that $1 < k < m$ and
\[f_k(y) = y - (y \gg k), \mbox{ for } 2^{n-1} < y < 2^{n}.\]
And note that: $y \gg k = \lfloor \frac{y}{2^k}\rfloor$.

\medskip
Let's show that for every $z$ such that $0 \le z < 2^{n-2k-1}$, there exists an
integer $z' < 2^{n-2k+1}$ if $k > 2$ and $z'<2^{n-2k+2}$ if $k=2$ such that:
\[ f_k(2^{n-1} + 2^{n-k} + z') = 2^{n-1} + 2^{n-k-1} + z \mbox{ (*)}\]

Note: $y_0 = 2^{n-1} + \sum_{i=1}^r 2^{n-lk},$ with 
$r=\lfloor \frac{n}{k}\rfloor$.  We have 
\[f_k(y_0) = 2^{n-1} + 2^{n-k-1}\]

\begin{eqnarray*}
(a+b) \gg k & = & \left\lfloor \frac{a+b}{2^k}\right\rfloor\\ & = &
\left\lfloor \frac{2^k\lfloor\frac{a}{2^k}\rfloor + a\ \mathrm{mod}\ 2^k + 
2^k\lfloor\frac{b}{2^k}\rfloor + b\ \mathrm{mod}\ 2^k}{2^k}\right\rfloor\\ &=&
\left\lfloor\frac{a}{2^k}\right\rfloor + 
\left\lfloor\frac{b}{2^k}\right\rfloor +
\left\lfloor\frac{a\ \mathrm{mod}\ 2^k + b\ \mathrm{mod}\ 2^k}{2^k}\right\rfloor
\mbox{ (1)}
\end{eqnarray*}

We have
\[ (n-rk) - k = n - (r+1)k < n - \frac{n}{k} k = 0\]
\[ (n - (r-1)k) - k = n - rk \ge 0 \]

So 
\[2^{n-rk} < 2^k \mbox{ and } 2^{n-ik} \ge 2^k\mbox{ if }i \le r-1.\]
 Thus,
\[y_0\ \mathrm{mod}\ 2^k = 2^{n-rk}\] and from (1)
\[(y_0 + b) \gg k = \left\lfloor\frac{y_0}{2^k}\right\rfloor +
\left\lfloor\frac{b}{2^k}\right\rfloor +
\left\lfloor\frac{2^{n-rk} + b\ \mathrm{mod}\ 2^k}{2^k}\right\rfloor
\mbox{ (2)}\]

If $0 \le b < 2^k - 2^{n-rk}$,
\[\left\lfloor\frac{b}{2^k}\right\rfloor = 0 \mbox{ and }
\left\lfloor\frac{2^{n-rk}+b\ \mathrm{mod}\ 2^k}{2^k}\right\rfloor =
\left\lfloor\frac{2^{n-rk}+b}{2^k}\right\rfloor = 0.\]

So,
\[f_k(y_0 + b) = f_k(y_0) + b = 2^{n-1} + 2^{n-k-1} + b.\]

\begin{itemize}
\item
If we have $2^{n-2k-1}\le 2^k - 2^{n-rk}$, then $z < 2^k - 2^{n-rk}$, so 
\[ f_k(y_0 + z) = 2^{n-1} + 2^{n-k-1} + z\]
Note: $z' = \sum_{i=2}^r 2^{n-ik} + z$, so $y_0 + z = 2^{n-1}+2^{n-1}+z'$.\\
We have $2^{n-2k-1} \le 2^{k}-2^{n-rk}$, so $n-2k-1 \le k-1$, then $r\le3$.
\begin{description}
\item[*] If $r=2$, $z' = 2^{n-2k} + z < 2^{n-2k} + 2^{n-2k-1} < 2^{n-2k+1}$.
\item[*] If $r=3$, $z < 2^k - 2^{n-3k} \le 2^{n-2k} - 2^{n-3k}$ so,\\
$z' < 2^{n-2k} + 2^{n-2k} = 2^{n-2k+1}$.
\end{description}
So we have (*).

\item
From now on then, let's suppose that $2^k - 2^{n-rk} < 2^{n-2k-1}$ and\\
$z \ge 2^k - 2^{n-rk}$.  We have $k \le n-2k-1$ so $r \ge 3$.

Note: $y_1 = y_0 + 2^k - 2^{n-rk} = 2^{n-1} + \sum_{i=1}^{r-1}2^{n-ik} + 2^k$.\\
From (2), we have
\[ (y_0 + 2^k - 2^{n-rk}) \gg k = \left\lfloor\frac{y_0}{2^k}\right\rfloor + 1\]
and
\begin{eqnarray*}
f_k(y_1)& = &f_k(y_0) + 2^k - 2^{n-rk} - 1\\ &=&
2^{n-1} + 2^{n-k-1} + 2^k - 2^{n-rk} - 1.
\end{eqnarray*}

If $1 \le i \le r-1$, $n-ik \ge k$, so $y_1\ \mathrm{mod}\ 2^k = 0$. And from
(1)
\[(y_1+b) \gg k = \left\lfloor\frac{y_1}{2^k}\right\rfloor +
\left\lfloor\frac{b}{2^k}\right\rfloor\]
\[f_k(y_1+b) = f_k(y_1) + b - \left\lfloor\frac{b}{2^k}\right\rfloor \mbox{ (3)}
\]

Note: $\delta = z - 2^k + 2^{n-rk} + 1$.  Remember that $z \ge 2^k - 2^{n-rk}$ 
so $\delta > 0$.

\begin{description}
\item[*]
If $\delta < 2^k$, from (3) we have
\[f_k(y_1 + \delta) = f_k(y_1) + \delta = 2^{n-1} + 2^{n-k-1} + z,\]
and if we note: $z' = \sum_{i=2}^r2^{n-ik} + z + 1$, we have
\[ y_1 + \delta = 2^{n-1} + 2^{n-k} + z'.\]
Given that $z < 2^{n-2k-1}$, we have
\[ z' < \sum_{i=2}^r 2^{n-ik} + 2^{n-2k-1} + 1,\]
Given that $k > 1$, $n-2k > n-2k-1 > n-3k$, so
\[ \sum_{i=2}^{r-1}2^{n-ik} + 2^{n-2k-1} \le \sum_{i=2k}^{(r-1)k}2^{n-i},\]
and
\[ 1+2^{n-rk} \le 2^{n-(r-1)k} - 1\]
so we have prooved (*) because
\[z' < 2^{n-2k+1}\]
\item[*]
If $\delta \ge 2^k$, from (3) we have
\begin{eqnarray*}
f_k\left(y_1 + \delta + \left\lfloor\frac{\delta}{2^k-1}\right\rfloor\right)&=&
f_k(y_1) + \delta + \left\lfloor\frac{\delta}{2^k-1}\right\rfloor - \\&&
\left\lfloor\frac{\delta + \left\lfloor
\frac{\delta}{2^k-1}\right\rfloor}{2^k}\right\rfloor \mbox{ (**)}
\end{eqnarray*}
If $p$ is a positive integer and $x$ a real, let's show that
\[ \left\lfloor\frac{\lfloor x \rfloor}{p}\right\rfloor =
\left\lfloor\frac{x}{p}\right\rfloor\]

We have,
\[\left\lfloor\frac{\lfloor x \rfloor}{p}\right\rfloor \le
\frac{\lfloor x \rfloor}{p} \le \frac{x}{p}\] and

\[p\left(\left\lfloor\frac{\lfloor x \rfloor}{p}\right\rfloor + 1\right) >
p \frac{\lfloor x \rfloor}{p} = \lfloor x \rfloor\]

Given that the two members of the inequality are integers, we have
\[p\left(\left\lfloor\frac{\lfloor x \rfloor}{p}\right\rfloor + 1\right) \ge
\lfloor x \rfloor + 1 > x\]
thus,
\[\left\lfloor\frac{\lfloor x \rfloor}{p}\right\rfloor \le \frac{x}{p} <
\left\lfloor\frac{\lfloor x \rfloor}{p}\right\rfloor + 1\]

We then deduce that
\begin{eqnarray*}
\left\lfloor\frac{\delta + \left\lfloor\frac{\delta}{2^k-1}\right\rfloor}{2^k}
\right\rfloor & = &
\left\lfloor\frac{\left\lfloor \delta + \frac{\delta}{2^k-1}\right\rfloor}{2^k}
\right\rfloor\\ & = &
\left\lfloor\frac{\left\lfloor \frac{2^k\delta}{2^k-1}\right\rfloor}{2^k}
\right\rfloor\\ & = &
\left\lfloor\frac{\delta}{2^k-1}\right\rfloor
\end{eqnarray*}

From (**), we have
\[f_k\left(y_1 + \delta + \left\lfloor\frac{\delta}{2^k-1}\right\rfloor\right)
= f_k(y_1) + \delta = 2^{n-1} + 2^{n-k-1} + z.\]
Note: $z' = \sum_{i=2}^r 2^{n-ik} + z + 1 + \left\lfloor\frac{\delta}{2^k-1}
\right\rfloor$, we have
\[ y_1 + \delta = 2^{n-1} + 2^{n-k} + z'.\]
We have, $z < 2^{n-2k-1}$ and $k > 1$ so
\begin{eqnarray*}
\left\lfloor\frac{\delta}{2^k-1}\right\rfloor & \le &
\frac{\delta}{2^k-1}\\ & \le &
\frac{z}{2^k-1}\\ & \le &
\frac{z}{2^{k-1}} \\ & < &
\frac{2^{n-2k-1}}{2^{k-1}} \\
\left\lfloor\frac{\delta}{2^k-1}\right\rfloor & \le & 2^{n-3k}-1 
\end{eqnarray*}

Remember that $r \ge 3$, so we have
\begin{eqnarray*}
z' & \le & \sum_{i=2}^r2^{n-ik} + z + 2^{n-3k}\\
& < & \sum_{i=2}^r 2^{n-ik} + 2^{n-2k-1} + 2^{n-3k} \\
& = & 2^{n-2k} + 2^{n-2k-1} + 2^{n-3k+1} + \sum_{i=4}^r 2^{n-ik}
\end{eqnarray*}

If $k > 2$, $n-2k-1 > n-3k+1$ so
\[z' < 2^{n-2k+1},\]

If $k = 2$,
\[z' < 2^{n-3} + \sum_{i=4}^r 2^{n-ik} < 2^{n-2}.\]

So we have demonstrated (*).
\end{description}
\end{itemize}

So far, we've showed that for $2 \le k \le m-1$, and for each integer
$z < 2^{n-2k-1}$, there exists an $z' < 2^{n-2k+1}$ if $k > 2$ and
$z' < 2^{n-2}$ if $k = 2$ such that:
\[(2^{n-1} + 2^{n-k} + z') - ((2^{n-1} + 2^{n-k} + z') \gg k) =
2^{n-1} + 2^{n-1} + z \mbox{ (*) }\]

Note: $x_0 = 2^{n-1} + 2^{n-m} + \lfloor 2^{n-2m-1}\rfloor$.

We can show by induction and thanks to (*) that there exists a sequel
$(x_i)_{0 \le i \le m-2}$ such that:
\[x_i = 2^{n-1} + 2^{n-m+i} + z_i,\]
with $z_{m-2}< 2^{n-2}$ and $z_i < 2^{n-2(m-i)+1}$ if $i < m-2$ and
\[x_i - (x_i \gg (m-i)) = x_i-1, \mbox{ for } 1 \le i \le m-2 \mbox{ (4)}\]

Plus, for $0 \le i \le m-2$
\[(x_i \gg (m-i-1)) \ge 2^{n-m+i} + 2^{n-2(m-i)+1}, \mbox{ because } n-2m+1 > 0,\]
so,
\[x_i - (x_i \gg (m-i-1)) < 2^{n-1} \mbox { (5).}\]
Note: $p = \left\lfloor\frac{n+1}{2}\right\rfloor$.

Suppose: $x = 2^{n-1} + 2^{n-1-k} + z$, with $k \ge p$, and $z < 2^{n-1-k}$.\\
Then, $x \gg k = 2^{n-1-k}$ because:
\[n-1-2k \le n-1-2p < n- 1 - 2(\frac{n+1}{2} - 1) = 0\] so,
\[x - (x \gg k) = 2^{n-1} + z \mbox{ (6)}\]

Let's say that we run the algorithm on some input, and $N$ is the value of
$l$ when $x$ becomes less than $2^{n-1} + 2^{n-p}$.  If we note $b$ the
number of bits equal to $1$ in $x$ ($b \ge 1$, because $x \ge 2^{n-1}$), from
(6) we deduce that the value of $l$ when the algorithm terminates is
$N + b - 1$. (7)

We've just executed \textbf{L4} so from 
\textbf{A7} we have,
\[\frac{x}{2^{n-1}} - 1 < \frac{1}{2^m}\]
or rearranging the inequality
\[x < 2^{n-1} + 2^{n-1-m}.\]
 We deduce too from \textbf{A7},
\[k = m \mbox{ and } N \le k \mbox{ or } k = m+1 \mbox{ and } N \le k-1,\]
Thus we have,
\[N \le m \mbox{ and } x < 2^{n-1} + 2^{n-1-m}. \mbox{ (8)}\]

We have $b \le n-p+1$.
\begin{description}
\item[*] If $b < n-p+1$,\\
Replacing $m$ by $p-1$ in (8), we have
\[x < 2^{n-1} + 2^{n-p} \mbox{ and } N \le p-1.\]
So from (7), we have
\[l \le N + b - 1 \le n-2.\]
\item[*]If $b = n-p+1$:\\
The current value of $x$ is then $2^{n-1} + 2^{n-p} - 1$.
Let $y$ be an integer verifying
\[2^{n-1} + 2^{n-p} \le y < 2^{n-1} + 2^{n-p+1}.\]
Let's show that $y$ couldn't verify
\[\left\{\begin{array}{l}
y - (y \gg (p-1)) = x\\
or\\
y - (y \gg (p-1)) < 2^{n-1} \mbox{ and } y - (y \gg p) = x.
\end{array}\right. \]

We have
\[(y \gg (p-1)) \ge 2^{n-p} + 2^{n-2p+1}\] because
$n-2p+1 \ge n - 2\frac{n+1}{2}+1 = 0$.  So,
\begin{eqnarray*}
y - (y \gg (p-1)) &<& 2^{n-1} + 2^{n-p+1} - 2^{n-p} - 2^{n-2p+1}\\
&\le& 2^{n-1} + 2^{n-p} - 1\\
\mbox{Thus, }y - (y \gg (p-1)) &<& x.
\end{eqnarray*}

Suppose that
\begin{eqnarray*}
y &<& 2^{n-1} + (y >> (p-1))\\
&\le& 2^{n-1} + 2^{n-p} + 2^{n-2p+1},
\end{eqnarray*}
we then deduce
\begin{eqnarray*}
y - (y \gg p) &<& 2^{n-1} + 2^{n-p} + 2^{n-2p+1} - 2^{n-1-p}\\
&=& 2^{n-1} + 2^{n-1-p} + 2^{n-2p+1}.
\end{eqnarray*}

We have,
\begin{eqnarray*}
2^{n-1} + 2^{n-1-p} + 2^{n-2p+1} \le x & \Leftrightarrow &
2^{n-2p+1} \le 2^{n-1-p} - 1\\ & \Leftrightarrow &
n-2p+1 \le n-2-p\\ & \Leftrightarrow &
p \ge 3.
\end{eqnarray*}

Let's suppose $n \ge 5$.  So $p \ge 3$ and $y - (y \gg p) < x$.  Thus, if $y$
is the previous value of $x$, we must have
\[ 2^{n-1} + 2^{n-p+1} \le y,\]
\[ 1 + 2^{-p+2} \le \frac{y}{2^{n-1}}\]
\[ p-2 \ge \left\lceil \frac{y}{2^{n-1}} - 1 \right\rceil.\]
Given that the value of $m$ in (8) is $\left\lceil \frac{y}{2^{n-1}} - 1 
\right\rceil$, we conclude 
\[ N \le m \le p-2.\]
And from (7)
\[l \le N + b - 1 \le n-2.\] 
\end{description}

To summarize, if $n \ge 5$ then $l \le n-2$.  

Let's show that it's a maximum.  From (4) and (5), we could deduce that if we 
take as input $x_{m-2}$ for the algorithm, the sequel $x_{m-2}, x_{m-3}, 
\ldots, x_0$ are the successive values taken by $x$.\\
We have: $x_0 = 2^{n-1} + 2^{n-m} + \lfloor 2^{n-2m-1} \rfloor$.  From (5),
\[x_0 - (x_0 \gg (m-1)) < 2^{n-1}.\]And given that
\[x_0 \gg m = 2^{n-1-m} + 2^{n-2m},\]we have the next value of $x$, 
\begin{eqnarray*}
x = x_0 - (x_0 \gg m) & = &
2^{n-1} + 2^{n-1-m} - 2^{n-2m} + \lfloor 2^{n-2m-1}\rfloor\\ & = &
2^{n-1} + 2^{n-1-m} - 1.
\end{eqnarray*}
 
We have: $m+1 = \left\lfloor\frac{n}{2}\right\rfloor + 1 =
\left\lfloor\frac{n+2}{2}\right\rfloor \ge p$.  So from (7), we deduce that
the value of $l$ at the end of the algorithm is:
\[ (m-1) + (n-m) - 1 = n-2.\]

Thus, if $n \ge 5$, the maximum value of $l$ is $n-2$.  And we could verify
that if $n < 5$, this value is $n-1$.

\newpar{27} We have,
\[x(1 - \eta) \le 10^n x_0' \le x(1 + \epsilon)\]
\[{x'}_{k-1}{}^2(1 - \eta) \le y_k \le {x'}_{k-1}{}^2(1 + \epsilon).\]

Note: $N_k = n + \frac{b_1}{2} + \frac{b_2}{4} + \cdots + \frac{b_k}{2^k}$.
Let's show by induction that:
\[\left(\frac{x}{10^{N_{k-1}}}\right)^{2^k}(1 - \eta)^{2^{k+1}-1} \le
y_k \le \left(\frac{x}{10^{N_{k-1}}}\right)^{2^k}(1 + \epsilon)^{2^{k+1} - 1}\] and
\[\log_{10}x  + 2 \log_{10}(1 - \eta) - \frac{1}{2^{k-1}} < N_{k-1} <
\log_{10}x + 2 \log_{10}(1 + \epsilon)\] for $k \ge 1$.

\begin{description}
\item[(1)] If $k = 1$, we have:
\[\frac{x}{10^n}(1-\eta) \le {x'}_0 \le \frac{x}{10^n}(1 + \epsilon)\] so

\[\left(\frac{x}{10^n}\right)^2(1-\eta)^3 \le {x'}_0{}^2(1-\eta) \le y_1
\le {x'}_0{}^2(1+\epsilon)\le \left(\frac{x}{10^n}\right)^2(1+\epsilon)^3.\]

$N_0=n$ and $10^n \le x < 10^{n+1}$.  So,
\[x(1-\eta)^2 < 10^{n+1} \mbox{ and } 10^n < x(1+\epsilon)^2,\] thus,
\[\log_{10}x + 2\log_{10}(1-\eta) < N_0 + 1 \mbox{ and }
N_0 < \log_{10} x + 2 \log_{10}(1 + \epsilon).\]

\item[(2)] Suppose the proprieties are verified for $k$.  We have
${x'}_k = \frac{y_k}{10^{b_k}}$.  So by hypothesis:
\[\left(\frac{x}{10^{N_k}}\right)^{2^k}(1-\eta)^{2^{k+1}-1} \le {x'}_k \le
\left(\frac{x}{10^{N_k}}\right)^{2^k}(1+\epsilon)^{2^{k+1}-1},\]

and raising the inequalities to the power of two, we have:
\[\left(\frac{x}{10^{N_k}}\right)^{2^{k+1}}(1-\eta)^{2^{k+2}-2} \le {x'}_k{}^2 \le
\left(\frac{x}{10^{N_k}}\right)^{2^{k+1}}(1+\epsilon)^{2^{k+2}-2}.\] Thus,
\[\left(\frac{x}{10^{N_k}}\right)^{2^{k+1}}(1-\eta)^{2^{k+2}-1} \le
{x'}_k{}^2(1-\eta) \le y_{k+1},\] and
\[y_{k+1} \le {x'}_k{}^2(1+\epsilon) \le
\left(\frac{x}{10^{N_k}}\right)^{2^{k+1}}(1+\epsilon)^{2^{k+2}-1}.\]

So we have the desired result.  On the other hand, suppose $y_k \ge 10$, thus
$b_k = 1$.  We have by hypothesis:
\[ 10 \le \left(\frac{x}{10^{N_{k-1}}}\right)^{2^k}(1+\epsilon)^{2^{k+1}-1}.\]
Taking the logarithm we obtain:
\[ 1 \le 2^k(\log_{10}x - N_{k-1}) + (2^{k+1}-1)\log_{10}(1+\epsilon)\]
and dividing by $2^k$
\begin{eqnarray*}
N_k = N_{k-1} + \frac{1}{2^k} & \le & \log_{10}x + (2 - \frac{1}{2^k})
\log_{10}(1+\epsilon)\\
& < & \log_{10}x + 2 \log_{10}(1+\epsilon).
\end{eqnarray*}

By hypothesis: $\log_{10}x + 2 \log_{10}(1-\eta) - \frac{1}{2^{k-1}} < N_{k-1}$.
Thus,
\[\log_{10}x + 2 \log_{10}(1-\eta) - \frac{1}{2^k} < N_{k-1} + \frac{1}{2^k}
= N_k.\]

\medskip If $y_k < 10$, then $b_k = 0$.  We have by hypothesis:
\[\left(\frac{x}{10^{N_{k-1}}}\right)^{2^k}(1-\eta)^{2^{k+1}-1} < 10,\]
thus taking the logarithm and dividing by $2^k$ we have
\[\log_{10} x - N_{k-1} + \left(2 - \frac{1}{2^k}\right)\log_{10}(1-\eta)
< \frac{1}{2^k}.\]
$\log_{10}(1-\eta) < 0$, so we have:
\[\log_{10}x + 2 \log_{10}(1-\eta) - \frac{1}{2^k} < N_{k-1} = N_k.\]
And by hypothesis:
\[ N_k = N_{k-1} < \log_{10} x + 2 \log_{10}(1 + \epsilon).\]
\end{description}

\newpar{28} Let $n > 1$ be the precision of the machine in bits.  We're using
fixed-point arithmetic here.  
\begin{description}
\item[F1.]
[Initialize.] Set $y \la 1$, $k \la 0$.
\item[F2.]
[Test for and end.] If $x = 0$, stop.  $y$ is the result.
\item[F3.]
[Compare.] If $x < \log_b\left(1 + \frac{1}{2^k}\right)$, go to \textbf{L5}.
\item[F4.]
[Reduce values.] Set $x \la x - \log_b\left(1 + \frac{1}{2^k}\right)$,
$y \la y + (y \gg k)$, and go to \textbf{L2}.
\item[F5.]
[Increment.] $k \la k + 1$, and go to \textbf{L2}. $|$
\end{description} Note: $\epsilon = \frac{1}{2^n}$.  
And for $0 \le y < \epsilon$, let $r$ be the function defined by
\[ r(y) = \left\{ \begin{array}{ll}
	0 & \mbox{ if } 0 \le y < \frac{\epsilon}{2};\\
	\epsilon & \mbox{ if } \frac{\epsilon}{2} \le y \le \epsilon.
        \end{array} \right. \]
For $0 \le y < 1$, let $M$ be the function defined by,
\[ M(y) = \left\lfloor \frac{y}{\epsilon} \right\rfloor \epsilon +
	r(y\; \mathrm{mod}\; \epsilon).\]
We have
\begin{equation}
\left| y - M(y) \right| =
\left|y\; \mathrm{mod}\; \epsilon - r(y\; \mathrm{mod}\; \epsilon)\right|
\le  \frac{\epsilon}{2} 
\end{equation}

Let $M(y)$ be the approximation of $y$ on the machine.

Let $x_0$ be the initial value of $x$ and suppose the algorithm terminates
for this input.  Suppose \textbf{F4} is executed $N$ time(s) and 
$(k_i)_{1 \le i \le N}$ is the sequel of values taken by $k$ when \textbf{F4}
is executed.

Note: $M\left(\log_b\left(1 + \frac{1}{2^{k_i}}\right)\right) =
\log_b\left(1 + \frac{1}{2^{k_i}}\right) + \delta_i$ and
$M(y \gg k) = \frac{y}{2^{k_i}} (1 + \epsilon_i)$.

\textbf{F4} is thus replaced by
\[ x \la x - \log_b\left(1 + \frac{1}{2^{k_i}}\right) - \delta_i \mbox{ and }
y \la y\left(1 + \frac{1}{2^{k_i}}\right)(1 + \epsilon_i).\]
We could show that at the end of the algorithm,
\[ y = b^{x_0 - \sum_{i = 1}^N \delta_i} \prod_{i=1}^N (1 + \epsilon_i).\]

\medskip
Let's show that the algorithm terminates for all input 
$x_0 \in [0, 1[$.  We can see easily that as long as there exists an 
integer $k$ such that 
\[0 < M\left(\log_b\left(1 + \frac{1}{2^{k}}\right)\right) \le x,\]
 the value of $x$ will decrease.


Let's suppose that
\[ x < M\left(\log_b\left(1 + \frac{1}{2^{k}}\right)\right) \mbox{ and }
M\left(\log_b\left(1 + \frac{1}{2^{k+1}}\right)\right) = 0.\]
Given that $M$ is the sum of two positive functions, we deduce
\[ \left\lfloor\frac{\log_b\left(1 + \frac{1}{2^{k+1}}\right)}{\epsilon}
\right\rfloor = 0 \mbox{ and }
r\left(\log_b\left(1 + \frac{1}{2^{k+1}}\right)\; \mathrm{mod} \; \epsilon
\right) = 0\]
thus
\[ \log_b\left(1 + \frac{1}{2^{k+1}}\right) < \epsilon \mbox{ and }
r\left(\log_b\left(1 + \frac{1}{2^{k+1}}\right)\right) = 0\]
and finally
\begin{equation}
0 \le \log_b\left(1 + \frac{1}{2^{k+1}}\right) < \frac{\epsilon}{2}.
\end{equation}
From (1), we have
\begin{eqnarray*}
M\left(\log_b\left(1 + \frac{1}{2^k}\right)\right) & \le &
\log_b\left(1 + \frac{1}{2^k}\right) + \frac{\epsilon}{2} \\ & < &
2 \log_b\left(1 + \frac{1}{2^{k+1}}\right) + \frac{\epsilon}{2} \\ & < &
\epsilon + \frac{\epsilon}{2},\; \mbox{ from (2)}
\end{eqnarray*}

$M\left(\log_b\left(1 + \frac{1}{2^k}\right)\right) > 0$ so we have
$M\left(\log_b\left(1 + \frac{1}{2^k}\right)\right) \ge \epsilon$, thus
\[ 0 \le M\left(\log_b\left(1 + \frac{1}{2^k}\right)\right)
- \epsilon < \frac{\epsilon}{2}.\]

The middle member of the inequalities is a number on the machine, and the unique
number on the machine less than $\frac{\epsilon}{2}$ is $0$. Thus,
\[ M\left(\log_b\left(1 + \frac{1}{2^k}\right)\right) = \epsilon.\]
So we deduce $x < \epsilon$, thus $x = 0$ and the algorithm terminates.

\newpar{29}  Let $x$ be a real number greater than $1$.

\noindent
(a) Note: $f(b) = \frac{b}{\ln b}$ for $b > 1$. We have,
\[f'(b) = \frac{\ln b - 1}{\ln b}.\]
So we deduce that $f(b) \ge f(e) = e$.  Thus $b \log_bx \ge e \ln x$.

\medskip \noindent
(b) We have: $2 < e < 3$.  And $\frac{2}{\ln 2} \simeq 2.88$ and
$\frac{3}{\ln 3} \simeq 2.73$. Thus,
\[ \frac{3}{\ln 3} < \frac{2}{\ln 2}.\]

We then deduce that $b \log_bx$ is a \emph{minimum} for the integer $b = 3$.

\medskip \noindent
(c) Note: $g(b) = \frac{b+1}{\ln b}$ for $b > 1$. We have,
\[ g'(b) = \frac{\ln b - 1 - \frac{1}{b}}{(\ln b)^2}.\]
If we note: $h(b) = \ln b - 1 - \frac{1}{b}$. Then,
\[ h'(b) = \frac{1}{b} - \frac{1}{b^2} \ge 0, \mbox{ for } b \ge 1.\]
Given that $h(1) = -2$ and $\lim_{b \rightarrow +\infty}h(b) = +\infty$, there
exists $b_0 \in \rbrack -1, +\infty\rbrack$ such that,
\[ h(b) \le 0 \mbox{ if } b \le b_0 \mbox{ and } h(b) > 0 \mbox{ if } b > b_0.\]
We then deduce that $b_0$ is a minimum for $g$.

Plus, $g'(3) < 0$ and $g'(4) > 0$,  so $3 < b_0 < 4$.
$g(3) \simeq 3.64$ and $g(4) \simeq 3.61$.

We then deduce that the minimum of $(b+1)\log_bx$ for $b$ an integer
greater than $1$ is $5 \log_4x$.
\end{document}
