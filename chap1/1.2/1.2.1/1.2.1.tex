\documentclass[a4paper,12pt]{article}
\begin{document}

\noindent
\textbf{1.} We prove the statement $Q(n)$ wich corresponds to the statement 
$P(n-1)$ for all integer $n$.

\bigskip
\noindent
\textbf{2.} In the proof by induction, we that $a^{n-2} = 1$ wich is not true
because the propriety is verified only for n = 1.

\bigskip
\noindent
\textbf{3.} The left side of the equality is defined only for $n > 1$.

\bigskip
\noindent
\textbf{4.} If $n = 1$, $F_1 = 1 \ge \phi^{-1}$\\
If $n = 2$, $F_2 = 1 \ge \phi^0$

\medskip
Suppose $F_n \ge F_{n-2}$ for $n \ge 2$, then we have:
\begin{eqnarray*}
F_{n+1} = F_n + F_{n-1} & \ge & \phi^{n-2} + \phi^{n-3} \\ & = &
\phi^{n-2} (1 + \phi) = \phi^{n-2} \phi^2 = \phi^n
\end{eqnarray*}

\bigskip
\noindent
\textbf{5.} If $n = 2$, $2$ is prime so it's a product of prime numbers.

\medskip
Suppose that for $k$ between $2$ and $n$, $k$ could be written as product
of prime numbers.

If $n+1$ is a prime, we could easily conclude.\\
Suppose $n+1$ is not a prime.  There exist two integers greater than $1$,
$a$ and $b$ such that: $n+1 = a b$.

\medskip
Because $a > 1$ and $b > 1$, we must have $a < n+1$ and $b < n+1$.\\
By induction, a and b could be expressed as products of prime numbers, so 
$n+1$ could also be expressed as such.

\bigskip
\noindent
\textbf{6.} Suppose we have: $a' m + b' n = c$, $a m + b n = d$ before
the execution of \textbf{E4}. From now on, we append the indice $1$ to 
the variables before \textbf{E4} is executed and the indice $2$ after
it was.
\[ a'_2 m + b'_2 n = a_1 m + b_1 n = d_1 = c_2 \] and
\begin{eqnarray*}
a_2 m + b_2 n & = & (a'_1 - q a_1) m + (b'_1 - q b_1) n\\
& = & (a'_1 m + b'_1 n) - q (a_1 m + b_1 n)\\
& = & c_1 - q d_1 = r = d_2
\end{eqnarray*}

\bigskip
\noindent
\textbf{7.} We have for $n \ge 0$: $S_n = \sum_{k = 0}^{n} (-1)^{n-k} k^2$.\\
So we deduce for each positive integer $n$: $S_n = n^2 - S_{n-1}$.\\
Note: $R_n = S_{2n}$ for $n > 0$.\\
We have: $R_n = R_{n-1} + (2n)^2 - (2n-1)^2 = R_{n-1} + 4 n - 1$\\
Thus, $R_n = R_0 + n(2n+1) = n(2n+1)$

\begin{eqnarray*}
S_{2n} & = & \frac{(2n)(2n+1)}{2}\\
S_{2n+1} & = & (2n+1)^2 - S_{2n} = (2n+1)^2 - \frac{(2n+1)(2n)}{2}\\
& = & \frac{(2n+1)(2n+2)}{2}
\end{eqnarray*}
So $S_n = \frac{n(n+1)}{2}$ for all integer $n \ge 0$.

\bigskip
\noindent
\textbf{8.} (a) For $n > 0$:
\[n^3 = k + (k+2) + (k+4) + \cdots + (k + 2(n-1)) \] with
\[k = 2 (1 + 2 + \cdots + (n-1)) + 1 = n(n-1) + 1 \]
Rewriting $n^3$, we have
\[n^3 = \sum_{i = 0}^{n-1}(n(n-1) + 2 i + 1) \]

\medskip
\noindent
If $n = 1$, $1^3 = 1$\\
Suppose we have, $n^3 = \sum_{i = 0}^{n-1}(n(n-1)+1+2i)$\\
Thus,
\begin{eqnarray*}
\sum_{i=0}^n((n+1)n+1+2i) & = &
n(n+3)+1 + \sum_{i=0}^{n-1}(n(n-1)+1+2i + 2n) \\ & = &
n(n+3)+1 + 2n^2 + \sum_{i=0}^{n-1}(n(n-1)+1+2i) \\ & = &
3n^2 + 3n + 1 + n^3\mbox{, by induction}\\ & = &
(n+1)^3
\end{eqnarray*}

\medskip
\noindent
(b)
\begin{eqnarray*}
(\sum_{i=0}^{n+1}i)^2 & = &
(\sum_{i=0}^ni)^2 + 2\sum_{i=0}^ni(n+1) + (n+1)^2 \\ & = &
(\sum_{i=0}^ni)^2 + (n+1) \sum_{i=0}^n(2i+1) \\ & = &
(\sum_{i=0}^ni)^2 + (n+1) (\sum_{i=0}^n (n(n+1)+2i+1) - n(n+1)^2) \\ & = &
(\sum_{i=0}^ni)^2 + (n+1) ((n+1)^3 - n(n+1)^2)\mbox{, by (a)} \\ & = &
1^3 + 2^3 + \cdots + n^3 + (n+1)^3\mbox{, by induction}
\end{eqnarray*}

\bigskip
\noindent
\textbf{9.} If $n = 0$, $(1 - a)^0 \ge 1 - 0 a$.\\
Suppose, $(1 - a)^n \ge 1 - na$ for $n \ge 0$.\\
Thus we have by induction:
\begin{eqnarray*}
(1-a)^{n+1} & \ge & (1-a)(1-na)\mbox{, because } 1-a > 0\\
& = & 1 - (n+1)a + na\\
& \ge & 1 - (n+1)a\mbox{, because } a > 0
\end{eqnarray*}

\bigskip
\noindent
\textbf{10.} If $n = 10$, $2^{10} = 1024 > 10^3 = 1000$.\\
Suppose $2^n > n^3$, for $n \ge 10$.\\
We have $n \ge 1$,\\
thus $3n+1 \le 4n \le 4n^2$\\
and $3n^2+3n+1 \le 7n^2 \le n^3$, because $n \ge 10$\\
thus $(n+1)^3 \le 2n^3 < 2^{n+1}$ by induction

\bigskip
\noindent
\textbf{11.} For $n$ positive, note: 
$S_n = \sum_{k=0}^n\frac{(2k+1)^3}{(2k+1)^4 + 4}$\\
For $x$ real, we have:
\[ \frac{x^3}{x^4 + 4} =
\frac{1}{4} \left(\frac{1}{x - (1+\imath)} + \frac{1}{x + (1+\imath)} +
\frac{1}{x-(1-\imath)} + \frac{1}{x + (1-\imath)}\right)\mbox{ (*)}
\]

Thus,
\begin{eqnarray*}
S_n & = & \frac{1}{4}\left(
\sum_{k=0}^n\frac{(-1)^k}{2k-\imath} + \sum_{k=0}^n\frac{(-1)^k}{2n+\imath} +
\sum_{k=0}^n\frac{(-1)^k}{2(k+1)+\imath} + 
\sum_{k=0}^n\frac{(-1)^k}{2(n+1)-\imath} \right)
\\
S_n & = & \frac{(-1)^n}{4} \left(\frac{1}{2(n+1)+\imath} +
\frac{1}{2(n+1)-\imath}\right)\mbox{ (**)}
\\
S_n & = & \frac{(-1)^n(n+1)}{4(n+1)^2 + 1}
\end{eqnarray*}

\noindent
If $n=0$, $S_0 = \frac{1^3}{1^4+4} = \frac{1}{4 1^2 + 1}$.\\
Suppose $S_n = \frac{(-1)^n (n+1)} {4(n+1)^2 + 1}$, for $n \ge 0$.\\
By induction:
\begin{eqnarray*}
S_{n+1} & = & \frac{(-1)^n(n+1)}{4(n+1)^2+1} +
\frac{(-1)^{n+1}(2(n+1)+1)^3}{(2(n+1)+1)^4 + 4}
\\ & = &
\frac{(-1)^n}{4} 
\left(\frac{1}{2(n+1)+\imath} + \frac{1}{2(n+1)-\imath}\right) +
\\&&
\frac{(-1)^{n+1}}{4}
\left(\frac{1}{2(n+1)-\imath} + \frac{1}{2(n+2)+\imath} +
\frac{1}{2(n+1)+\imath} + \frac{1}{2(n+2)-\imath}\right)
\mbox{, by (*) and (**)}
\\ & = &
\frac{(-1)^{n+1} (n+2)}{4(n+2)^2+1} 
\end{eqnarray*}

\bigskip
\noindent
\textbf{12.}
\begin{description}
\item[E1.]
[Initialize.] Set $a' \leftarrow b \leftarrow 1$,
$a \leftarrow b' \leftarrow 0$, $c \leftarrow |m|$, $d \leftarrow |n|$.
\item[E2.]
[Divide.] Let q and r be the real numbers verifying 
$q = \lfloor \frac{c}{d} \rfloor$ and $r = c - qd$ (thus, $0 \le r < |d|$)
\item[E3.]
[Remainder zero?] If $r=0$, if $m < 0$ set $a \leftarrow -a$ and if $n < 0$
set $b \leftarrow -b$ and the algorithm terminates; we have in this case
$am + bn = d$ as desired.
\item[E4.]
[Recycle.] Set $c \leftarrow d$, $d \leftarrow r$, $t \leftarrow a'$,
$a' \leftarrow a$, $a \leftarrow t - qa$, $t \leftarrow b'$, $b' \leftarrow b$,
$b \leftarrow t - qb$, and go back to \textbf{E2}. $|$
\end{description}

Let's proove that $a|m|+b|n|=d$, $a'|m|+b'|n|=c$, always hold whenever 
\textbf{E2} is executed.\\
After \textbf{E1} $a' = 1$, $b = 1$, $a = 0$, $b' = 0$, $c = |m|$, $d = |n|$.\\
After \textbf{E2}\\
After \textbf{E3}\\
After \textbf{E4}
\begin{eqnarray*}
a_2 |m| + b_2 |n| & = & (a'_1 - qa_1)|m| + (b'_1 - qb_1)|n|\\
& = & a'_1|m|+b'_1|m| - q(a_1|m|+b_1|n|)\\
& = & c_1 - qd_1\\
& = & r\\
& = & d_2
\end{eqnarray*}
and $a'_2|m| + b'_2|n| = a_1|m|+b_1|n| = d_1 = c_1$.

\medskip
Note that if $u$ and $v$ are integers then:
\[ \lfloor u + v\sqrt{2} \rfloor = u + \lfloor v\sqrt{2} \rfloor \]

Here's a simple algorithm that computes $\lfloor \sqrt{v} \rfloor$ if $v$
is a positive integer.
\begin{description}
\item[C1.]
[Initialize.] $m \leftarrow 0$
\item[C2.]
[Terminate.] If $(m+1)^2 > 2 v^2$, the algorithm terminates; $m$ is the answer.
\item[C3.]
[Recycle.] $m \leftarrow m+1$ and go back to $C2$.
\end{description}

And if $v$ is a negative integer,
\[ \lfloor v \sqrt{2} \rfloor = - \left(\lfloor -v\sqrt{2}\rfloor + 1\right) \]

Suppose the algorithm \textbf{F} terminates for $m = 1$ and $n = \sqrt{2}$.\\
Thus,
\begin{eqnarray*}
0 & = & c - qd \\
& = & (a'|m| + b'|n|) - q(a|m|+b|n|)\\
& = & (a'-qa) + (b'-qb)\sqrt{2}
\end{eqnarray*}

Thus, $\sqrt{2}$ is rationnal wich is absurd. We then conclude that the
computation won't terminate for $m = 1$ and $n = \sqrt{2}$.

\bigskip
\noindent
\textbf{13.}
\begin{description}
\item[E1.]
[Initialize.] Set $a' \leftarrow b \leftarrow 1$, $a \leftarrow b' \leftarrow 0
$, $c \leftarrow m$, $d \leftarrow n$, $T \leftarrow T+1$.
\item[E2.]
[Divide.] Set $T \leftarrow T+1$. Let $q$, $r$ be the quotient and remainder,
respectively, of $c$ divided by $d$. (We have $c=qd+r$, $0 \le r < d$.)
\item[E3.]
[Remainder zero?] Set $T \leftarrow T+1$. If $r=0$, the algorithm terminates;
we have in this case $am + bn = d$ as desired.
\item[E4.]
[Recycle.] Set $T \leftarrow T+1$, $c \leftarrow d$, $d \leftarrow r$,
$t \leftarrow a'$, $a' \leftarrow a$, $a \leftarrow t - qa$, $t \leftarrow b'$,
$b' \leftarrow b$, $b \leftarrow t - qb$, and go back to E2. $|$
\end{description}

\begin{description}
\item[A1.]
$m > 0$, $n > 0$, $T = 0$.
\item[A2.]
$c = m > 0$, $d = n > 0$, $T = 1$, $a = b' = 0$, $a' = b = 1$.
\item[A3.]
$am + bn = d$, $a'm + b'n = c = qd + r$, $0 \le r < d$\\
$\gcd(m, n) = \gcd(c, d)$\\
$T \le 3(n-r)-1$
\item[A4.]
$am + bn = d = \gcd(m, n)$\\
$T \le 3(n-r)$
\item[A5.]
$am + bn = d$, $a'm + b'n = c = qd + r$, $0 < r < d$, $\gcd(m,n) = \gcd(c, d)$,
$T \le 3(n-r)$
\item[A6.]
$am+bn=d$, $a'm + b'n = c$, $d > 0$, $\gcd(c,d) = \gcd(m,n)$, $T \le 3(n-d)+1$
\end{description}

\begin{description}
\item[A1.]
We have $T = 0$.
\item[A2.]
We increment $T$ by $1$, so $T = 1$.
\item[A3.]
Suppose it's the first time we execute \textbf{E2}.\\
We have, $0 \le r < d = n$,\\
Thus $3(n-r)-1 \ge 3 - 1 = 2 = T$.\\
Suppose we stepped from \textbf{E6} to \textbf{E2}, we have already
$ T - 1 \le 3(n - d)+1$.\\
So
\begin{eqnarray*}
T & \le & 3(n-d) + 2\\
T & \le & 3(n-r) + 3(r-d) + 2\\
& \le & 3(n-r) - 1\mbox{, because } 0 \le r < d
\end{eqnarray*}

\end{description}

\textbf{A4}, \textbf{A5}, \textbf{A6} are straitforward.

\medskip
If $r = 0$, we have $T \le 3(n-r) = 3n$.  The algorithm terminates after at
most $3n$ steps.

\bigskip
\noindent
\textbf{15.}
a) $\rbrack -\infty, 0 \rbrack \cap \mathbf{N}$ is non-empty subset of 
\textbf{N} wich don't have a minimal element.

b) Note $\dashv$ the relation defined by:
\[ n \dashv m \Leftrightarrow \left\{\begin{array}{lll}
	mn < 0 & \mbox{and} & m < 0 < n\mbox{,}\\
	& \mbox{or} &\\
        mn \ge 0 & \mbox{and} & |n| \le |m|. \end{array}\right. \]
Thus we have,
 \[ 0 \dashv -1 \dashv -2 \dashv \ldots \dashv -100 \dashv \ldots 1 \dashv 
2 \dashv \ldots \]

c) No.  Because the set $\rbrack 0, 1 \rbrack$ doesn't verify he propriety iii).

d) Let $x$, $y$, $z$ be elements of $T_n$ such that $x \prec y$ and
 $y \prec z$ with $x = (x_1, \ldots, x_n)$, $y = (y_1, \ldots, y_n)$, 
$z = (z_1, \ldots, z_n)$. By definition, there're $k$ and $l$ between $1$ 
and $n$ such that:
\[x_i = y_i \mbox{ for } 1 \le i < k \mbox{ and } x_k \prec y_k \] and
\[y_i = z_i \mbox{ for } 1 \le i < l \mbox{ and } y_l \prec z_l \]

Let $m$ be the minimum between $k$ and $l$. We have:
\[ x_i = y_i = z_i \mbox{ for } 1 \le i < m \mbox { and } x_m \prec y_m
\mbox{  or } y_m \prec z_m\]
Thus, $x_m \prec z_m$ given that $x_m \preceq y_m \preceq z_m$. 
So $x \prec y$.

\medskip
Let $x$, $y$ be in $T_n$ and $x \not= y$. Let $k$ be the smallest integer 
between $1$ and $n$ such that $x_k \not= y_k$.\\
Thus, $x_i=y_i$ for $1 \le i \le k$ and $x_k \prec y_k$ or $y_k \prec x_k$.\\
Thus, $x \prec y$ or $y \prec x$.

\medskip
Let's reason by induction:\\
If $n = 1$, $S$ is well-ordered by $\prec$.\\
Suppose $T_{n-1}$ is well-ordered by $\prec$.

Because $(S, \prec)$ is well-ordered, there exists an element
$m$ of $S$ such that $m \prec s$ for each $s$ in $S$.
Let $M$ be the subset of $S$ composed of the elements wich first member is
$m$.  If $x$ isn't in $M$, thus $a \prec x$ for each $a$ in $M$.

$M$ could be identified as a subset of $T_{n-1}$, because each element has its
first member equal to $m$.  By induction $M$ is well-ordered by $\prec$. Thus
$(T_n, \prec)$ is well-ordered.

\medskip
e) $T = \bigcup_{n \ge 1}T_n$.\\
$(x_1,x_2,\ldots,x_n) \prec (y_1,y_2,\ldots,y_m)$ if $x_i = y_j$ for
$1 \le j < k$ and $x_k < y_k$, for some $k < m$, $n$; or if $x_j = y_j$ for
$1 \le j \le n$ and $n < m$.

Let $x$, $y$, $z$ be in $T$ such that $x \prec y$ and $y \prec z$ with
$x = (x_1,\ldots,x_n)$, $y = (y_1,\ldots,y_m)$, $z = (z_1,\dots,z_l)$.
\begin{itemize}
\item
Suppose $x_j = y_j$ for $1 \le j \le n$ and $n < m$.\\
If we have $y_j = z_j$ for $1 \le j \le m$ and $m < l$,
\[x_j = z_j \mbox{ for } 1 \le j \le n \mbox{ and } n < m < l \mbox{, thus }
x \prec z.\]
If we have $y_j = z_j$ for $ 1 \le j < k$ and $y_k \prec z_k$, for some 
$k < m$, $l$.\\
If $k > n$, we have:
\[x_j = y_j = z_j \mbox{ for }1 \le j \le n \mbox{ and } n < l \mbox{, thus } 
x \prec z.\]
If $k \le n$, we have:
\[x_j = y_j = z_j \mbox{ for } 1 \le j < k \mbox{ and } x_k = y_k \prec z_k
\mbox{, thus } x \prec z.\]
\item
Suppose $x_j = y_j$ for $1 \le j < k$ and $x_k \prec y_k$, for some $k < m$, 
$n$;\\
If we have $y_j = z_j$ for $1 \le j \le m$ and $m < l$, thus
\[x_j = y_j = z_j \mbox{ for } 1 \le j < k \mbox{ and } x_k \prec y_k = z_k
\mbox{, thus } x \prec z.\]
If we have $y_j = z_j$ for $1 \le j < p$ and $y_p \prec z_p$, for some 
$p < m$, $l$. If $k < p$:
\[x_j = y_j = z_j \mbox{ for } 1 \le j < k \mbox { and } x_k \prec y_k = z_k,
k < n, \mbox{ thus } x \prec z\]
If $k \ge p$:
\[x_j = y_j = z_j \mbox{ for } 1 \le j < p \mbox{ and } 
x_p \preceq y_p \prec z_p, \mbox { for some } p < n, l.\]
Thus $x \prec z$.
\end{itemize}

\medskip
Let $x$, $y$ be in $T$ and $x \not= y$.\\
Note $x = (x1, \ldots, x_n)$ and $y = (y_1, \ldots, y_m)$ with $n$ and $m$ 
integers. If $n = m$, we have already treated the case.  We suppose $n < m$
without a loss of generality.\\
If $x_j = y_j$ for $1 \le j \le n$, we have $x \prec y$ because $n < m$.\\
Otherwise, let $k$ be the smallest integer such that $x_k \not= y_k$.  We have,
\[x_j = y_j \mbox{ for } 1 \le j < k \mbox{ and } x_k \prec y_k \mbox{ or }
y_k \prec x_k.\]
Thus $x \prec y$ or $y \prec x$.

\medskip
Let $A$ be the subset of $T$ composed of the elements $X_j$ of $T_j$ for $j$
integers defined by:
\[X_j(i) = 0 \mbox{ if } i < j \mbox{ and } X_j(j) = 1.\]
For $j$ integer, $X_{j+1} \prec X_j$ thus $(T, \prec)$ is not well-ordered.

\medskip
f) Suppose $\prec$ is a well-ordering  of $S$.  i) and ii) are satisfied.
And suppose there's an infinite sequence $x_1, x_2, x_3, \ldots$ with
$x_{n+1} < x_n$ for all $n \ge 1$.

$(S, \prec)$ is well-ordered thus, there exists $n$ positive integer such that\\
$x_n \preceq x_j$ for all $j \ge 0$.  This is absurd because $x_{n+1} \prec x_n$
so ii) is not verified.  So, such a sequence couldn't exist.

Suppose $(S, \prec)$ is not well-ordered.\\
Let $x_1$ be an element of $S$.  By definition there exists an elment $x_2$ of
$S$ such that $x_2 < x_1$.  There's an element of $S$ such that $x_3 \prec x_2$
and so on.

\medskip
g) Let $S$ be well-ordered by $\prec$, and let $P(x)$ be a statement about the
element $x$ of $S$.\\
Suppose that there exists $x_0 \in S$ such that $P(x_0)$ is false. Note $F$ the 
non-empty set defined by:
\[ F = \{ f \in S, P(f) \mbox{ is false } \} .\]
S is well-ordered so there exists an element $m$ of $F$ such that:
\[ m \prec f,  \forall f \in F\]
We then conclude that if $x \prec m$, $P(x)$ is true.  We then have also
$P(m)$ is true.  Wich is absurd.  So $P(x)$ is true for all $x$ in $S$. 
\end{document}
