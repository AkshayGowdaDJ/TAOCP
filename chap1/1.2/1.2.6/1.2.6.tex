\documentclass[a4paper,12pt]{article}
\newcommand{\newpar}[1]{\bigskip \noindent \textbf{#1.}}
\newcommand{\subpar}[1]{\medskip \noindent (#1)}
\newcommand{\la}{\leftarrow}
\newcommand{\ra}{\rightarrow}
\begin{document}

\newpar{1} $n$.

\newpar{2} $1$.

\newpar{3} $C_{52}^{13} = 635013559600$.

\newpar{4} $2^4\,41\,43\,23\,17\,5^2\,7^2\,47$.

\newpar{5} We have
\begin{eqnarray*}
  11^4 &=& (10 + 1)^4 \\
  &=& C_4^0 10^4 + C_4^1 10^3 + C_4^2 10^2 + C_4^1 10 + C_4^4 \\
  &=& 10^4 + 4\,10^3 + 6\,10^2 + 4\,10 + 1 \\
  &=& 14641
\end{eqnarray*}

\newpar{6} Using (4) and (9) we have,
\begin{quote}
  \begin{tabular}{|c|c|c|c|c|c|c|c|c|c|c|}
    \hline $r$ & $C_r^0$ & $C_r^1$ & $C_r^2$ & $C_r^3$ & $C_r^4$ &
    $C_r^5$ & $C_r^6$ & $C_r^7$ & $C_r^8$ & $C_r^9$ \\
    \hline $-3$ & $1$ & $-3$ & $6$ & $-10$ & $15$ & $-21$ & $28$ &
    $-36$ & $45$ & $-55$ \\
    \hline $-2$ & $1$ & $-2$ & $3$ & $-4$ & $5$ & $-6$ & $7$ & $-8$ &
    $9$ & $-10$ \\
    \hline $-1$ & $1$ & $-1$ & $1$ & $-1$ & $1$ & $-1$ & $1$ & $-1$ &
    $1$ & $-1$ \\
    \hline
  \end{tabular}
\end{quote}

\newpar{7}  Since we have $C_n^k = C_n^{n-k}$, we can impose that $k
\le n-k$ without a loss of generality.  But we have,

\[ C_n^k = \frac{n-k+1}{k}\ C_n^{k-1} \]

We then deduce that $C_n^k \ge C_n^{k-1}$ if $k \le n-k$.  Thus, the
value of $k$ that achieves the maximum is $\lfloor n/2\rfloor$.

\newpar{8} Leaving out the zeroes, the triangle is symmetric according
to the line $k = \lfloor r/2\rfloor$.

\newpar{9} $1$ if $n$ is positive or equal to zero and $0$ otherwise.

\newpar{10} Suppose that $p$ is prime.

\subpar{a} We have

\begin{eqnarray*}
  n! &=& \prod_{k=1}^n k \\
  &=& \left( \prod_{0 \le l \le \lfloor n/p\rfloor - 1}\ \prod_{lp <
    k\le (l+1)p} k \right) \left( \prod_{p \lfloor n/p\rfloor + 1\le k \le
    n} k \right) \\
  &=& \left( \prod_{0\le l \le \lfloor n/p\rfloor - 1} \prod_{0 < k
    \le p} (k+lp) \right) \left( \prod_{1 \le k \le n - p \lfloor n/p \rfloor}
  (k+p\lfloor n/p\rfloor) \right)
\end{eqnarray*}

Hence we deduce,

\begin{eqnarray*}
  p!(n-p)! &=& p! \left( \prod_{0\le l \le \lfloor (n-p)/p\rfloor-1}
  \prod_{0<k\le p} (k+lp)\right) \\ &&
  \left( \prod_{1\le k\le n-p(\lfloor (n-p)/p\rfloor+1)} (k + p \lfloor
  (n-p)/p\rfloor)\right)
\end{eqnarray*}

Given that $\lfloor (n-p)/p \rfloor = \lfloor n/p\rfloor - 1$, we then
deduce


\[  C_n^p = \lfloor n/p\rfloor \frac{\prod_{0< k < p}(k + p(\lfloor
    n/p\rfloor - 1))}{(p-1)!}\,
  \prod_{1\le k\le n-p\lfloor n/p\rfloor }
  \frac{k+p\lfloor n/p\rfloor}{k+p\lfloor (n-p)/p\rfloor} \]

Given that $p$ is prime, for $k$ varying between $1$ and $p-1$, $k$ is
invertible in $Z/pZ$. We then deduce that all the terms that appear as
denominators in the previous equality are invertible terms.  Hence

\[ C_n^p \equiv \lfloor n/p\rfloor \pmod p\]

\subpar{b} Let $k$ be an integer between $1$ and $p-1$.  We have
\[ p! = C_p^k k! (p-k)! \]

Since $p$ is prime, none of the term that appears in the two
factorial divide $p$.  We then deduce that $k! (p-k)!$ divides
$(p-1)!$. By rewriting the equality as
\[ C_p^k = p\,\frac{(p-1)!}{k!(p-k)!}\]

we deduce that $C_p^k \equiv 0 \pmod p$.

\subpar{c} Let $k$ be an integer between $0$ and $p-1$. We have
\begin{eqnarray*}
  C_{p-1}^k &=& \frac{\prod_{p-k \le i\le p-1} i}{k!} \\
  &=& \frac{\prod_{1\le i\le k} (p-i)}{k!}
\end{eqnarray*}

Given that $p$ is prime, the denominator is invertible in $Z/pZ$,
hence
\[ C_{p-1}^k \equiv (-1)^k \pmod p.\]

\subpar{d} Let $k$ be an integer between $2$ and $p-1$. We then have $p+2-k
\le p$, and since

\[  k!C_{p+1}^k = \prod_{p+2-k\le i\le p+1} i \]

we have $k! C_{p+1}^k \equiv 0 \pmod p$ and since $k!$ is invertible, we
then deduce
\[ C_{p+1}^k \equiv 0 \pmod p.\]

\subpar{e} We have

\begin{eqnarray*}
  n! &=& \prod_{1\le i\le n} i \\
  &=& \left( \prod_{0 \le j \le \lfloor n/p\rfloor - 1} \prod_{jp < i
    \le (j+1)p} i \right) \left( \prod_{p\lfloor n/p\rfloor + 1 \le i
    \le n} i\right) \\
  &=& \left( \prod_{0 \le j\le \lfloor n/p\rfloor - 1} (j+1)p
  \prod_{jp < i <(j+1)p} i \right)\left( \prod_{p\lfloor n/p\rfloor +
    1 \le i \le n} i\right) \\
  &=& p^{\lfloor n/p\rfloor} \lfloor n/p\rfloor!
  \left( \prod_{ 0 \le j \le \lfloor n/p\rfloor - 1} \prod_{jp < i <
    (j+1)p} i\right)\left( \prod_{p\lfloor n/p\rfloor + 1 \le i \le n}
  i\right) \\
  &=& p^{\lfloor n/p\rfloor} \lfloor n/p\rfloor! \left( \prod_{0\le j
    \le \lfloor n/p\rfloor - 1} \prod_{0 < i < p}(i+jp)\right)
  \prod_{1\le i\le n - p \lfloor n/p\rfloor} (i+p\lfloor n/p\rfloor) \\
  n!  &=& p^{\lfloor n/p\rfloor} \lfloor n/p\rfloor! \left( \prod_{0\le j
    \le \lfloor n/p\rfloor - 1} \prod_{0 < i < p}(i+jp)\right)
  \prod_{1\le i\le n \bmod p} (i+p\lfloor n/p\rfloor)\, \mbox{(*)}
\end{eqnarray*}

On the other hand, we have

\begin{eqnarray*}
  \left\lfloor \frac{k}{p}\right\rfloor + \left\lfloor
  \frac{n-k}{p}\right\rfloor &=& \left\lfloor \frac{n-k}{p} + \left\lfloor
  \frac{k}{p}\right\rfloor\right\rfloor \\
  &=& \left\lfloor \frac{n - k \bmod p}{p}\right\rfloor \\
  &=& \left\lfloor \left\lfloor \frac{n}{p}\right\rfloor + \frac{n
    \bmod p - k\bmod p}{p}\right\rfloor \\
  &=& \left\lfloor \frac{n}{p}\right\rfloor + \left\lfloor  \frac{n
    \bmod p - k\bmod p}{p}\right\rfloor \\
  &=& \left\lfloor \frac{n}{p}\right\rfloor + \left\lfloor
  \frac{(n-k)\bmod p}{p}\right\rfloor \\
  &=& \left\lfloor \frac{n}{p}\right\rfloor, \mbox{since $k \le n$}
\end{eqnarray*}

From (*), we then deduce

\begin{eqnarray*}
  C_n^k &=& C_{\lfloor n/p\rfloor}^{\lfloor k/p\rfloor}
  \frac{\prod_{j=\lfloor (n-k)/p\rfloor}^{\lfloor
      n/p\rfloor-1}\prod_{0<i<p}(i+jp)} {\prod_{j=0}^{\lfloor
      k/p\rfloor-1}\prod_{0<i<p}(i+jp)} C_{n\bmod p}^{k\bmod p} \\
  &\equiv&  C_{\lfloor n/p\rfloor}^{\lfloor k/p\rfloor} C_{n\bmod
    p}^{k\bmod p} \pmod p
\end{eqnarray*}

\subpar{f} The result is easily obtained by induction using (e).

\newpar{11} Let $(a_i)_{i \in N}$, $(b_i)_{i \in N}$ and $(c_i)_{i \in
  N}$ three sequels representing respectively  $a$, $b$ and $a+b$ in
the $p$-ary number system.  Let's remark that all three of them become
stationary to $0$ for a sufficiently large value of $i$. Using
1.2.5--12, we have

\begin{eqnarray*}
  \mu\left( C_{a+b}^a \right) &=& \mu((a+b)!) - \mu(a!) - \mu(b!) \\
  &=& \frac{\sum_{i \in N} (a_i + b_i - c_i)}{p-1}
\end{eqnarray*}

We can partition $N$ into intervals that distinguish the additions
that produce carries and those which don't.  Let $(i_k)_{i \in N}$ and
$(j_k)_{k \in N}$ be two increasing sequels such that
\begin{eqnarray*}
  a_{i_k} + b_{i_k} &=& c_{i_k} + p \\
  a_{i_k+1} + b_{i_k + 1} + 1 &=& c_{i_k + 1} + p \\
  \ldots \\
  a_{j_k - 1} + b_{j_k - 1} + 1 &=& c_{j_k - 1} + p \\
  a_{j_k} + b_{j_k} + 1 &=& c_{j_k}
\end{eqnarray*}

The last equality implies that $j_k < i_{k+1}$ and we have
\[ \sum_{i_k \le i \le j_k} (a_i + b_i - c_i) = (p-1) (j_k - i_k +
1),\]
and obviously
\[ \sum_{j_k < i < i_{k+1}} (a_i + b_i - c_i) = 0\]
because none of the terms produce a carry.

So finally,
\begin{eqnarray*}
  \mu\left( C_{a+b}^a\right) &=& \frac{\sum_{k \in N} \sum_{i_k \le i
      < i_{k+1}} (a_i + b_i - c_i)}{p-1} \\
  &=& \sum_{k \in N} (j_k - i_k + 1)
\end{eqnarray*}

We then deduce that the multiplicity of $p$ is the number of carries
that occur when $a$ is added to $b$ in the $p$-ary number system.

\end{document}
