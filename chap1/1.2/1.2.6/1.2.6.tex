\documentclass[a4paper,12pt]{article}
\newcommand{\newpar}[1]{\bigskip \noindent \textbf{#1.}}
\newcommand{\subpar}[1]{\medskip \noindent (#1)}
\newcommand{\la}{\leftarrow}
\newcommand{\ra}{\rightarrow}
\begin{document}

\newpar{1} $n$.

\newpar{2} $1$.

\newpar{3} $C_{52}^{13} = 635013559600$.

\newpar{4} $2^4\,41\,43\,23\,17\,5^2\,7^2\,47$.

\newpar{5} We have
\begin{eqnarray*}
  11^4 &=& (10 + 1)^4 \\
  &=& C_4^0 10^4 + C_4^1 10^3 + C_4^2 10^2 + C_4^1 10 + C_4^4 \\
  &=& 10^4 + 4\,10^3 + 6\,10^2 + 4\,10 + 1 \\
  &=& 14641
\end{eqnarray*}

\newpar{6} Using (4) and (9) we have,
\begin{quote}
  \begin{tabular}{|c|c|c|c|c|c|c|c|c|c|c|}
    \hline $r$ & $C_r^0$ & $C_r^1$ & $C_r^2$ & $C_r^3$ & $C_r^4$ &
    $C_r^5$ & $C_r^6$ & $C_r^7$ & $C_r^8$ & $C_r^9$ \\
    \hline $-3$ & $1$ & $-3$ & $6$ & $-10$ & $15$ & $-21$ & $28$ &
    $-36$ & $45$ & $-55$ \\
    \hline $-2$ & $1$ & $-2$ & $3$ & $-4$ & $5$ & $-6$ & $7$ & $-8$ &
    $9$ & $-10$ \\
    \hline $-1$ & $1$ & $-1$ & $1$ & $-1$ & $1$ & $-1$ & $1$ & $-1$ &
    $1$ & $-1$ \\
    \hline
  \end{tabular}
\end{quote}

\newpar{7}  Since we have $C_n^k = C_n^{n-k}$, we can impose that $k
\le n-k$ without a loss of generality.  But we have,

\[ C_n^k = \frac{n-k+1}{k}\ C_n^{k-1} \]

We then deduce that $C_n^k \ge C_n^{k-1}$ if $k \le n-k$.  Thus, the
value of $k$ that achieves the maximum is $\lfloor n/2\rfloor$.

\newpar{8} Leaving out the zeroes, the triangle is symmetric according
to the line $k = \lfloor r/2\rfloor$.

\newpar{9} $1$ if $n$ is positive or equal to zero and $0$ otherwise.

\newpar{10} Suppose that $p$ is prime.

\subpar{a} We have

\begin{eqnarray*}
  n! &=& \prod_{k=1}^n k \\
  &=& \left( \prod_{0 \le l \le \lfloor n/p\rfloor - 1}\ \prod_{lp <
    k\le (l+1)p} k \right) \left( \prod_{p \lfloor n/p\rfloor + 1\le k \le
    n} k \right) \\
  &=& \left( \prod_{0\le l \le \lfloor n/p\rfloor - 1} \prod_{0 < k
    \le p} (k+lp) \right) \left( \prod_{1 \le k \le n - p \lfloor n/p \rfloor}
  (k+p\lfloor n/p\rfloor) \right)
\end{eqnarray*}

Hence we deduce,

\begin{eqnarray*}
  p!(n-p)! &=& p! \left( \prod_{0\le l \le \lfloor (n-p)/p\rfloor-1}
  \prod_{0<k\le p} (k+lp)\right) \\ &&
  \left( \prod_{1\le k\le n-p(\lfloor (n-p)/p\rfloor+1)} (k + p \lfloor
  (n-p)/p\rfloor)\right)
\end{eqnarray*}

Given that $\lfloor (n-p)/p \rfloor = \lfloor n/p\rfloor - 1$, we then
deduce


\[  C_n^p = \lfloor n/p\rfloor \frac{\prod_{0< k < p}(k + p(\lfloor
    n/p\rfloor - 1))}{(p-1)!}\,
  \prod_{1\le k\le n-p\lfloor n/p\rfloor }
  \frac{k+p\lfloor n/p\rfloor}{k+p\lfloor (n-p)/p\rfloor} \]

Given that $p$ is prime, for $k$ varying between $1$ and $p-1$, $k$ is
invertible in $Z/pZ$. We then deduce that all the terms that appear as
denominators in the previous equality are invertible terms.  Hence

\[ C_n^p \equiv \lfloor n/p\rfloor \pmod p\]

\subpar{b} Let $k$ be an integer between $1$ and $p-1$.  We have
\[ p! = C_p^k k! (p-k)! \]

Since $p$ is prime, none of the term that appears in the two
factorial divide $p$.  We then deduce that $k! (p-k)!$ divides
$(p-1)!$. By rewriting the equality as
\[ C_p^k = p\,\frac{(p-1)!}{k!(p-k)!}\]

we deduce that $C_p^k \equiv 0 \pmod p$.

\subpar{c} Let $k$ be an integer between $0$ and $p-1$. We have
\begin{eqnarray*}
  C_{p-1}^k &=& \frac{\prod_{p-k \le i\le p-1} i}{k!} \\
  &=& \frac{\prod_{1\le i\le k} (p-i)}{k!}
\end{eqnarray*}

Given that $p$ is prime, the denominator is invertible in $Z/pZ$,
hence
\[ C_{p-1}^k \equiv (-1)^k \pmod p.\]

\subpar{d} Let $k$ be an integer between $2$ and $p-1$. We then have $p+2-k
\le p$, and since

\[  k!C_{p+1}^k = \prod_{p+2-k\le i\le p+1} i \]

we have $k! C_{p+1}^k \equiv 0 \pmod p$ and since $k!$ is invertible, we
then deduce
\[ C_{p+1}^k \equiv 0 \pmod p.\]

\subpar{e} We have

\begin{eqnarray*}
  n! &=& \prod_{1\le i\le n} i \\
  &=& \left( \prod_{0 \le j \le \lfloor n/p\rfloor - 1} \prod_{jp < i
    \le (j+1)p} i \right) \left( \prod_{p\lfloor n/p\rfloor + 1 \le i
    \le n} i\right) \\
  &=& \left( \prod_{0 \le j\le \lfloor n/p\rfloor - 1} (j+1)p
  \prod_{jp < i <(j+1)p} i \right)\left( \prod_{p\lfloor n/p\rfloor +
    1 \le i \le n} i\right) \\
  &=& p^{\lfloor n/p\rfloor} \lfloor n/p\rfloor!
  \left( \prod_{ 0 \le j \le \lfloor n/p\rfloor - 1} \prod_{jp < i <
    (j+1)p} i\right)\left( \prod_{p\lfloor n/p\rfloor + 1 \le i \le n}
  i\right) \\
  &=& p^{\lfloor n/p\rfloor} \lfloor n/p\rfloor! \left( \prod_{0\le j
    \le \lfloor n/p\rfloor - 1} \prod_{0 < i < p}(i+jp)\right)
  \prod_{1\le i\le n - p \lfloor n/p\rfloor} (i+p\lfloor n/p\rfloor) \\
  n!  &=& p^{\lfloor n/p\rfloor} \lfloor n/p\rfloor! \left( \prod_{0\le j
    \le \lfloor n/p\rfloor - 1} \prod_{0 < i < p}(i+jp)\right)
  \prod_{1\le i\le n \bmod p} (i+p\lfloor n/p\rfloor)\, \mbox{(*)}
\end{eqnarray*}

On the other hand, we have

\begin{eqnarray*}
  \left\lfloor \frac{k}{p}\right\rfloor + \left\lfloor
  \frac{n-k}{p}\right\rfloor &=& \left\lfloor \frac{n-k}{p} + \left\lfloor
  \frac{k}{p}\right\rfloor\right\rfloor \\
  &=& \left\lfloor \frac{n - k \bmod p}{p}\right\rfloor \\
  &=& \left\lfloor \left\lfloor \frac{n}{p}\right\rfloor + \frac{n
    \bmod p - k\bmod p}{p}\right\rfloor \\
  &=& \left\lfloor \frac{n}{p}\right\rfloor + \left\lfloor  \frac{n
    \bmod p - k\bmod p}{p}\right\rfloor \\
  &=& \left\lfloor \frac{n}{p}\right\rfloor + \left\lfloor
  \frac{(n-k)\bmod p}{p}\right\rfloor \\
  &=& \left\lfloor \frac{n}{p}\right\rfloor, \mbox{since $k \le n$}
\end{eqnarray*}

From (*), we then deduce

\begin{eqnarray*}
  C_n^k &=& C_{\lfloor n/p\rfloor}^{\lfloor k/p\rfloor}
  \frac{\prod_{j=\lfloor (n-k)/p\rfloor}^{\lfloor
      n/p\rfloor-1}\prod_{0<i<p}(i+jp)} {\prod_{j=0}^{\lfloor
      k/p\rfloor-1}\prod_{0<i<p}(i+jp)} C_{n\bmod p}^{k\bmod p} \\
  &\equiv&  C_{\lfloor n/p\rfloor}^{\lfloor k/p\rfloor} C_{n\bmod
    p}^{k\bmod p} \pmod p
\end{eqnarray*}

\subpar{f} The result is easily obtained by induction using (e).

\newpar{11} Let $(a_i)_{i \in N}$, $(b_i)_{i \in N}$ and $(c_i)_{i \in
  N}$ three sequels representing respectively  $a$, $b$ and $a+b$ in
the $p$-ary number system.  Let's remark that all three of them become
stationary to $0$ for a sufficiently large value of $i$. Using
1.2.5--12, we have

\begin{eqnarray*}
  \mu\left( C_{a+b}^a \right) &=& \mu((a+b)!) - \mu(a!) - \mu(b!) \\
  &=& \frac{\sum_{i \in N} (a_i + b_i - c_i)}{p-1}
\end{eqnarray*}

We can partition $N$ into intervals that distinguish the additions
that produce carries and those which don't.  Let $(i_k)_{i \in N}$ and
$(j_k)_{k \in N}$ be two increasing sequels such that
\begin{eqnarray*}
  a_{i_k} + b_{i_k} &=& c_{i_k} + p \\
  a_{i_k+1} + b_{i_k + 1} + 1 &=& c_{i_k + 1} + p \\
  \ldots \\
  a_{j_k - 1} + b_{j_k - 1} + 1 &=& c_{j_k - 1} + p \\
  a_{j_k} + b_{j_k} + 1 &=& c_{j_k}
\end{eqnarray*}

The last equality implies that $j_k < i_{k+1}$ and we have
\[ \sum_{i_k \le i \le j_k} (a_i + b_i - c_i) = (p-1) (j_k - i_k +
1),\]
and obviously
\[ \sum_{j_k < i < i_{k+1}} (a_i + b_i - c_i) = 0\]
because none of the terms produce a carry.

So finally,
\begin{eqnarray*}
  \mu\left( C_{a+b}^a\right) &=& \frac{\sum_{k \in N} \sum_{i_k \le i
      < i_{k+1}} (a_i + b_i - c_i)}{p-1} \\
  &=& \sum_{k \in N} (j_k - i_k + 1)
\end{eqnarray*}

We then deduce that the multiplicity of $p$ is the number of carries
that occur when $a$ is added to $b$ in the $p$-ary number system.

\newpar{12} Using \textbf{f} from exercise \textbf{10}, if the $2$-ary
system representations of $n$ and $k$ when $k \le n$ are
\[
\begin{array}{l}
  n = a_r 2^r + \cdots + a_1 2 + a_0, \\
  k = b_r 2^r + \cdots + b_1 2 + b_0,
\end{array}
\mbox{ then } C_n^k \equiv C_{a_r}^{b_r} \ldots C_{a_1}^{b_1}
C_{a_0}^{b_0} \equiv 1 \pmod 2
\]

Hence $a_i \ge b_i$ for all the choices of $k \le n$, otherwise $C_n^k
\equiv 0 \pmod 2$. This is only possible if $a_i = 1$ for $0 \le i \le
r$.  Hence $n = 2^{r+1}-1$.

\newpar{13} Let's show the formula by induction on $n \ge 0$.  If
$n=0$, we have

\[ \sum_{k=0}^0 C_{r+k}^k = 1 = C_{r+1}^0.\]

Suppose that we have the equality for $n$, then we have

\begin{eqnarray*}
  \sum_{0\le k\le n+1}C_{r+k}^k &=& \sum_{0\le k \le n}C_{r+k}^k +
  C_{r+n+1}^{n+1} \\
  &=& C_{r+n+1}^n + C_{r+n+1}^{n+1}, \mbox{ by induction } \\
  &=& C_{r+n+2}^{n+1}
\end{eqnarray*}

\newpar{14} Let $k$ be an integer between $0$ and $n$.  We have
\[ (k+1)^4 - k^4 = 4 k^3 + 6 k^2 + 4 k + 1\]

Hence
\[ (n+1)^4 = 4 \sum_{k=0}^n k^3 + 6 \sum_{k=0}^n k^2 + 4 \sum_{k=0}^n k
+ (n+1)\]

Thus
\[ 4 \sum_{k=0}^n k^3 = (n+1)^4 - 2 n\left(n +
\frac{1}{2}\right)(n+1) - 2 n(n+1) - (n+1)\]

And finally
\[ \sum_{k=0}^n k^3 = \left( \frac{n(n+1)}{2} \right)^2 (*)\]

Similarly
\[ (k+1)^5 - k^5 = 5 k^4 + 10 k^3 + 10 k^2 + 5 k + 1.\]

Hence
\[ (n+1)^5 = 5 \sum_{k=0}^n k^4 + 10 \sum_{k=0}^n k^3 + 10
\sum_{k=0}^n k^2 + 5 \sum_{k=0}^n k + (n+1).\]

From (*), we then deduce
\[ 5 \sum_{k=0}^n k^4 = (n+1)^5 - \frac{5}{2} n^2(n+1)^2 -
\frac{10}{3}n\left(n+\frac{1}{2}\right) (n+1) -
\frac{5}{2}n(n+1)-(n+1).\]

And finally

\[ \sum_{k=0}^n k^4 = \frac{1}{5}n(n+2)\left(n+\frac{1}{2}\right)
\left(n^2 + n - \frac{1}{3}\right).\]

\newpar{15} Let's show the property by induction on $r$.  Suppose
$r=0$, then we have
\[ (x+y)^0 = 1 = \sum_k C_0^k x^k y^{-k}.\]

Suppose that we have the property for $r$, then we have

\begin{eqnarray*}
  (x+y)^{r+1} &=& (x + y) (x+y)^n \\
  &=& (x+y) \sum_k C_r^k x^k y^{r-k}, \mbox{ by induction} \\
  &=& \sum_k C_r^k x^{k+1} y^{r-k} + \sum_k C_r^k x^k y^{r+1-k} \\
  &=& \sum_k C_r^{k-1} x^k y^{r+1-k} + \sum_k C_r^k x^k y^{r+1-k} \\
  &=& \sum_k C_{r+1}^k x^k y^{r+1-k}
\end{eqnarray*}

\newpar{16} We have

\begin{eqnarray*}
  (-1)^k C_{-k}^{n-1} &=& (-1)^k \frac{-k (-k-1)\ldots
    (-k-n+2)}{(n-1)!} \\
  &=& (-1)^k (-n) (-n-1)\ldots(-k-n+2) \frac{(-1)^{n-k}}{(k-1)!} \\
  &=& (-1)^n C_{-n}^{k-1}
\end{eqnarray*}

\newpar{17} We have
\begin{eqnarray*}
  (1+x)^{r+s} &=& (1+x)^r (1+x)^s \\
  &=& \sum_k C_r^k x^k \sum_k C_s^k x^k \\
  &=& \sum_n \sum_k C_r^k C_s^{n-k} x^n \\
\end{eqnarray*}

The two polynomials are equals thus their coefficients are equals
too.  Thus
\[ C_{r+s}^n = \sum_k C_r^k C_s^{n-k}.\]

\newpar{18} We have
\begin{eqnarray*}
  \sum_k C_r^{m+k} C_s^{n+k} &=& \sum_k C_r^k C_s^{n-m+k} \\
  &=& \sum_k C_r^k C_s^{s+m-n-k},\ \mbox{from (6)} \\
  &=& C_{r+s}^{s+m-n}, \mbox{from (21)}\\
  &=& C_{r+s}^{r-m+n}, \mbox{from (6)}
\end{eqnarray*}

\newpar{19} Let's show the equality by induction on $r \ge 0$.  If
$r=0$, we have
\begin{eqnarray*}
  \sum_k C_0^k C_n^{s+k}(-1)^{-k} &=& \sum_k \delta_{k,0}
  C_n^{s+k}(-1)^{k} \\
  &=& C_n^s
\end{eqnarray*}

Suppose that we have the equality for $r$, then we have
\begin{eqnarray*}
  \sum_k C_{r+1}^k C_n^{s+k}(-1)^{r+1-k} &=&
  \sum_k \left( C_r^{k-1} + C_r^k \right) C_n^{s+k} (-1)^{r+1-k} \\
  &=& \sum_k C_r^k C_n^{s+k+1} (-1)^{r-k} - \sum_k C_r^k C_n^{s+k}
  (-1)^{r-k} \\
  &=& C_{s+1}^{n-r} - C_s^{n-r},\mbox{ by induction} \\
  &=& C_s^{n-r} \left( \frac{s+1}{s+1+r-n} - 1 \right) \\
  &=& C_s^{n-r} \frac{n-r}{s+1+r-n} \\
  &=& C_s^{n-r-1}
\end{eqnarray*}

\newpar{20} We have
\begin{eqnarray*}
  \sum_{k=0}^r C_{r-k}^m C_s^{k-t} (-1)^{k-t} &=&
  (-1)^{r-m-t} \sum_{k=0}^r C_{-(m+1)}^{r-k-m} C_s^{k-t},\mbox{ from
    (19)} \\
  &=& (-1)^{r-m-t} \sum_{k=-t}^{r-t} C_s^k C_{-(m+1)}^{r-k-t-m} \\
  &=& (-1)^{r-m-t} \sum_k C_s^k C_{-(m+1)}^{r-k-t-m} \\
  &=& (-1)^{r-m-t} C_{s-m-1}^{r-t-m}, \mbox{ from (21)} \\
  &=& C_{r-t-s}^{r-t-m},\mbox{ from (17)}
\end{eqnarray*}

In the same vein, we have
\begin{eqnarray*}
  \sum_{k=0}^r C_{r-k}^m C_{s+k}^n &=& \sum_{k=0}^r C_{r-k}^m
  C_{-(m+1)}^{s+k-n} (-1)^{s+k-n}, \mbox{ using (19)} \\
  &=& C_{r+s+1}^{r+s-m-n}, \mbox{ using (24) with $t = n-s$} \\
  &=& C_{r+s+1}^{m+n+1}
\end{eqnarray*}

\newpar{21} Because in the definition of $C_n^k$, $k$ should be an
integer.

\newpar{22} For $s = n-1-r+nt$, we have
\[\sum_{k\ge 0} C_{r-tk}^k C_{s-t(n-k)}^{n-k} \frac{r}{r-tk} = \sum_{k\ge 0} C_{r-tk}^k
  C_{n-1-r+tk}^{n-k} \frac{r}{r-tk} \mbox{ (*)}\]

Let's suppose that $r$ and $t$ are integers. If $k \le r - tk$ then
\[ (n-1-r+tk) - (n-k) \le -1 < 0.\]

Hence $C_{r-tk}^k C_{n-1-r+tk}^{n-k} = 0$.  Thus we deduce that (*) is
equal to $0$ for $r$ and $t$ integers.

But the expression in (*) is a polynomial in $r$.  For each fix integer
$t$, the corresponding polynomial in $r$ has an infinity of root for
$r$ integer.  We then deduce that (*) holds for $r$ real and $t$
integer.

Similarly, (*) is also a polynomial in $t$.  Using the same reasoning,
we deduce finally for $r$ and $t$ real numbers
\[\sum_{k\ge 0} C_{r-tk}^k C_{n-1-r+tk}^{n-k} \frac{r}{r-tk} = 0 =
C_{n-1}^n.\]

\newpar{23}  Suppose that (26) holds for $(r,s,t,n)$ and
$(r,s-t,t,n-1)$.  Then we have
\begin{eqnarray*}
  S(r,s+1,t,n) &=& \sum_{k\ge 0} C_{r-tk}^k C_{s+1-t(n-k)}^{n-k}
  \frac{r}{r-tk}\\
  &=& \sum_{k\ge 0} C_{r-tk}^k \left( C_{s-t(n-k)}^{n-k-1} +
  C_{s-t(n-k)}^{n-k}\right) \frac{r}{r-tk} \\
  &=& S(r,s-t,t,n-1) + S(r,s,t,n) \\
  &=& C_{r+s-tn}^{n-1} + C_{r+s-tn}^n \\
  &=& C_{r+s+1-tn}^n
\end{eqnarray*}

\newpar{24} Each member of the equality in (26) are polynomials in
$s$.  And \textbf{22} and \textbf{23} implies that these polynomials
are equals on an infinite number of reals.  We then deduce that
they're equals.

\newpar{25} If we have $t=0$, we're reduced to the binomial theorem.
Suppose that $t \not= 0$.
\begin{eqnarray*}
  \left| \frac{A_{k+1}(r,t)}{A_k(r,t)} \right| &=&
  \left| \frac{r-(k+1)t - k}{k+1}\times \frac{r-kt}{r-(k+1)t} \prod_{i=0}^{k-1}
  \frac{r - (k+1)t - i}{r - kt - i}\right| \\
  &\le& 2(|r|+|t|+1) \left| \prod_{i=0}^{k-1} \frac{r - (k+1)t - i}{r - kt
    - i}\right| \\
  &=& 2(|r|+|t|+1) \left| \prod_{i=0}^{k-1}\left( 1 +
  \frac{t}{kt+i-r}\right)\right|
\end{eqnarray*}

If $t > 0$ and we suppose that $k \ge 1 + |r/t|$, we then deduce that
\begin{eqnarray*}
  \left| \frac{A_{k+1}(r,t)}{A_k(r,t)}\right| &\le& 2(|r|+|t|+1)
  \left(1 + \frac{t}{kt-|r|}\right)^k \\
  &\le& 2(|r|+|t|+1) \exp\left(\frac{k}{kt-|r|}\right) \\
  &\le& 2(|r|+|t|+1) \exp(1 + |r|) \mbox{ (*)}
\end{eqnarray*}

We have also,
\[ A_k(r,-t) = (-1)^k A_k(-r-1,t-1) \frac{-r-1-k(t-1)}{r+kt}\]
We then deduce from (*) that if $t < -1$, then the ratio of two
consecutive terms of the sequel $A_k(r,t)$ is bounded.  Suppose now
that $-1 \le t < 0$. Then for $k \ge \frac{r+1}{1-|t|}$, we have
\[ k-1 \ge r+k|t| = r-kt \ge \lfloor r-kt\rfloor.\]
If we note $\{x\} = x - \lfloor x\rfloor$, then we have
\begin{eqnarray*}
  |A_k(r,t)| &=& \frac{|r|}{k!} \left| \prod_{i=1}^{k-1} (r-kt -i)
  \right| \\
  &=&  \frac{|r|}{k!} \prod_{i=1}^{\lfloor r-kt\rfloor}(r-kt -i)
  \prod_{i=\lfloor r-kt\rfloor + 1}^{k-1}(i-(r-kt)) \\
  &\le& \frac{|r|}{k!} \prod_{i=0}^{\lfloor r-kt\rfloor-1}(\{r-kt\} +
  i) \prod_{i=\lfloor r-kt\rfloor +1}^{k-1} i \\
  &\le& \frac{|r|}{k!} (k-1)! \\
  &=& |r|
\end{eqnarray*}

When then deduce that $\sum_k A_k(r,t) z^k$ is defined for $|z|$ small
enough.  Using (26) with $s = tn$, we have
\[ \sum_k A_k(r,t) C_{tk}^{n-k} = C_r^n.\]

Hence
\begin{eqnarray*}
  x^r &=& \sum_k C_r^k (x-1)^k \\
  &=& \sum_k \sum_j A_j(r,t) C_{tj}^{k-j} (x-1)^k \\
  &=& \sum_j A_j(r,t) \sum_k C_{tj}^{k-j} (x-1)^k \\
  &=& \sum_j A_j(r,t) \sum_k C_{tj}^k (x-1)^{k+j} \\
  &=& \sum_j A_j(r,t) (x-1)^j x^{tj} \\
  &=& \sum_j A_j(r,t) z^k
\end{eqnarray*}

\newpar{26} From \textbf{25.} we have
\[ 1 = \sum_k A_k(r,t) x^{tk-r}x^k.\]

Hence
\begin{eqnarray*}
  0 &=& \frac{d}{dx} \left( \sum_k A_k(r,t) x^{tk-r}(x-1)^k \right) \\
  &=& \sum_k A_k(r,t) ((tk-r) x^{-r-1} + k x^{-r}(x-1)^{-1}) z^k \\
  &=& - r x^{-1} \sum_k C_{r-tk}^k z^k + \sum_k A_k(r,t) k
  (x-1)^{-1} z^k \\
\end{eqnarray*}

Hence
\begin{eqnarray*}
  \sum_k C_{r-tk}^k z^k &=& \frac{x^{r+1}z}{r(x-1)}\ \frac{d}{dz}
  \left( \sum_k A_k(r,t) z^k\right) \\
  &=& \frac{x^{1+t}}{r} \frac{d}{dz} x^r \\
  &=& x^{t+r} \frac{dx}{dz}
\end{eqnarray*}

But by definition we have
\[  1 = \frac{dz}{dz} = \frac{d}{dz}(x^{t+1}-x^t) = x^{t-1}((t+1)x - t)
\frac{dx}{dz} \]

And finally
\[ \sum_k C_{r-tk}^k z^k = \frac{x^{r+1}}{(t+1)x - t} .\]
\end{document}
