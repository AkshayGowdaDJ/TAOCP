\documentclass[a4paper,12pt]{article}
\newcommand{\newpar}[1]{\bigskip \noindent \textbf{#1.}}
\newcommand{\subpar}[1]{\medskip \noindent (#1)}
\newcommand{\stirlingone}[2]{{#1 \atopwithdelims[]#2}}
\newcommand{\stirlingtwo}[2]{{#1 \atopwithdelims\{\}#2}}
\newcommand{\la}{\leftarrow}
\newcommand{\ra}{\rightarrow}
\begin{document}

\newpar{1} $n$.

\newpar{2} $1$.

\newpar{3} ${{52} \choose {13}} = 635013559600$.

\newpar{4} $2^4\,41\,43\,23\,17\,5^2\,7^2\,47$.

\newpar{5} We have
\begin{eqnarray*}
  11^4 &=& (10 + 1)^4 \\
  &=& {4 \choose 0} 10^4 + {4 \choose 1} 10^3 + {4 \choose 2} 10^2 +
  {4 \choose 1} 10 + {4 \choose 4} \\
  &=& 10^4 + 4\,10^3 + 6\,10^2 + 4\,10 + 1 \\
  &=& 14641
\end{eqnarray*}

\newpar{6} Using (4) and (9) we have,
\begin{quote}
  \begin{tabular}{|c|c|c|c|c|c|c|c|c|c|c|}
    \hline $r$ & ${r \choose 0}$ & ${r \choose 1}$ & ${r \choose 2}$ &
    ${r \choose 3}$ & ${r \choose 4}$ &
    ${r \choose 5}$ & ${r \choose 6}$ & ${r \choose 7}$ & ${r \choose
      8}$ & ${r \choose 9}$ \\
    \hline $-3$ & $1$ & $-3$ & $6$ & $-10$ & $15$ & $-21$ & $28$ &
    $-36$ & $45$ & $-55$ \\
    \hline $-2$ & $1$ & $-2$ & $3$ & $-4$ & $5$ & $-6$ & $7$ & $-8$ &
    $9$ & $-10$ \\
    \hline $-1$ & $1$ & $-1$ & $1$ & $-1$ & $1$ & $-1$ & $1$ & $-1$ &
    $1$ & $-1$ \\
    \hline
  \end{tabular}
\end{quote}

\newpar{7}  Since we have ${n \choose k} = {n \choose {n-k}}$, we can impose that $k
\le n-k$ without a loss of generality.  But we have,

\[ {n \choose k} = \frac{n-k+1}{k}\ {n \choose {k-1}} \]

We then deduce that ${n \choose k} \ge {n \choose {k-1}}$ if $k \le n-k$.  Thus, the
value of $k$ that achieves the maximum is $\lfloor n/2\rfloor$.

\newpar{8} Leaving out the zeroes, the triangle is symmetric according
to the line $k = \lfloor r/2\rfloor$.

\newpar{9} $1$ if $n$ is positive or equal to zero and $0$ otherwise.

\newpar{10} Suppose that $p$ is prime.

\subpar{a} We have

\begin{eqnarray*}
  n! &=& \prod_{k=1}^n k \\
  &=& \left( \prod_{0 \le l \le \lfloor n/p\rfloor - 1}\ \prod_{lp <
    k\le (l+1)p} k \right) \left( \prod_{p \lfloor n/p\rfloor + 1\le k \le
    n} k \right) \\
  &=& \left( \prod_{0\le l \le \lfloor n/p\rfloor - 1} \prod_{0 < k
    \le p} (k+lp) \right) \left( \prod_{1 \le k \le n - p \lfloor n/p \rfloor}
  (k+p\lfloor n/p\rfloor) \right)
\end{eqnarray*}

Hence we deduce,

\begin{eqnarray*}
  p!(n-p)! &=& p! \left( \prod_{0\le l \le \lfloor (n-p)/p\rfloor-1}
  \prod_{0<k\le p} (k+lp)\right) \\ &&
  \left( \prod_{1\le k\le n-p(\lfloor (n-p)/p\rfloor+1)} (k + p \lfloor
  (n-p)/p\rfloor)\right)
\end{eqnarray*}

Given that $\lfloor (n-p)/p \rfloor = \lfloor n/p\rfloor - 1$, we then
deduce


\[  {n \choose p} = \lfloor n/p\rfloor \frac{\prod_{0< k < p}(k + p(\lfloor
    n/p\rfloor - 1))}{(p-1)!}\,
  \prod_{1\le k\le n-p\lfloor n/p\rfloor }
  \frac{k+p\lfloor n/p\rfloor}{k+p\lfloor (n-p)/p\rfloor} \]

Given that $p$ is prime, for $k$ varying between $1$ and $p-1$, $k$ is
invertible in $Z/pZ$. We then deduce that all the terms that appear as
denominators in the previous equality are invertible terms.  Hence

\[ {n \choose p} \equiv \lfloor n/p\rfloor \pmod p\]

\subpar{b} Let $k$ be an integer between $1$ and $p-1$.  We have
\[ p! = {p \choose k} k! (p-k)! \]

Since $p$ is prime, none of the term that appears in the two
factorial divide $p$.  We then deduce that $k! (p-k)!$ divides
$(p-1)!$. By rewriting the equality as
\[ {p \choose k} = p\,\frac{(p-1)!}{k!(p-k)!}\]

we deduce that ${p \choose k} \equiv 0 \pmod p$.

\subpar{c} Let $k$ be an integer between $0$ and $p-1$. We have
\begin{eqnarray*}
  {{p-1} \choose k} &=& \frac{\prod_{p-k \le i\le p-1} i}{k!} \\
  &=& \frac{\prod_{1\le i\le k} (p-i)}{k!}
\end{eqnarray*}

Given that $p$ is prime, the denominator is invertible in $Z/pZ$,
hence
\[ {{p-1} \choose k} \equiv (-1)^k \pmod p.\]

\subpar{d} Let $k$ be an integer between $2$ and $p-1$. We then have $p+2-k
\le p$, and since

\[  k!{{p+1} \choose k} = \prod_{p+2-k\le i\le p+1} i \]

we have $k! {{p+1} \choose k} \equiv 0 \pmod p$ and since $k!$ is invertible, we
then deduce
\[ {{p+1} \choose k} \equiv 0 \pmod p.\]

\subpar{e} We have

\begin{eqnarray*}
  n! &=& \prod_{1\le i\le n} i \\
  &=& \left( \prod_{0 \le j \le \lfloor n/p\rfloor - 1} \prod_{jp < i
    \le (j+1)p} i \right) \left( \prod_{p\lfloor n/p\rfloor + 1 \le i
    \le n} i\right) \\
  &=& \left( \prod_{0 \le j\le \lfloor n/p\rfloor - 1} (j+1)p
  \prod_{jp < i <(j+1)p} i \right)\left( \prod_{p\lfloor n/p\rfloor +
    1 \le i \le n} i\right) \\
  &=& p^{\lfloor n/p\rfloor} \lfloor n/p\rfloor!
  \left( \prod_{ 0 \le j \le \lfloor n/p\rfloor - 1} \prod_{jp < i <
    (j+1)p} i\right)\left( \prod_{p\lfloor n/p\rfloor + 1 \le i \le n}
  i\right) \\
  &=& p^{\lfloor n/p\rfloor} \lfloor n/p\rfloor! \left( \prod_{0\le j
    \le \lfloor n/p\rfloor - 1} \prod_{0 < i < p}(i+jp)\right)
  \prod_{1\le i\le n - p \lfloor n/p\rfloor} (i+p\lfloor n/p\rfloor) \\
  n!  &=& p^{\lfloor n/p\rfloor} \lfloor n/p\rfloor! \left( \prod_{0\le j
    \le \lfloor n/p\rfloor - 1} \prod_{0 < i < p}(i+jp)\right)
  \prod_{1\le i\le n \bmod p} (i+p\lfloor n/p\rfloor)\, \mbox{(*)}
\end{eqnarray*}

On the other hand, we have

\begin{eqnarray*}
  \left\lfloor \frac{k}{p}\right\rfloor + \left\lfloor
  \frac{n-k}{p}\right\rfloor &=& \left\lfloor \frac{n-k}{p} + \left\lfloor
  \frac{k}{p}\right\rfloor\right\rfloor \\
  &=& \left\lfloor \frac{n - k \bmod p}{p}\right\rfloor \\
  &=& \left\lfloor \left\lfloor \frac{n}{p}\right\rfloor + \frac{n
    \bmod p - k\bmod p}{p}\right\rfloor \\
  &=& \left\lfloor \frac{n}{p}\right\rfloor + \left\lfloor  \frac{n
    \bmod p - k\bmod p}{p}\right\rfloor \\
  &=& \left\lfloor \frac{n}{p} \right\rfloor - [ n\bmod p < k\bmod p ]
\end{eqnarray*}

From (*), we then deduce

\begin{eqnarray*}
  {n \choose k} &=& p^{[n\bmod p < k \bmod p]} \times \frac{\lfloor
    n/p\rfloor!}{\lfloor k/p\rfloor!\lfloor (n-k)/p\rfloor!} \times
  \\ &&
  \frac{\prod_{j=\lfloor (n-k)/p\rfloor}^{\lfloor
      n/p\rfloor-1}\prod_{0<i<p}(i+jp)} {\prod_{j=0}^{\lfloor
      k/p\rfloor-1}\prod_{0<i<p}(i+jp)} \times
  \frac{\prod_{1\le i\le n\bmod p} (i+p\lfloor
    n/p\rfloor)}{\prod_{1\le i\le k\bmod p} (i+p\lfloor k/p\rfloor)}
  \times \\ &&
  \frac{1}{\prod_{1\le i\le n\bmod p} (i+p\lfloor n/p\rfloor)}\\
\end{eqnarray*}

We then deduce that if $n \bmod p < k \bmod p$, then
\[ {n \choose k} \equiv 0 \pmod p.\]

Otherwise, we have

\[ {n \choose k} \equiv { \lfloor n/p\rfloor \choose \lfloor k/p
  \rfloor } {n \bmod p \choose k \bmod p} \pmod p.\]

Which is also true if $n \bmod p < k \bmod p$.

\subpar{f} The result is easily obtained by induction using (e).

\newpar{11} Let $(a_i)_{i \in N}$, $(b_i)_{i \in N}$ and $(c_i)_{i \in
  N}$ three sequels representing respectively  $a$, $b$ and $a+b$ in
the $p$-ary number system.  Let's remark that all three of them become
stationary to $0$ for a sufficiently large value of $i$. Using
1.2.5--12, we have

\begin{eqnarray*}
  \mu\left( {{a+b} \choose a} \right) &=& \mu((a+b)!) - \mu(a!) - \mu(b!) \\
  &=& \frac{\sum_{i \in N} (a_i + b_i - c_i)}{p-1}
\end{eqnarray*}

We can partition $N$ into intervals that distinguish the additions
that produce carries and those which don't.  Let $(i_k)_{i \in N}$ and
$(j_k)_{k \in N}$ be two increasing sequels such that
\begin{eqnarray*}
  a_{i_k} + b_{i_k} &=& c_{i_k} + p \\
  a_{i_k+1} + b_{i_k + 1} + 1 &=& c_{i_k + 1} + p \\
  \ldots \\
  a_{j_k - 1} + b_{j_k - 1} + 1 &=& c_{j_k - 1} + p \\
  a_{j_k} + b_{j_k} + 1 &=& c_{j_k}
\end{eqnarray*}

The last equality implies that $j_k < i_{k+1}$ and we have
\[ \sum_{i_k \le i \le j_k} (a_i + b_i - c_i) = (p-1) (j_k - i_k +
1),\]
and obviously
\[ \sum_{j_k < i < i_{k+1}} (a_i + b_i - c_i) = 0\]
because none of the terms produce a carry.

So finally,
\begin{eqnarray*}
  \mu\left( {{a+b} \choose a}\right) &=& \frac{\sum_{k \in N} \sum_{i_k \le i
      < i_{k+1}} (a_i + b_i - c_i)}{p-1} \\
  &=& \sum_{k \in N} (j_k - i_k + 1)
\end{eqnarray*}

We then deduce that the multiplicity of $p$ is the number of carries
that occur when $a$ is added to $b$ in the $p$-ary number system.

\newpar{12} Using \textbf{f} from exercise \textbf{10}, if the $2$-ary
system representations of $n$ and $k$ when $k \le n$ are
\[
\begin{array}{l}
  n = a_r 2^r + \cdots + a_1 2 + a_0, \\
  k = b_r 2^r + \cdots + b_1 2 + b_0,
\end{array}
\mbox{ then } {n \choose k} \equiv {{a_r} \choose {b_r}} \ldots {{a_1}
  \choose {b_1}}
{{a_0} \choose {b_0}} \equiv 1 \pmod 2
\]

Hence $a_i \ge b_i$ for all the choices of $k \le n$, otherwise ${n
  \choose k}
\equiv 0 \pmod 2$. This is only possible if $a_i = 1$ for $0 \le i \le
r$.  Hence $n = 2^{r+1}-1$.

\newpar{13} Let's show the formula by induction on $n \ge 0$.  If
$n=0$, we have

\[ \sum_{k=0}^0 {{r+k} \choose k} = 1 = {{r+1} \choose 0}.\]

Suppose that we have the equality for $n$, then we have

\begin{eqnarray*}
  \sum_{0\le k\le n+1}{{r+k} \choose k} &=& \sum_{0\le k \le n}{{r+k}
    \choose k} +
  {{r+n+1} \choose {n+1}} \\
  &=& {{r+n+1} \choose n} + {{r+n+1} \choose {n+1}}, \mbox{ by induction } \\
  &=& {{r+n+2} \choose {n+1}}
\end{eqnarray*}

\newpar{14} Let $k$ be an integer between $0$ and $n$.  We have
\[ (k+1)^4 - k^4 = 4 k^3 + 6 k^2 + 4 k + 1\]

Hence
\[ (n+1)^4 = 4 \sum_{k=0}^n k^3 + 6 \sum_{k=0}^n k^2 + 4 \sum_{k=0}^n k
+ (n+1)\]

Thus
\[ 4 \sum_{k=0}^n k^3 = (n+1)^4 - 2 n\left(n +
\frac{1}{2}\right)(n+1) - 2 n(n+1) - (n+1)\]

And finally
\[ \sum_{k=0}^n k^3 = \left( \frac{n(n+1)}{2} \right)^2 (*)\]

Similarly
\[ (k+1)^5 - k^5 = 5 k^4 + 10 k^3 + 10 k^2 + 5 k + 1.\]

Hence
\[ (n+1)^5 = 5 \sum_{k=0}^n k^4 + 10 \sum_{k=0}^n k^3 + 10
\sum_{k=0}^n k^2 + 5 \sum_{k=0}^n k + (n+1).\]

From (*), we then deduce
\[ 5 \sum_{k=0}^n k^4 = (n+1)^5 - \frac{5}{2} n^2(n+1)^2 -
\frac{10}{3}n\left(n+\frac{1}{2}\right) (n+1) -
\frac{5}{2}n(n+1)-(n+1).\]

And finally

\[ \sum_{k=0}^n k^4 = \frac{1}{5}n(n+2)\left(n+\frac{1}{2}\right)
\left(n^2 + n - \frac{1}{3}\right).\]

\newpar{15} Let's show the property by induction on $r$.  Suppose
$r=0$, then we have
\[ (x+y)^0 = 1 = \sum_k {0 \choose k} x^k y^{-k}.\]

Suppose that we have the property for $r$, then we have

\begin{eqnarray*}
  (x+y)^{r+1} &=& (x + y) (x+y)^n \\
  &=& (x+y) \sum_k {r \choose k} x^k y^{r-k}, \mbox{ by induction} \\
  &=& \sum_k {r \choose k} x^{k+1} y^{r-k} + \sum_k {r \choose k} x^k y^{r+1-k} \\
  &=& \sum_k {r \choose {k-1}} x^k y^{r+1-k} + \sum_k {r \choose k} x^k y^{r+1-k} \\
  &=& \sum_k {{r+1} \choose k} x^k y^{r+1-k}
\end{eqnarray*}

\newpar{16} We have

\begin{eqnarray*}
  (-1)^k {{-k} \choose {n-1}} &=& (-1)^k \frac{-k (-k-1)\ldots
    (-k-n+2)}{(n-1)!} \\
  &=& (-1)^k (-n) (-n-1)\ldots(-k-n+2) \frac{(-1)^{n-k}}{(k-1)!} \\
  &=& (-1)^n {{-n} \choose {k-1}}
\end{eqnarray*}

\newpar{17} We have
\begin{eqnarray*}
  (1+x)^{r+s} &=& (1+x)^r (1+x)^s \\
  &=& \sum_k {r \choose k} x^k \sum_k {s \choose k} x^k \\
  &=& \sum_n \sum_k {r \choose k} {s \choose {n-k}} x^n \\
\end{eqnarray*}

The two power series are equals thus their coefficients are equals
too.  Thus
\[ {{r+s} \choose n} = \sum_k {r \choose k} {s \choose {n-k}}.\]

\newpar{18} We have
\begin{eqnarray*}
  \sum_k {r \choose {m+k}} {s \choose {n+k}} &=& \sum_k {r \choose k}
      {s \choose {n-m+k}} \\
  &=& \sum_k {r \choose k} {s \choose {s+m-n-k}},\ \mbox{from (6)} \\
  &=& {{r+s} \choose {s+m-n}}, \mbox{from (21)}\\
  &=& {{r+s} \choose {r-m+n}}, \mbox{from (6)}
\end{eqnarray*}

\newpar{19} Let's show the equality by induction on $r \ge 0$.  If
$r=0$, we have
\begin{eqnarray*}
  \sum_k {0 \choose k} {s+k \choose n}(-1)^{-k} &=& \sum_k \delta_{k,0}
  {s+k \choose n}(-1)^{k} \\
  &=& {s \choose n}
\end{eqnarray*}

Suppose that we have the equality for $r$, then we have
\begin{eqnarray*}
  \sum_k {{r+1} \choose k} {s+k \choose n}(-1)^{r+1-k} &=&
  \sum_k \left( {r \choose {k-1}} + {r \choose k} \right) {s+k \choose n} (-1)^{r+1-k} \\
  &=& \sum_k {r \choose k} {s+k+1 \choose n} (-1)^{r-k} -\\
  && \sum_k {r \choose k} {s+k \choose n}
  (-1)^{r-k} \\
  &=& {{s+1} \choose {n-r}} - {s \choose {n-r}},\mbox{ by induction} \\
  &=& {s \choose {n-r}} \left( \frac{s+1}{s+1+r-n} - 1 \right) \\
  &=& {s \choose {n-r}} \frac{n-r}{s+1+r-n} \\
  &=& {s \choose {n-r-1}}
\end{eqnarray*}

\newpar{20} Using (19), we have
\begin{eqnarray*}
  \sum_{k=0}^r {{r-k} \choose m} {s \choose {k-t}} (-1)^{k-t} &=&
  (-1)^{r-m-t} \sum_{k=0}^r {{-(m+1)} \choose {r-k-m}} {s \choose
    {k-t}} \\
  &=& (-1)^{r-m-t} \sum_{k=-t}^{r-t} {s \choose k} {{-(m+1)} \choose {r-k-t-m}} \\
  &=& (-1)^{r-m-t} \sum_k {s \choose k} {{-(m+1)} \choose {r-k-t-m}} \\
  &=& (-1)^{r-m-t} {{s-m-1} \choose {r-t-m}}, \mbox{ from (21)} \\
  &=& {{r-t-s} \choose {r-t-m}},\mbox{ from (17)}
\end{eqnarray*}

In the same vein, we have
\begin{eqnarray*}
  \sum_{k=0}^r {{r-k} \choose m} {{s+k} \choose n} &=& \sum_{k=0}^r
      {{r-k} \choose m}
  {{-(m+1)} \choose {s+k-n}} (-1)^{s+k-n}, \mbox{ using (19)} \\
  &=& {{r+s+1} \choose {r+s-m-n}}, \mbox{ using (24) with $t = n-s$} \\
  &=& {{r+s+1} \choose {m+n+1}}
\end{eqnarray*}

\newpar{21} Because in the definition of ${n \choose k}$, $k$ should be an
integer.

\newpar{22} For $s = n-1-r+nt$, we have
\begin{eqnarray*}
  \sum_{k\ge 0} {{r-tk} \choose k} {{s-t(n-k)} \choose {n-k}} &&
  \frac{r}{r-tk} = \\ &&\sum_{k\ge 0} {{r-tk} \choose k}
       {{n-1-r+tk} \choose {n-k}} \frac{r}{r-tk} \mbox{ (*)}
\end{eqnarray*}

Let's suppose that $r$ and $t$ are integers. If $k \le r - tk$ then
\[ (n-1-r+tk) - (n-k) \le -1 < 0.\]

Hence ${{r-tk} \choose k} {{n-1-r+tk} \choose {n-k}} = 0$.  Thus we deduce that (*) is
equal to $0$ for $r$ and $t$ integers.

But the expression in (*) is a polynomial in $r$.  For each fix integer
$t$, the corresponding polynomial in $r$ has an infinity of root for
$r$ integer.  We then deduce that (*) holds for $r$ real and $t$
integer.

Similarly, (*) is also a polynomial in $t$.  Using the same reasoning,
we deduce finally for $r$ and $t$ real numbers
\[\sum_{k\ge 0} {{r-tk} \choose k} {{n-1-r+tk} \choose {n-k}}
\frac{r}{r-tk} = 0 = {{n-1} \choose n}.\]

\newpar{23}  Suppose that (26) holds for $(r,s,t,n)$ and
$(r,s-t,t,n-1)$.  Then we have
\begin{eqnarray*}
  S(r,s+1,t,n) &=& \sum_{k\ge 0} {{r-tk} \choose k} {{s+1-t(n-k)}
    \choose {n-k}}
  \frac{r}{r-tk}\\
  &=& \sum_{k\ge 0} {{r-tk} \choose k} \left( {{s-t(n-k)} \choose {n-k-1}} +
  {{s-t(n-k)} \choose {n-k}}\right) \frac{r}{r-tk} \\
  &=& S(r,s-t,t,n-1) + S(r,s,t,n) \\
  &=& {{r+s-tn} \choose {n-1}} + {{r+s-tn} \choose n} \\
  &=& {{r+s+1-tn} \choose n}
\end{eqnarray*}

\newpar{24} Each member of the equality in (26) are polynomials in
$s$.  And \textbf{22} and \textbf{23} implies that these polynomials
are equals on an infinite number of reals.  We then deduce that
they're equals.

\newpar{25} If we have $t=0$, we're reduced to the binomial theorem.
Suppose that $t \not= 0$.
\begin{eqnarray*}
  \left| \frac{A_{k+1}(r,t)}{A_k(r,t)} \right| &=&
  \left| \frac{r-(k+1)t - k}{k+1}\times \frac{r-kt}{r-(k+1)t} \prod_{i=0}^{k-1}
  \frac{r - (k+1)t - i}{r - kt - i}\right| \\
  &\le& 2(|r|+|t|+1) \left| \prod_{i=0}^{k-1} \frac{r - (k+1)t - i}{r - kt
    - i}\right| \\
  &=& 2(|r|+|t|+1) \left| \prod_{i=0}^{k-1}\left( 1 +
  \frac{t}{kt+i-r}\right)\right|
\end{eqnarray*}

If $t > 0$ and we suppose that $k \ge 1 + |r/t|$, we then deduce that
\begin{eqnarray*}
  \left| \frac{A_{k+1}(r,t)}{A_k(r,t)}\right| &\le& 2(|r|+|t|+1)
  \left(1 + \frac{t}{kt-|r|}\right)^k \\
  &\le& 2(|r|+|t|+1) \exp\left(\frac{k}{kt-|r|}\right) \\
  &\le& 2(|r|+|t|+1) \exp(1 + |r|) \mbox{ (*)}
\end{eqnarray*}

We have also,
\[ A_k(r,-t) = (-1)^k A_k(-r-1,t-1) \frac{-r-1-k(t-1)}{r+kt}\]
We then deduce from (*) that if $t < -1$, then the ratio of two
consecutive terms of the sequel $A_k(r,t)$ is bounded.  Suppose now
that $-1 \le t < 0$. Then for $k \ge \frac{r+1}{1-|t|}$, we have
\[ k-1 \ge r+k|t| = r-kt \ge \lfloor r-kt\rfloor.\]
If we note $\{x\} = x - \lfloor x\rfloor$, then we have
\begin{eqnarray*}
  |A_k(r,t)| &=& \frac{|r|}{k!} \left| \prod_{i=1}^{k-1} (r-kt -i)
  \right| \\
  &=&  \frac{|r|}{k!} \prod_{i=1}^{\lfloor r-kt\rfloor}(r-kt -i)
  \prod_{i=\lfloor r-kt\rfloor + 1}^{k-1}(i-(r-kt)) \\
  &\le& \frac{|r|}{k!} \prod_{i=0}^{\lfloor r-kt\rfloor-1}(\{r-kt\} +
  i) \prod_{i=\lfloor r-kt\rfloor +1}^{k-1} i \\
  &\le& \frac{|r|}{k!} (k-1)! \\
  &\le& |r|
\end{eqnarray*}

When then deduce that $\sum_k A_k(r,t) z^k$ is defined for $|z|$ small
enough.  Using (26) with $s = tn$, we have
\[ \sum_k A_k(r,t) {{tk} \choose {n-k}} = {r \choose n}.\]

Hence
\begin{eqnarray*}
  x^r &=& \sum_k {r \choose k} (x-1)^k \\
  &=& \sum_k \sum_j A_j(r,t) {{tj} \choose {k-j}} (x-1)^k \\
  &=& \sum_j A_j(r,t) \sum_k {{tj} \choose {k-j}} (x-1)^k \\
  &=& \sum_j A_j(r,t) \sum_k {{tj} \choose k} (x-1)^{k+j} \\
  &=& \sum_j A_j(r,t) (x-1)^j x^{tj} \\
  &=& \sum_j A_j(r,t) z^k
\end{eqnarray*}

\newpar{26} From \textbf{25.} we have
\[ 1 = \sum_k A_k(r,t) x^{tk-r}x^k.\]

Hence
\begin{eqnarray*}
  0 &=& \frac{d}{dx} \left( \sum_k A_k(r,t) x^{tk-r}(x-1)^k \right) \\
  &=& \sum_k A_k(r,t) ((tk-r) x^{-r-1} + k x^{-r}(x-1)^{-1}) z^k \\
  &=& - r x^{-1} \sum_k {{r-tk} \choose k} z^k + \sum_k A_k(r,t) k
  (x-1)^{-1} z^k \\
\end{eqnarray*}

Hence
\begin{eqnarray*}
  \sum_k {{r-tk} \choose k} z^k &=& \frac{x^{r+1}z}{r(x-1)}\ \frac{d}{dz}
  \left( \sum_k A_k(r,t) z^k\right) \\
  &=& \frac{x^{1+t}}{r} \frac{d}{dz} x^r \\
  &=& x^{t+r} \frac{dx}{dz}
\end{eqnarray*}

But by definition we have
\[  1 = \frac{dz}{dz} = \frac{d}{dz}(x^{t+1}-x^t) = x^{t-1}((t+1)x - t)
\frac{dx}{dz} \]

And finally
\[ \sum_k {{r-tk} \choose k} z^k = \frac{x^{r+1}}{(t+1)x - t} .\]

\newpar{27} We have using \textbf{25}
\begin{eqnarray*}
  \sum_n A_n(r+s, t) z^n &=& x^{r+s} \\
  &=& x^r x^s \\
  &=& \sum_i A_i(r, t) z^i \sum_j A_j(s, t) z^j \\
  &=& \sum_n \sum_{i,j \atop i+j=n} A_i(r,t) A_j(s,j) z^n \\
  &=& \sum_n \sum_i A_i(r,t) A_{n-i}(s,t)z^n
\end{eqnarray*}
We then deduce that
\[ A_n(r+s,t) = \sum_k A_k(r,t) A_{n-k}(s,t).\]

Using \textbf{25} and \textbf{26}, we have
\begin{eqnarray*}
  x^r \times \frac{x^{s+1}}{(t+1)x-t} &=& \sum_i A_i(r,t) z^i \times
  \sum_j {{s-tj} \choose j} z^j \\
  &=& \sum_n \sum_i A_i(r,t) {{s-t(n-i)} \choose {n-i}} z^n
\end{eqnarray*}

But we have also from \textbf{26}
\begin{eqnarray*}
  \frac{x^{r+s+1}}{(t+1)x-t} &=& \sum_n {{r+s-tn} \choose n} z^n
\end{eqnarray*}
We then deduce that
\[ \sum_i A_i(r,t) {{s-t(n-i)} \choose {n-i}} = {{r+s-tn} \choose n}.\]

\newpar{28}  Let's show the equality by induction on the integer $n
\ge 0$.  If $n=0$, then we're reduced to the equation (21).  Suppose
that we have the equality for $n$ for all values of $r,s,t$.  Then we
have using (26)
\begin{eqnarray*}
  \sum_k {r+tk \choose k}{s-tk\choose n+1-k} &=&
  \sum_k {r+tk \choose k}{s-tk\choose n+1-k}\frac{r}{r+tk} +\\
  && \sum_k {r+tk \choose k}{s-tk\choose n+1-k}\frac{tk}{r+tk} \\
  &=& {r + s \choose n+1} + t \sum_k {r+tk-1 \choose k-1}{s-tk \choose
    n+1-k} \\
  &=& {r+s \choose n+1} + t \sum_k {r+t-1 + tk \choose k}{s-t-tk
    \choose n-k} \\
\end{eqnarray*}
And by induction we then deduce that
\begin{eqnarray*}
\sum_k {r+tk \choose k}{s-tk\choose n+1-k}  &=& {r+s \choose n+1} + t
\sum_{k\ge 0} {r+s-1-k\choose n+1-k} t^k \\
&=& \sum_{k\ge 0} {r+s-k \choose n+1-k} t^k
\end{eqnarray*}

\newpar{29} We have
\begin{eqnarray*}
  \sum_k {r \choose k} (-1)^{r-k} \sum_{i=0}^r b_i k^i &=&
  \sum_{k=0}^r {r \choose k} (-1)^{r-k} \sum_{i=0}^r b_i k^i \\
  &=& \sum_{i=0}^r b_i \sum_{k=0}^r {r \choose k} (-1)^{r-k} k^i \\
  &=& r! \sum_{i=0}^r b_i \sum_{k=0}^r \frac{k^i}{(-1)^{r-k}k!(r-k)!}
\end{eqnarray*}
But we have
\begin{eqnarray*}
  (-1)^{r-k}k! (r-k)! &=& \prod_{i=1}^k i \prod_{i=1}^{r-k}(-i) \\
  &=& \prod_{i=0}^{k-1} (k-i) \prod_{i=k+1}^r (k-i) \\
  &=& \prod_{0\le i \le r \atop i\not=k} (k-i)
\end{eqnarray*}
Hence
\begin{eqnarray*}
  \sum_k {r \choose k} (-1)^{r-k} \sum_{i=0}^r b_i k^i &=& r!
  \sum_{i=0}^r b_i \sum_{k=0}^r \frac{k^i}{\prod_{0\le j \le r\atop
      j\not=k} (k-j)} \\
  &=& r!\,b_r
\end{eqnarray*}
The last equality is obtained by taking $x_i = i-1$ for $1\le i\le
r+1$ in exercise 1.2.3-33.

\newpar{30}  We have
\begin{eqnarray*}
  \sum_k {n+k \choose m+2k} {2k \choose k} \frac{(-1)^k}{k+1} &=&
  \sum_k {n+k \choose m+2k} {-k-1 \choose k} \frac{1}{k+1}\mbox{ using
    (17)} \\
  &=& \sum_k {n+k \choose n-m -k} {-k-1 \choose k} \frac{1}{k+1} \\
  &=& \sum_k {2n - m - (n-m -k) \choose n-m-k}{-k-1\choose k} \frac{1}{k+1}
\end{eqnarray*}

Using (26) with the substitutions: $r \la -1$, $t \la 1$, $n \la n-m$,
$s \la 2n - m$, we then deduce

\[ \sum_k {n+k \choose m+2k} {2k \choose k} \frac{(-1)^k}{k+1} = {n-1
  \choose m-1}.\]

\newpar{31}  We have
\begin{eqnarray*}
  && \sum_k {m-r+s \choose k} {n+r-s \choose n-k} {r+k \choose
    m+n}\\
  &=& \sum_k \sum_j {m-r+s \choose k} {n+r-s \choose n-k}{r
    \choose m+n -j}{k \choose j} \\
  &=& \sum_k \sum_j {m-r+s \choose
    j}{m-r+s-j \choose k-j} {n+r-s \choose n-k} {r \choose m+n -j} \\
  &=& \sum_k \sum_j {m-r+s \choose j} {m-r+s -j \choose m-(k+r-s)}{n+r-s
    \choose k+r-s} {r \choose m+n - j} \\
  &=& \sum_k \sum_j {m-r+s \choose j} {m-r+s - j \choose m-k} {n+r-s
    \choose k} {r \choose m+n -j } \\
  &=& \sum_j {m-r+s \choose j} {r \choose m+n -j} \sum_k {m-r+s -j
    \choose m-k} {n+r-s \choose k} \\
  &=& \sum_j {m-r+s \choose j} {r \choose m+n - j} {m + n - j \choose
    m} \\
  &=& \sum_j {m-r + s \choose j} {r \choose m} {r -m \choose n - j} \\
  &=& {r \choose m} \sum_j {m-r+s \choose j} {r-m \choose n-j} \\
  &=& {r \choose m} {s \choose n}
\end{eqnarray*}

\newpar{32} We have
\begin{eqnarray*}
  \sum_k \stirlingone{n}{k} x^k &=& (-1)^n \sum_k
  (-1)^{n-k}\stirlingone{n}{k} (-x)^k \\
  &=& (-1)^n (-x)^{\underline{n}}, \mbox{ by definition} \\
  &=& x^{\overline{n}}
\end{eqnarray*}

\newpar{33} Let's show the equality by induction on $n$.  If $n=0$, we
have
\[ 1 = (x + y)^{\overline{0}} = \sum_k {0 \choose k} x^{\overline{k}}
y^{\overline{0-k}}.\]

Suppose we have the equality for $n$, then we have
\begin{eqnarray*}
  \sum_k {n+1 \choose k} x^{\overline{k}} y^{\overline{n+1-k}} &=&
  \sum_k \left( {n \choose k-1} + {n \choose k} \right)
  x^{\overline{k}} y^{\overline{n+1-k}} \\
  &=& \sum_k {n \choose k} x^{\overline{k+1}} y^{\overline{n-k}} +
  \sum_k {n \choose k} x^{\overline{k}} y^{\overline{n+1-k}} \\
  &=& \sum_k {n \choose k} x^{\overline{k}} y^{\overline{n-k}} ((x + k
  )+ (y + n - k)) \\
  &=& (x + y + n) \sum_k {n \choose k} x^{\overline{k}}
  y^{\overline{n-k}}\\
  &=& (x + y + n) (x+y)^{\overline{n}}, \mbox{ by induction} \\
  &=& (x + y)^{\overline{n+1}}
\end{eqnarray*}

\newpar{34}  Given that $a^{\overline{n}} = n!{a+n-1 \choose n}$, we
then have
\begin{eqnarray*}
  S &=& \sum_k {n \choose k}
  x(x-kz+1)^{\overline{k-1}}(y+kz)^{\overline{n-k}} \\ &=& \sum_k {n
    \choose k} x (k-1)!(n-k)! {x-k(z-1)-1 \choose k-1} {y + kz + n - k
    - 1 \choose n-k} \\
  &=& n!\sum_k \frac{x}{k} {x -k (z-1)-1 \choose k-1}{y+nz - 1 -
    (z-1)(n-k) \choose n-k} \\
  &=& n! \sum_k \frac{x}{x - k(z-1)} {x - k(z-1) \choose k}{y+nz-1 -
    (z-1)(n-k) \choose n-k}
\end{eqnarray*}
Using (26) with $r=x$, $s = y+nz-1$ and $t = z-1$, we then deduce that
\begin{eqnarray*}
  \sum_k {n\choose k}x(x-kz+1)^{\overline{k-1}}(y+kz)^{\overline{n-k}}
  &=& n! {x + y + n - 1 \choose n} \\
  &=& (x+y)^{\overline{n}}
\end{eqnarray*}

\newpar{35} We have
\begin{eqnarray*}
  x^{\underline{n}} &=& n! {x \choose n} \\
  &=& n! \left( {x-1 \choose n} + {x-1 \choose n-1} \right) \\
  &=& (x-1)^{\underline{n}} + n (x-1)^{\underline{n-1}} \\
  &=& x^{-1} \left( x^{\underline{n+1}} + n x^{\underline{n}}\right)\\
  &=& x^{-1} \left( \sum_k (-1)^{n+1-k} \stirlingone{n+1}{k} x^k + n
  \sum_k (-1)^{n-k}\stirlingone{n}{k}x^k\right) \\
  &=& \sum_k (-1)^{n-k} \left( -\stirlingone{n+1}{k} + n
  \stirlingone{n}{k}\right) x^{k-1}
\end{eqnarray*}

From (44), we then deduce that
\[ - \stirlingone{n}{k-1} = - \stirlingone{n+1}{k} + n
\stirlingone{n}{k}.\]

Hence
\[ \stirlingone{n+1}{k} = n \stirlingone{n}{k} +
\stirlingone{n}{k-1}.\]

We have
\begin{eqnarray*}
  x^{n+1} &=& x\ x^n \\
  &=& x \sum_k \stirlingtwo{n}{k} x^{\underline{k}} \\
  &=& x \sum_k \stirlingtwo{n}{k} k! {x \choose k} \\
  &=& x \sum_k \stirlingtwo{n}{k} k! \left( {x-1 \choose k} + {x-1
    \choose k-1} \right) \\
  &=& x \sum_k \stirlingtwo{n}{k} \left( (x-1)^{\underline{k}} + k
  (x-1)^{\underline{k-1}} \right) \\
  &=& \sum_k \stirlingtwo{n}{k} \left( x^{\underline{k+1}} + k
  x^{\underline{k}} \right) \\
  &=& \sum_k \left(k \stirlingtwo{n}{k} + \stirlingtwo{n}{k-1}\right)
  x^{\underline{k}}
\end{eqnarray*}

Hence
\[ \sum_k \stirlingtwo{n+1}{k} x^{\underline{k}} = \sum_k \left(k
\stirlingtwo{n}{k} + \stirlingtwo{n}{k-1}\right) x^{\underline{k}}.\]

Taking successively $x = 0, 1, \ldots$ and dividing the equation by
$x, x-1, \ldots$ we deduce easily that
\[ \stirlingtwo{n+1}{k} = k \stirlingtwo{n}{k} +
\stirlingtwo{n}{k-1}.\]

\newpar{36} We have
\[ \sum_k {n \choose k} = \sum_k {n \choose k} 1^k 1^{n-k} = (1 + 1)^n
= 2^n.\]
And
\[ \sum_k {n \choose k} (-1)^k = \sum_k {n \choose k} (-1)^k 1^{n-1} =
(1-1)^n = \delta_{0,n}.\]

\newpar{37} We have
\[ 2^n + \delta_{0,n} = \sum_k {n \choose k} + \sum_k {n \choose k}
(-1)^k = 2 \sum_{2k} {n \choose 2k}.\]

Hence
\[ \sum_{2k} {n \choose 2k} = \frac{2^n + \delta_{0,n}}{2}.\]

\newpar{38} Note $\omega = e^{2\pi i/m}$.  We have
\begin{eqnarray*}
  \sum_{l\ge 0} {n \choose lm + k} &=& \sum_l {n \choose l} [ l \equiv
    k \pmod m ] \\
  &=& \sum_l {n \choose l} \frac{1}{m} \sum_{0 \le j < m}
  \omega^{j(l-k)} \\
  &=& \frac{1}{m} \sum_{0 \le j < m} \omega^{-jk} \sum_l {n \choose l}
  \omega^{jl} \\
  &=& \frac{1}{m} \sum_{0\le j < m} \omega^{-jk} (1+\omega^j)^n \\
  &=& \frac{1}{m} \sum_{0\le j <m} \omega^{j(n/2 - k)}
  \left(\omega^{j/2} + \omega^{-j/2}\right)^n \\
  &=& \frac{1}{m} \sum_{0\le j <m} \omega^{j(n/2 - k)} \left(2
  \cos\left(j\frac{\pi}{m}\right)\right)^n \\
  &=& \Re\left(\frac{1}{m} \sum_{0\le j <m} \omega^{j(n/2 - k)} \left(2
  \cos\left(j\frac{\pi}{m}\right)\right)^n \right)
\end{eqnarray*}
The last equality is true since the left-hand side of the equality is
a real number.  Hence
\[ \sum_{l\ge 0} {n \choose lm + k} = \frac{1}{m} \sum_{0\le j < m}
\left( 2 \cos\left(\frac{j\pi}{m} \right) \right)^n
\cos\left(\frac{j(n-2k)\pi}{m}\right).\]

\newpar{39} From \textbf{32}, we have
\[ \sum_k \stirlingone{n}{k} = 1^{\overline{n}} = n!.\]
And by definition
\[ \sum_k \stirlingone{n}{k}(-1)^k = (-1)^n 1^{\underline{n}} = [ n = 0].\]

\newpar{40} \subpar{a} By doing the variable substitution $t \la 1-t$,
we deduce that $B(x, y) = B(y, x)$.  Hence
\begin{eqnarray*}
  B(1, x) &=& B(x, 1) \\
  &=& \int_0^1 t^{x-1} dt \\
  &=& \left[ \frac{t^x}{x} \right]_0^1 \\
  &=& \frac{1}{x}
\end{eqnarray*}

\subpar{b} We have
\begin{eqnarray*}
  B(x+1, y) + B(x, y+1) &=& \int_0^1 t^{x-1}(1-t)^{y-1}(t + 1 - t)
  dt\\
  &=& B(x, y)
\end{eqnarray*}

\subpar{c} By using an integration by parts, we have
\begin{eqnarray*}
  B(x, y+1) &=& \left[ \frac{t^x}{x}(1-t)^y \right]_0^1 +
  \frac{y}{x}\int_0^1 t^x (1-t)^{y-1} dt \\
  &=& \frac{y}{x} B(x+1, y)
\end{eqnarray*}

And from (b), we deduce
\[ B(x, y) = \frac{x+y}{y} B(x, y+1).\]

\newpar{41} \subpar{a} From \textbf{40} (c), for each integer $k$ we have
\[ \frac{B(x, y + k + 1)}{B(x, y+k)} = \frac{y + k}{x+y+k}.\]
Hence
\begin{eqnarray*}
  \frac{B(x, y+m+1)}{B(x, y)} &=& \prod_{k=0}^m \frac{B(x,
    y+k+1)}{B(x, y+k)} \\
  &=& \prod_{k=0}^m \frac{y+k}{x+y+k} \\
  &=& \frac{\Gamma_m(x+y)}{\Gamma_m(y)m^x}
\end{eqnarray*}

\subpar{b} Note $n = \lfloor y\rfloor$.  We have
\[ n + m + 1 \le y + m + 1 < n+m+2.\]
Hence
\[ B(x, n+m+1) \ge B(x, y+m+1) \ge B(x, n+m+2).\]
Thus
\[ \frac{\Gamma_{n+m}(x)}{(n+m)^x} \ge B(x, y+m+1) \ge
\frac{\Gamma_{n+m+1}(x)}{(n+m+1)^x}.\]
And finally
\[ \frac{\Gamma_{n+m}(x)}{(1+n/m)^x} \ge m^x B(x, y+m+1) \ge
\frac{\Gamma_{n+m+1}(x)}{(1+(n+1)/m)^x}.\]

We then deduce that
\[ \lim_{m \to +\infty} m^x B(x, y+m+1) = \Gamma(x).\]

From (a), we then deduce that
\[ B(x, y) = \frac{\Gamma(x) \Gamma(y)}{\Gamma(x+y)}.\]

\newpar{42}  From \textbf{41} (b), we have
\begin{eqnarray*}
  B(k+1, r+1-k) &=& \frac{k!\Gamma(r+1-k)}{\Gamma(r+2)} \\
  &=& \frac{k!}{(r+1)r(r-1) \ldots (r+1-k)} \\
  &=& \frac{1}{(r+1){r \choose k}}
\end{eqnarray*}
Hence
\[ {r \choose k} = \frac{1}{(r+1)B(k+1, r+1-k)}.\]

\newpar{43} We have
\[ B(1/2, 1/2) = \int_0^1 \frac{dt}{\sqrt{t(1-t)}}.\]

$\theta \mapsto \sin^2\theta$ is a homeomorphism from $[0, \pi/2]$ to
$[0, 1]$. Thus by doing the variable substitution $t \la
\sin^2\theta$, we have
\begin{eqnarray*}
  B(1/2, 1/2) &=& \int_0^{\pi/2} \frac{2 \sin\theta \cos\theta
    d\theta}{\sqrt{\sin^2\theta \cos^2\theta}} \\
  &=& \pi
\end{eqnarray*}

\newpar{44} We have from \textbf{41} (b) and given that $\Gamma(x+1) =
x\Gamma(x)$
\begin{eqnarray*}
  \frac{1}{{r \choose 1/2} {2r \choose r}} &=&
  (r+1)B(3/2,r+1/2)(2r+1)B(r+1, r+1) \\
  &=& (r+1)(2r+1)
  \frac{\Gamma(3/2)\Gamma(r+1/2)}{\Gamma(r+2)}\times\frac{\Gamma(r+1)\Gamma(r+1)}{\Gamma(2r+2)}
  \\ &=& \frac{\Gamma(3/2)\Gamma(r+1/2) \Gamma(r+1)}{\Gamma(2r+1)} \\
\end{eqnarray*}

We have $\Gamma(3/2) = 1/2 \times \Gamma(1/2) = \sqrt{\pi}/2$.  And
\begin{eqnarray*}
  \frac{\Gamma_m(r+1/2)\Gamma_m(r+1)}{\Gamma_{2m}(2r+1)} &=&
  \frac{\frac{m^{r+1/2}m!}{(r+1/2)(r+3/2)\cdots(r+(m+1)/2} \times
    \frac{m^{r+1}m!}{(r+1)(r+2)\cdots(r+m+1)}}
       {\frac{(2m)^{2r+1}(2m)!}{(2r+1)(2r+2)\cdots(2r+2m+1)}} \\
       &=& \frac{2^{2m-2r}\sqrt{m}}{r+m+1} \times
       \frac{(m!)^2}{(2m)!}\\
       &{\simeq \atop m \to +\infty}& \frac{2^{2m-2r}}{\sqrt{m}}\times
       \frac{\left(\frac{m}{e}\right)^{2m} 2 \pi
         m}{\left(\frac{2m}{e}\right)^{2m} \sqrt{4\pi m}} \\
       &{\simeq \atop m \to +\infty}& \frac{\sqrt{\pi}}{2^{2r}}
\end{eqnarray*}

Thus finally
\begin{eqnarray*}
  {r \choose 1/2}{2r \choose r} &=&
  \frac{2\Gamma(2r+1)}{\sqrt{\pi}\ \Gamma(r+1/2)\Gamma(r+1)} \\
  &=& \lim_{m \to +\infty}
  \frac{2\Gamma_{2m}(2r+1)}{\sqrt{\pi}\ \Gamma_m(r+1/2)\Gamma_m(r+1)}
\\
&=& \frac{2^{2r+1}}{\pi}
\end{eqnarray*}

\newpar{45} We have
\begin{eqnarray*}
  \frac{{r \choose k}}{r^k} &=& \frac{1}{(r+1)r^kB(k+1, r+1-k)} \\
  &=& \frac{\Gamma(r+2)}{(r+1)r^kk!\Gamma(r+1-k)} \\
  &=& \frac{r(r-1)\cdots(r+1-k)}{r^kk!} \\
  &{\to \atop r \to +\infty}& \frac{1}{k!}
\end{eqnarray*}

\newpar{46} We have
\begin{eqnarray*}
  {x+y \choose y} &=& \frac{1}{(x+y+1)B(y+1, x+1)} \\
  &=& \frac{\Gamma(x+y+1)}{\Gamma(x+1)\Gamma(y+1)} \\
  &{\simeq \atop x,y \to +\infty}&
  \frac{\left(\frac{x+y}{e}\right)^{x+y}\sqrt{(x+y)}}{\left(\frac{x}{e}\right)^{x}\left(\frac{y}{e}\right)^{y}
    \sqrt{2\pi x y}} \\
  &=& \sqrt{\frac{1}{2\pi}\left(\frac{1}{x}+\frac{1}{y}\right)}
  \left(1 + \frac{y}{x}\right)^x   \left(1 + \frac{x}{y}\right)^y
\end{eqnarray*}

Hence
\[ {2n \choose n} {\simeq \atop n \to +\infty} \frac{4^n}{\sqrt{\pi
    n}}.\]

\newpar{47}  For $k$ integer, we have
\begin{eqnarray*}
  {r \choose k} {r-1/2 \choose k} &=&
  \frac{\prod_{i=0}^{k-1}(r-i)}{k!} \times
  \frac{\prod_{i=0}^{k-1}(r-1/2-i)}{k!} \\
  &=& \frac{\prod_{i=0}^{k-1}(2r-2i)}{2^kk!} \times
  \frac{\prod_{i=0}^{k-1}(2r-2i-1)}{2^kk!} \\
  &=& \frac{\prod_{i=0}^{2k-1}(2r-i)}{4^k (k!)^2} \\
  &=& \frac{1}{4^k} \times \frac{\prod_{i=0}^{k-1}(2r-i)}{k!} \times
  \frac{\prod_{i=k}^{2k-1}(2r-i)}{k!} \\
  &=& \frac{1}{4^k} \times \frac{\prod_{i=0}^{k-1}(2r-i)}{k!} \times
  \frac{\prod_{i=0}^{k-1}(2r-k-i)}{k!} \\
  &=& \left. {2r \choose k}{2r-k \choose k} \right/4^k
\end{eqnarray*}

And using (20), we have
\[ \left. {2r \choose k}{2r-k \choose k} \right/4^k = \left. {2r
  \choose 2k}{2k \choose k} \right/ 4^k\]

And taking $r = -1/2$, we deduce
\[ (-1)^k {-1/2 \choose k} = \left. (-1)^k {-1-k \choose k}\right/4^k = \left.
   {2k \choose k}\right/4^k\]

\newpar{48}  We have the fraction decomposition
\[ \frac{1}{x(x+1)\ldots(x+n)} = \sum_{k=0}^n \frac{a_k}{x+k}.\]

By multiplying each member by $x+i$ and taking $x = -i$, we have
\begin{eqnarray*}
  a_i &=& \frac{1}{\prod_{0\le k\le n, k\not=i} (k-i)} \\
  &=& \frac{(-1)^i}{(n-i)!i!} \\
  &=& (-1)^i \frac{{n \choose i}}{n!}
\end{eqnarray*}

Hence
\[ \sum_{k\ge 0} {n \choose k}\frac{(-1)^k}{k+x} = \frac{1}{x {n+x
    \choose n}}.\]

\newpar{49} We have
\begin{eqnarray*}
  (1-x^2)^r (1-x)^{-r} &=& \left( \sum_{k\ge 0} {r \choose k}(-1)^k
  x^{2k}\right) \left( \sum_{k\ge 0} {-r \choose k} (-1)^k x^k\right)
\\
&=& \sum_{k \ge 0} \sum_{l \ge 0} {r \choose k}{-r \choose l}
(-1)^{k+l} x^{2k+l} \\
&=& \sum_{m \ge 0} \left( \sum_{k \ge 0} {r \choose k}{-r \choose
  m-2k}(-1)^{m-k} \right) x^m
\end{eqnarray*}

We then deduce that
\[  {r \choose m} =  \sum_{k \ge 0} {r \choose k}{-r \choose
  m-2k}(-1)^{m-k} .\]

\newpar{50} We have
\begin{eqnarray*}
  S &=&  \sum_k {n \choose k} x(x-kz)^{k-1}(-x + kz)^{n-k} \\
  &=& \sum_k {n
    \choose k} (-1)^{n-k} x (x - kz)^{n-1} \\
  &=& \sum_k {n \choose k} (-1)^{n-k} \sum_l {n-1 \choose l}
  x^{n-l}z^l k^l \\
  &=& \delta_{n,0}, \mbox{ from (34)}
\end{eqnarray*}

\newpar{51} We have
\begin{eqnarray*}
 S &=&  \sum_k {n \choose k} x (x-kz)^{k-1}(y+kz)^{n-k} \\
 &=&  \sum_k {n \choose k} x (x-kz)^{k-1}((x+y) - (x-kz))^{n-k} \\
 &=& \sum_k {n \choose k} x(x-kz)^{k-1} \sum_l {n-k \choose l} (x+y)^l
 (-1)^{n-k-l}(x-kz)^{n-k-l} \\
 &=& \sum_l x (x+y)^l (-1)^{n-l} \sum_k (-1)^k {n \choose k}{n - k
   \choose l}(x - kz)^{n-l-1} \\
 &=& \sum_l x (x+y)^l (-1)^{n-l} \sum_k (-1)^k {n \choose l}{n-l
   \choose k} (x-kz)^{n-l-1}, \mbox{ from (20)} \\
 &=& \sum_l {n \choose l}(x+y)^l \sum_k {n-l \choose k}(-1)^{n-l-k} x(x -
 kz)^{n-l-1} \\
 &=& \sum_l {n \choose l}(x+y)^l \delta_{n,l}, \mbox{ from \textbf{50}} \\
 &=& (x+y)^n
\end{eqnarray*}

\newpar{52} We have
\begin{eqnarray*}
  S &=& \sum_k {-1 \choose k}(-1-k)^{k-1}(1+k)^{-1-k} \\
  &=& \sum_{k\ge 0} - (1+k)^{-2}
\end{eqnarray*}

This series is convergent whereas the left-hand side of the equality
is not defined.

\newpar{53} \subpar{a} If $m=0$, we have
\begin{eqnarray*}
  \sum_{k=0}^0 {r \choose k} {s \choose n-k} (nr-(r+sk)) &=& {s
    \choose n}nr \\
  &=& (0+1)(n-0) {r \choose 0+1}{s \choose n-0}
\end{eqnarray*}

Suppose the equality is true for $m$.  Then we have by induction
\begin{eqnarray*}
 S_{m+1} &=&  \sum_{k=0}^{m+1} {r \choose k}{s \choose n-k} (nr -
 (r+s)k) \\
 &=& (m+1)(n-m){r\choose m+1}{s \choose n-m} + \\ && {r \choose m+1}{s \choose
   n -m -1}(nr-(r+s)(m+1)) \\
 &=& (m+1)(s-n+m+1) {r \choose m+1}{s \choose n-m-1} + \\
 && {r \choose m+1}{s \choose n-m-1}(nr-(r+s)(m+1)) \\
 &=& {r \choose m+1}{s \choose n-m-1} ((m+1)(-r -n + m + 1) + nr) \\
 &=& {r \choose m+1}{s \choose n-m-1}((m+1)(-r+m+1) + n(r-m-1)) \\
 &=& {r \choose m+1}{s \choose n-m-1}(r-m-1)(n-m-1) \\
 &=& (m+2)(n-m-1) {r \choose m+2} {s \choose n-m-1}
\end{eqnarray*}

\subpar{b} We have
\begin{eqnarray*}
  T_m &=& \sum_{k=0}^m {2k-1\choose k}{2n-2k\choose n-k}
  \frac{-1}{2k-1} \\
  &=& \sum_{k=0}^m \left( {1/2 \choose k} + \delta_{k0}
  \right)(-1)^k2^{2k-1} {2n - 2k \choose n-k} \\
  &=& \sum_{k=0}^m \left( {1/2 \choose k} + \delta_{k0}\right) (-1)^k
  2^{2k-1} {-1/2 \choose n-k} 2^{2(n-k)}(-1)^{n-k} \\
  &=& (-1)^n 2^{2n-1} \left( \sum_{k=0}^m {1/2 \choose k}{-1/2 \choose
    n-k} + {-1/2 \choose n} \right)
\end{eqnarray*}

Thus using (a) with $r=1/2$ and $s=-1/2$, we have
\begin{eqnarray*}
  T_m &=& (-1)^n2^{2n-1}\left( \frac{2(m+1)(n-m)}{n}{1/2 \choose m+1}{-1/2
    \choose n-m} + {-1/2 \choose n}\right) \\
  &=& 2^{-2} \frac{(m+1)(n-m)}{n(2m+1)} {2(m+1)
    \choose m+1} {2(n-m) \choose n-m} + \frac{1}{2} {2n \choose n} \\
  &=& \frac{n-m}{2n} {2m \choose m}{2(n-m) \choose n-m} + \frac{1}{2}
  {2n \choose n}
\end{eqnarray*}

\newpar{54}  We have
\[ (1+x)^r = \sum_{k=0}^r {r \choose k} x^k = \sum_k {r \choose k}
x^k.\]

Hence
\begin{eqnarray*}
  \sum_r {n \choose r} (-1)^{n-r}(1+x)^r &=& \sum_r \sum_k {n \choose
    r}{r \choose k} (-1)^{n-r} x^k \\
  &=& \sum_k x^k \sum_r {n \choose r}{r \choose k}(-1)^{n-r} \\
  &=& \sum_k x^k \delta_{nk}, \mbox{ from (33)} \\
  &=& x^n
\end{eqnarray*}

Hence the line number $n$ and column number $r$ of the inverse of the
Pascal triangle matrix is ${n \choose r} (-1)^{n-r}$.

\newpar{55} From (45) and (44), it's easy to deduce that the inverse
of the matrix of Stirling's numbers of the second kind is
$\left((-1)^{n-k}\stirlingone{n}{k}\right)_{n, k}$.

From \textbf{32}, we have $x^{\overline{n}} = \sum_k
\stirlingone{n}{k} x^k$. And
\begin{eqnarray*}
  x^n &=& (-1)^n (-x)^n \\
  &=& (-1)^n \sum_k \stirlingtwo{n}{k} (-x)^{\underline{k}} \\
  &=& (-1)^n \sum_k \stirlingtwo{n}{k} (-1)^k x^{\overline{k}} \\
  &=& \sum_k \stirlingtwo{n}{k}(-1)^{n-k} x^{\overline{k}}
\end{eqnarray*}

We then deduce that the inverse of the matrix of Stirling's numbers of
the first kind is $\left((-1)^{n-k}\stirlingtwo{n}{k}\right)_{n,k}$.

\newpar{56} Let's note for $a < b < c$ integers
\[ C(a, b, c) = {a \choose 1} + {b \choose 2} + {c \choose 3}.\]

If $a < b-1$, then we have
\[ C(a+1, b, c) = C(a, b, c) + 1.\]

If we have $a = b-1$ and $b < c-1$, then we have
\begin{eqnarray*}
  C(a, b, c) &=& b-1 + \frac{b(b-1)}{2} + {c \choose 3} \\
  &=& \frac{b(b+1)}{2} - 1 + {c \choose 3} \\
  &=& C(0, b+1, c) - 1
\end{eqnarray*}

And finally if $a = b-1$ and $b = c-1$, we have
\begin{eqnarray*}
  C(a, b, c) &=& c-2 + \frac{(c-1)(c-2)}{2} + \frac{c(c-1)(c-2)}{6} \\
  &=& c-2 + \frac{(c-1)(c-2)}{2} + \frac{(c+1)c(c-1)}{6} -
  \frac{c(c-1)}{2} \\
  &=& {c+1 \choose 3} - 1 \\
  &=& C(0, 1, c+1) - 1
\end{eqnarray*}

In short we have
\[
C(a, b, c) = \left\{
\begin{array}{l}
  C(a+1, b, c)-1\mbox{ if $a<b-1$} \\
  C(0, b+1, c)-1\mbox{ if $a=b-1$ and $b<c-1$} \\
  C(0, 1, c+1)-1\mbox{ if $a=b-1$ and $b=c-1$}
\end{array}
\right.
\]

And since $C(0, 1, 2) = 0$, we've got a recursive algorithm to compute
the values $a, b, c$ such that $C(a, b, c) = n$ for all $n$ integers.
``56.lisp'' is such an implementation.  We have
\begin{eqnarray*}
  C(0, 1, 2) &=& 0 \\
  C(0, 1, 3) &=& 1 \\
  C(0, 2, 3) &=& 2 \\
  C(1, 2, 3) &=& 3 \\
  C(0, 1, 4) &=& 4 \\
  C(0, 2, 4) &=& 5 \\
  C(1, 2, 4) &=& 6 \\
  C(0, 3, 4) &=& 7 \\
  C(1, 3, 4) &=& 8 \\
  C(2, 3, 4) &=& 9 \\
  C(0, 1, 5) &=& 10 \\
  C(0, 2, 5) &=& 11 \\
  C(1, 2, 5) &=& 12 \\
  C(0, 3, 5) &=& 13 \\
  C(1, 3, 5) &=& 14 \\
  C(2, 3, 5) &=& 15 \\
  C(0, 4, 5) &=& 16 \\
  C(1, 4, 5) &=& 17 \\
  C(2, 4, 5) &=& 18 \\
  C(3, 4, 5) &=& 19 \\
  C(0, 1, 6) &=& 20
\end{eqnarray*}

\newpar{57} We have
\begin{eqnarray*}
  \sum_{n \ge 0} c_n \ln n! &=& \sum_{n \ge 0} c_n \sum_{k \ge 0}
  a_{k+1} (k+1)! {n \choose k+1} \\
  &=& \sum_{k\ge 0} a_{k+1}(k+1)!\sum_{n \ge 0} c_n {n \choose k+1}
\end{eqnarray*}
If we choose $c_n = {m \choose n} (-1)^{m-n}$, then we have
\begin{eqnarray*}
  \sum_{n\ge 0} c_n \ln n! &=& \sum_{k\ge 0}a_{k+1}(k+1)! \sum_{n\ge
    0} {m \choose n}{n \choose k+1} (-1)^{m-n} \\
  &=& \sum_{k\ge 0} a_{k+1}(k+1)! \delta_{m, k+1}, \mbox{ from (33)}\\
  &=& m! a_m
\end{eqnarray*}

Hence
\begin{eqnarray*}
  a_m &=& \frac{1}{m!} \sum_{n\ge 0} {m \choose n} (-1)^{m-n} \ln n!\\
  &=& \frac{(-1)^m}{m!} \sum_{n\ge 0} {m \choose n} (-1)^n
  \sum_{k=1}^n \ln k \\
  &=& \frac{(-1)^m}{m!} \sum_{k\ge 1} \ln k \sum_{n \ge k} {m \choose
    n}(-1)^n \\
  &=& \frac{(-1)^m}{m!} \sum_{k\ge 1} \ln k \left( \sum_{n \ge 0} {m
    \choose n}(-1)^n - \sum_{0 \le n \le k-1}{m \choose
    n}(-1)^n\right)\\
  &=& \frac{(-1)^m}{m!} \sum_{k\ge 1} \ln k \left( (1-1)^m + (-1)^k
      {m-1 \choose k-1}\right), \mbox{ from (18)}
\end{eqnarray*}

And finally
\[ a_m = \frac{(-1)^m}{m!} \sum_{k\ge 1} (-1)^k {m-1 \choose k-1}\ln
k.\]

\newpar{58} Let's show the property by induction on $n$.  If $n=0$, we
have
\[  1 = \prod_{k=0}^{-1} (1+q^k x) = \sum_k {0 \choose k} q^{k(k-1)/2}
x^k.\]

Suppose we have the property for $n$ by induction we have
\begin{eqnarray*}
  \prod_{k=0}^n(1+q^k x) &=& (1+q^n x) \sum_k {n \choose k}_q
  q^{k(k-1)/2}x^k \\
  &=& \sum_k {n \choose k}_q q^{k(k-1)/2}x^k + \sum_k {n \choose k-1}_q
  q^{(k-1)(k-2)/2 + n} x^k \\
  &=& \sum_k {n \choose k}_q \left(1 + \frac{1-q^k}{1-q^{n-k+1}}
  q^{n-k+1}\right) q^{k(k-1)/2}x^k \\
  &=& \sum_k {n+1 \choose k}_q q^{k(k-1)/2} x^k
\end{eqnarray*}

\newpar{59} Let's show by induction on $n \ge 0$ that we have
\[ A_{nk} = k {n+1 \choose k+1} + {n \choose k}.\]

We have
\[ A_{0k} = \delta_{0k} = k {1 \choose k+1} + {0 \choose k}.\]

Suppose the property is true for $n$, then we have
\begin{eqnarray*}
  A_{(n-1)k} + A_{(n-1)(k-1)} + {n \choose k} &=& k {n \choose k+1} +
  {n-1 \choose k} + (k-1) {n \choose k} + \\ && {n-1 \choose k-1} + {n
    \choose k} \\
  &=& k \left( {n \choose k+1} + {n \choose k} \right) + {n-1 \choose
    k} + \\ && {n-1 \choose k-1} \\
  &=& k {n+1 \choose k+1} + {n \choose k}
\end{eqnarray*}

\newpar{60} Let's note $N_R(n, k)$ the combination with repetitions of
$k$ elements from a set of $n$ elements. By definition, we have
\[ {n \choose k} = \sum_{0 < N_1 < N_2 < \ldots < N_k < n+1} 1.\]

And by analogy,
\[  N_R(n, k) = \sum_{0 < N_1 \le N_2 \le \ldots \le N_k < n+1} 1.\]

Given $a\le b$ if and only if $a < b+1$, a quick induction gives us
\begin{eqnarray*}
  N_R(n, k) &=& \sum_{0 < N_1 < N_2+1 < \ldots < N_k + k-1 < n+k} 1 \\
  &=& \sum_{0 < N_1\prime < N_2\prime < \ldots < N_k\prime < n+k} 1 \\
  &=& {n+k-1 \choose k}
\end{eqnarray*}

\newpar{62} Note $a_{n,m} = \sum_k
\stirlingone{n+1}{k+1}\stirlingtwo{k}{m}(-1)^{k-m}$.  We have from
$(46)$ and $(47)$
\begin{eqnarray*}
  a_{n,m} &=&
  \sum_k \left( n \stirlingone{n}{k+1} + \stirlingone{n}{k}\right)
  \stirlingtwo{k}{m} (-1)^{k-m} \\
  &=& n \sum_k \stirlingone{n}{k+1} \stirlingtwo{k}{m} (-1)^{k-m} +
  (-1)^{m-n} \delta_{m,n} \\
  &=& n a_{n-1,m} + (-1)^{m-n}\delta_{m,n} \\
  &=& n a_{n-1,m} + \delta_{m,n}
\end{eqnarray*}

We have
\[ a_{n,m} = \sum_{m \le k \le n} \stirlingone{n+1}{k+1}
\stirlingtwo{k}{m}(-1)^{k-m}.\]

Thus if $n<m$, $a_{n,m} = 0$ and $a_{m,m} = 1$.  And if $n > m$, we
have $a_{n,m} = n a_{n-1,m}$.  So finally
\[ a_{n,m} = [ n \ge m ] \frac{n!}{m!}.\]

\newpar{61} Note $S(m,n,l) = \sum_k(-1)^k {l+m \choose l+k}{m+n
  \choose m+k}{n+l \choose n+k}$.  Using \textbf{exercise 31}, we have
\begin{eqnarray*}
 S(m,n,l) &=& \sum_k(-1)^k {l+m \choose l+k}{m+n \choose m+k}{n+l
   \choose l-k} \\
  &=& \sum_k (-1)^k {l+m \choose l+k} \sum_i {k+l \choose i} {m-k
   \choose l-k-i}{m+n+i \choose m+l} \\
 &=& \sum_k \sum_i \frac{(m+n+i)!}{(n+i-l)! i! (m-l+i)!} \times
 \frac{(-1)^k}{(k+l-i)!(l-k-i)!} \\
 &=& \sum_i \frac{(m+n+i)!}{(n+i-l)!i!(m-l+i)!(2l-2i)!} \sum_k (-1)^k
      {2(l-i) \choose k+l-i} \\
  &=& \sum_i \frac{(m+n+i)!}{(n+i-l)!i!(m-l+i)!(2l-2i)!} \times
      (-1)^{l+i} \delta_{l,i} \\
      &=& \frac{(m+n+l)!}{m!n!l!}
\end{eqnarray*}

\newpar{63} For $l, m,$ and $n$ integers and $n \ge 0$, let's note
\[ S(s, r, m, n, l) = \sum_{j,k}(-1)^{j+k}{j+k \choose k+l}{r \choose
  j}{n \choose k}{s + n - j -k \choose m-j}.\]

Our aim is to show that
\[ S(s, r, m, n, l) = (-1)^l {n+r \choose n+l} {s-r \choose
  m-n-l}.(*)\]

Notice that the right hand side of (*) is a polynomial of degree $m$ in
$r$.  And it's easy to see that if $j > m$, the last term in the sum
in $S(s, r, m, n, l)$ would be equal to zero.  Hence we can always
assume that $j \le m$.  Thus $S(s, r, m, n, l)$ is a polynomial of
degree at most equal to $m$ in $r$.  Thus to prove (*), we only need
to prove the equality on $m+1$ different values of $r$ when all the
other parameters are fixed.

Let's consider first the case where $s = m$ and let's show that
\[ S(m, r, m, n, l) = (-1)^l {n+r \choose n+l}{m-r \choose
  m-n-l}. (**)\]

Suppose that $m < n+l$.  The right-hand side of (**) is thus equal to
zero. And we have
\[ S(m, r, m, n, l) = \sum_k (-1)^k {n \choose k} \sum_j (-1)^j {j+k
  \choose j-l} {r \choose j}{m+n-j-k\choose m-j}.\]

The degree of $\sum_j (-1)^j {j+k\choose j-l}{r \choose
  j}{m+n-j-k\choose m-j}$ as a polynomial in $k$ is less than or equal
to $(j-l) + (m-j) = m - l < n$.  Hence from (34), we deduce that
\[ S(m, r, m, n, l) = 0.\]

From now on, let's assume that $m \ge n+l$.  Suppose that $0 \le n+l <
r \le m$.  Again the right-hand side of (**) is equal to zero.  And we
have
\[ S(m, r, m, n, l) = \sum_j (-1)^j {r \choose j} \sum_k (-1)^k {j+k
  \choose k+l}{n \choose k} {m+n-j-k \choose n-k}.\]

The degree of $\sum_k (-1)^k {j+k \choose k+l}{n \choose k}{m+n+j-k
  \choose n-k}$ as a polynomial in $j$ is less than or equal to $(k+l)
+ (n-k) = l+n < r$.  Hence again from (33), we have
\[ S(m, r, m, n, l) = 0.\]

We have thus far proved the equality (**) on $m-n-l$ different values
of $r$.  Now, suppose that $0 \le -r-1+l < n+l$.  Note that we thus
have $r \le l-1 < n+l$ so these values of $r$ are different from the
previous one.  Notice also that the right-hand side of (**) is equal
to zero.  We have
\[ {r \choose j} = (-1)^j {j-r-1 \choose j} = (-1)^j \sum_p {j-l
  \choose j-p}{-r-1+l\choose p}.\]

Hence
\begin{eqnarray*}
  S(m,r,m,n,l) &=& \sum_{j,k,p}(-1)^k {j+k \choose k+l}{j-l\choose
    j-p}{-r-1+l\choose p}{n \choose k}\\ &&{m+n-j-k\choose m-j} \\
  &=& \sum_{j,k,p}(-1)^k {j+k \choose j-l}{j-l\choose
    j-p}{-r-1+l\choose p}{n \choose k}\\ &&{m+n-j-k\choose m-j} \\
  &=& \sum_{j,k,p}(-1)^k {j+k \choose j-p}{k+p \choose k+l}{-r-1+l
    \choose p} {n \choose k}\\ && {m+n-j-k \choose m-j} \\
  &=& \sum_{j,k,p}(-1)^{k+p+m} {k+p \choose k+l}{-r-1+l \choose p}{n
    \choose k} \\ && {-p -k -1 \choose j - p}{k-n-1\choose m-j} \\
  &=& \sum_{k,p}(-1)^{k+p+m} {k+p \choose k+l} {-r-1+l \choose p}{n
    \choose k} {-n-p-2 \choose m-p} \\
  &=& \sum_k (-1)^k {n \choose k} \sum_p (-1)^{p+m} {k+p \choose
    p-l}{-r-1+l \choose p}\\ && {-n-p-2 \choose m-p}
\end{eqnarray*}
The degree of $\sum_p (-1)^{p+m} {k+p \choose p-l}{-r-1+l\choose
  p}{-n-p-2\choose m-p}$ as a polynomial in $k$ is less than or equal
$p-l \le (-r-1+l) -l < n$.  Using (33) once again, we deduce that
$S(m,r,m,n,l) = 0$.

We have thus established the equality (**) on $m$ different values of
$r$.  Finally let's show that we also have (**) for $r = l$.   We have
\begin{eqnarray*}
  S(m,l,m,n,l) &=& \sum_{j,k}(-1)^{j+k}{j+k \choose l+k}{l \choose
    j}{n \choose k}{m+n-j-k\choose m-j} \\
  &=& \sum_{j,k} (-1)^{j+k} \delta_{j,l} {n \choose k}{m+n-j-k \choose
    m-j} \\
  &=& \sum_k (-1)^{l+k}{n \choose k}{m+n-l-k \choose m-l} \\
  &=& \sum_k (-1)^{l+k}{n \choose k}{m+n-l-k \choose n-k} \\
  &=& \sum_k (-1)^{n+l}{n \choose k}{l-m-1 \choose n-k} \\
  &=& (-1)^{n+l} {n+l-m-1 \choose n} \\
  &=& (-1)^l {m-l \choose n}
\end{eqnarray*}

Thus, we have established the equality (**) on $m+1$ different values
of $r$ where all the polynomials in the equality are of degree less
than $m$.  Hence we have the equality for all values of $r$.

Let's now prove (*) by induction on $s \ge m$.  From (**), the
equality is true when $s = m$.  Suppose that we have (*).  We have
\[ {s+1 + n-j-k\choose m-j} = {s+n-j-k\choose m-j} + {s+n-j-k\choose
  m-1-j}.\]

Hence by induction,
\begin{eqnarray*}
  S(s+1, r, m, n, l) &=& S(s, r, m, n, l) + S(s, r, m-1, n, l) \\
  &=& (-1)^l {n+r\choose n+l}\left({s-r\choose m-n-l} + {s-r\choose
    m-n-l-1}\right) \\
  &=& (-1)^l {n+r \choose n+l}{s+1-r\choose m-n-l}
\end{eqnarray*}

Since the two terms in (*) are polynomials in $s$ and they're equals
on an infinite number of values of $s$, we then deduce that the
equality holds for all values of $s$.

\newpar{64} Let's note $P(n, m)$ the number of ways to partition a set
of $n$ elements into $m$ nonempty disjoint subsets.  Thus, we have
\[ P(1, n) = \delta_{1,n} \mbox{ and } P(n, 0) = \delta_{n, 0}.\]

Now suppose $m > 0$ and let's compute $P(n+1, m)$.  There's two cases
that we need to consider.  Firstly, the $(n+1)^{th}$ element is the
only element of its subset.  Thus it's like the remaining $n$ elements
are partitioned into the remaining $m-1$ nonempty disjoint subsets.
Hence there is $P(n, m-1)$ of them.

Secondly the $(n+1)^{th}$ element isn't the only element of its
subset.  This case is the same as if we partitioned the first $n$
elements into $m$ nonempty disjoint subsets, then we place the
$(n+1)^{th}$ element into one of them.  So there is $m P(n, m)$
partitions covered by this case.  Hence,
\[ P(n+1, m) = m P(n, m) + P(n, m-1).\]

And we can prove that $P(n, m) = \stirlingtwo{n}{m}$ by induction.

\newpar{66} We have
\begin{eqnarray*}
  {x \choose n+1} &=& (-1)^{n+1} {n-x \choose n+1} \\
  &=& (-1)^{n+1} \sum_{k\ge 0} {n-z-1 \choose n+1-k}{z-x+1 \choose
    k}\\ &=&
  (-1)^{n+1} \sum_{k\ge -1} {n-z-1 \choose n-k}{z-x+1 \choose k+1} \\
  &=& (-1)^{n+1} {n-z-1 \choose n+1} + \\&&\sum_{k\ge 0} (-1)^{n-k}{n-z-1
    \choose n-k}(-1)^{k+1}{z-x+1 \choose k+1} \\
  &=& {z+1 \choose n+1} + \sum_{k\ge 0} {z-k \choose n-k}{k+x-z-1
    \choose k+1}.
\end{eqnarray*}

We then deduce
\begin{eqnarray*}
  {x \choose n+1} - {y \choose n+1} = \sum_{k\ge 0} {z-k \choose n-k}
  t_k (*)
\end{eqnarray*}
where $t_k = {k+x-z-1 \choose k+1} - {k+y-z-1 \choose k+1}$.

If $z > n$, then $0 < {z \choose n+1} \le {y \choose n+1}$ since $y
\ge z$.  Otherwise $n-1 \le z \le n$, hence
\[ {z \choose n+1} \le 0 \le {y \choose n+1}.\]

Thus,
\[ {z+1 \choose n+1} = {z \choose n+1} + {z \choose n} \le {y \choose
  n+1} + {z \choose n} = {x \choose n}.\]

We then deduce that $z+1 \le x$ since $z \ge n-1$.

So we can conclude that ${k+x-z-1 \choose k+1} \ge 0$.  If we have $y
\ge z+1$, then we conclude easily that $t_k \ge 0$ since $x \ge y$.
Otherwise, $z \le y < z+1$.  Hence ${k + y - z - 1\choose k+1} < 0$
and thus $t_k > 0$.

Since $z \ge n-1$, we then conclude that all the terms in the
right-hand side of (*) are positive.

We then have
\begin{eqnarray*}
  {x \choose n} - {y \choose n} - {z \choose n-1} &=& \sum_{k\ge 0}
  {z-k \choose n-1-k}(t_k-\delta_{0k}) \\
  &=& \sum_{k\ge0} \frac{n-k}{z-n+1} {z-k \choose n-k}
  (t_k-\delta_{0k}) \\
  &=& \frac{n}{z-n+1}{z \choose n}(t_0 - 1) + \\
  && \sum_{k \ge 1} \frac{n-k}{z-n+1}{z-k \choose n-k}t_k \\
  &\le& \frac{n}{z-n+1}{z \choose n}(t_0 - 1) + \\
  && \frac{n-1}{z-n+1}\sum_{k\ge 1} {z-k \choose n-k}t_k \\
  &=& \frac{n}{z-n+1}{z \choose n}(t_0 - 1) + \\
  && \frac{n-1}{z-n+1} \left( {x \choose n+1} - {y\choose n+1} - {z
    \choose n}t_0\right) \\
  &=& \frac{n}{z-n+1}{z \choose n}(t_0 - 1) + \\
  && \frac{n-1}{z-n+1} \left( {z \choose n} - {z \choose n}t_0\right) \\
  &=& \frac{x-y-1}{z-n+1} {z \choose n} \\
  &\le& 0
\end{eqnarray*}
\end{document}
