\documentclass[a4paper,12pt]{article}
\newcommand{\newpar}[1]{\bigskip \noindent \textbf{#1.}}
\newcommand{\subpar}[1]{\medskip \noindent (#1)}
\newcommand{\la}{\leftarrow}
\newcommand{\ra}{\rightarrow}
\begin{document}

\newpar{1} $52!$.

\newpar{2} $p_{n(n-1)} = n(n-1) \ldots 2 = p_{nn}$.  We have the
equality since if we've already choose $n-1$ objects, the $n^{th}$
object always come last.

\newpar{3} Using the first method we obtain 5 3 1 2 4, 3 5 1 2 4, 3 1
5 2 4, 3 1 2 5 4, 3 1 2 4 5.

Using the second one we have: 4 2 3 5 1, 4 1 3 5 2, 4 1 2 5 3, 3 1 2 5
4, 3 1 2 4 5.

\newpar{4} The number of decimal digits of a positive integer $n$ is
$\lfloor \log_{10} n\rfloor + 1$.  Thus, $1000!$ has $2568$ decimal
digits.
We have
\[ 0.604 < \log_{10} \left(\frac{1000!}{10^{2567}}\right) < 0.605.\]
Hence
\[ 4.01 < \frac{1000!}{10^{2567}} < 4.03.\]

And finally $\lfloor \frac{1000!}{10^{2567}} \rfloor = 4$  which is
the most significant digit of $1000!$.  And the least significant
digit is of course $0$.

\newpar{5} We obtain $8! \simeq 40318.045$.

\newpar{6} Note $\mu(p)$ the multiplicity of the prime $p$.  We have
using (8)
\begin{eqnarray*}
  \mu(2) &=& 10 + 5 + 2 + 1 = 18 \\
  \mu(3) &=& 6 + 2 = 8 \\
  \mu(5) &=& 4 \\
  \mu(7) &=& 2 \\
  \mu(11) &=& 1 \\
  \mu(13) &=& 1 \\
  \mu(17) &=& 1 \\
  \mu(19) &=& 1
\end{eqnarray*}

Hence
\[ 20! = 2^{18} 3^8 5 ^4 7^2\,11\,13\,17\,19.\]

\newpar{7} If $x$ is real, we have
\begin{eqnarray*}
  x? &=& \frac{1}{2} x (x+1) \\
  &=& \frac{1}{2}x(x-1) + x \\
  &=& x + (x-1)?
\end{eqnarray*}

\newpar{8} For $n\ge 0$, we have

\begin{eqnarray*}
  \frac{m^n m!}{(n+1)(n+2) \ldots(n+m)}
  &=& n!\frac{m^n m!}{(m+n)!} \\
  &=& n! \frac{1}{\prod_{k=1}^n (1+k/m)} \\
  &{\longrightarrow \atop {m \to +\infty}}& n!
\end{eqnarray*}

\newpar{9}  We have $\Gamma(x) = x!/x$, thus $\Gamma(1/2) =
\sqrt{\pi}$.  From (15), we have

\begin{eqnarray*}
  \Gamma(-1/2) &=& \lim_{m\to +\infty} \frac{m^{-1/2}
    m!}{\prod_{k=0}^m (k-1/2)} \\
  &=& -2\lim_{m\to +\infty} \frac{m^{-1/2}m!}{\prod_{k=0}^{m-1}(k +
    1/2)} \\
  &=& -2 \Gamma(1/2) \\
  &=& -2 \sqrt{\pi}
\end{eqnarray*}

\newpar{10} We have using (15)
\begin{eqnarray*}
  \Gamma(x) &=& \lim_{m\to +\infty} \frac{m^x m!}{x(x+1)(x+2)\ldots
    (x+m)} \\
  &=& (x-1) \lim_{m \to +\infty} \frac{m^{x-1} m!}{(x-1)x \ldots
    (x-1+m)} \times \frac{m}{m+x} \\
  &=& (x-1) \Gamma(x-1)
\end{eqnarray*}

\newpar{11} We have using (8) and if we note $e_{r+1} = 0$
\begin{eqnarray*}
  \mu &=& \sum_{k>0} \left\lfloor \frac{n}{2^k} \right\rfloor \\
  &=& \sum_{k=1}^{e_r} \left\lfloor \frac{n}{2^k} \right\rfloor \\
  &=& \sum_{i=1}^r \sum_{e_{i+1} < k \le e_i} \left\lfloor
  \frac{n}{2^k} \right\rfloor \\
  &=& \sum_{i=1}^r \sum_{e_{i+1} < k\le e_i} \frac{\sum_{j=1}^i
    2^{e_j}}{2^k} \\
  &=& \sum_{i=1}^r \left(\sum_{j=1}^i 2^{e_j}\right)
  \frac{1}{2^{e_{i+1}+1}} \times \frac{1 - \frac{1}{2^{e_i -
        e_{i+1}}}}{\frac{1}{2}} \\
  &=& \sum_{i=1}^r \left(\sum_{j=1}^i 2^{e_j}\right) \left(
  \frac{1}{2^{e_{i+1}}} - \frac{1}{2^{e_i}} \right) \\
  &=& \sum_{i=2}^{r+1} \frac{1}{2^{e_i}} \sum_{j=1}^{i-1} 2^{e_j} -
  \sum_{i=1}^r \frac{1}{2^{e_i}} \sum_{j=1}^i 2^{e_j} \\
  &=& \sum_{j=1}^r 2^{e_j} - 1 + \sum_{i=2}^r \frac{-
    2^{e_i}}{2^{e_i}} \\
  &=& n - r
\end{eqnarray*}

Thus the multiplicity of $2$ in $n!$ is $n-r$.

\newpar{12} We have
\begin{eqnarray*}
  \mu &=& \sum_{i=1}^k \left\lfloor \frac{n}{p^i} \right\rfloor \\
  &=& \sum_{i=1}^k \frac{1}{p^i} \sum_{j=i}^k a_j p^j \\
  &=& \sum_{j=1}^k \sum_{i=1}^j \frac{a_j p^j}{p^i} \\
  &=& \sum_{j=1}^k a_j p^j \frac{1}{p}\ \frac{1 - p^{-j}}{1 -
    p^{-1}}\\
  &=& \sum_{j=1}^k a_j \frac{p^j - 1}{p-1} \\
  &=& \frac{n - \sum_{j=0}^k a_j}{p-1}
\end{eqnarray*}

\newpar{13} Since $p$ is prime, then all the integers $1, 2, \ldots,
p-1$ are invertible.  Moreover $x^2 \equiv 1 \bmod p$ if and only if
$x\equiv 1 \bmod p$ or $x \equiv -1 \bmod p$.  Thus by pairing each
integer between $2, 3, \ldots, p-2$ with their respective inverse, we
obtain
\[ (p-1)! \equiv p-1 \bmod p.\]
Hence $(p-1)! \bmod p = p-1$.

\newpar{14} Note $\mu(n)$ the multiplicity of $p$ in $n$.  Let's show
that for non-negative integer $l$, we have
\[ (p^l)!/(-p)^{\mu((p^l)!)} \equiv 1 \pmod p.\ \mbox{(*)}\]

Using (8) we have $\mu((p^l)!) = \frac{p^l-1}{p-1}$.  We then have
\[ \mu((p^l)!) - p^{l-1} = \mu((p^{l-1})!).\ \mbox{(**)}\]
Hence
\begin{eqnarray*}
  \frac{(p^l)!}{(-p)^{\mu((p^l)!)}} &=&
  \frac{\prod_{r=1}^{p^{l-1}} \prod_{(r-1)p < i\le rp}
    i}{(-p)^{\mu((p^l)!)}} \\
  &=& \frac{\prod_{r=1}^{p^{l-1}} \prod_{0<i\le p}(i+(r-1)p)}
       {(-p)^{\mu((p^l)!)}} \\
  &=& \frac{p^{p^{l-1}}\prod_{r=1}^{p^{l-1}}r
         \prod_{0<i<p}(i+(r-1)p)}{(-p)^{\mu((p^l)!)}} \\
  &=& \frac{\prod_{r=1}^{p^{l-1}}r
         \prod_{0<i<p}(i+(r-1)p)}{(-1)^{\mu((p^l)!)}
         p^{\mu((p^{l-1})!)}}, \mbox{using (**)} \\
  &=& \frac{\prod_{r=1}^{p^{l-1}} r
         \prod_{0<i<p}(i+(r-1)p)}{(-1)^{\mu((p^l)!)}
         \prod_{r=1}^{p^{l-1}} p^{\mu(r)}} \\
  &=& (-1)^{\mu((p^l)!)}\prod_{r=1}^{p^{l-1}} \frac{r}{p^{\mu(r)}}
       \prod_{i=1}^{p-1}(i+(r-1)p) \\
  &\equiv& (-1)^{\mu((p^l)!)} \prod_{r=1}^{p^{l-1}} -
       \frac{r}{p^{\mu(r)}}\pmod p,\ \mbox{using Wilson's theorem} \\
  &\equiv& (-1)^{\mu((p^{l-1})!)} \prod_{r=1}^{p^{l-1}}
       \frac{r}{p^{\mu(r)}} \pmod p,\ \mbox{using (**)} \\
  &\equiv& \frac{(p^{l-1})!}{(-p)^{\mu((p^{l-1})!)}} \pmod p
\end{eqnarray*}

We then deduce (*) by a simple induction.  For the general case, if we
note $\mu(n!) = \mu$, then we have
\begin{eqnarray*}
  \frac{n!}{(-p)^{\mu}} &=& \frac{\prod_{l=0}^k \prod_{a_k
      p^k+\cdots+a_{l+1}p^{l+1} < i \le a_k p^k + \cdots + a_l p^l}
    i}{(-p)^{\sum_{l=0}^k a_l \frac{p^l-1}{p-1}}} \\
  &=& \prod_{l=0}^k (-p)^{-a_l\frac{p^l-1}{p-1}} \prod_{j=0}^{a_l - 1}
  \prod_{a_kp^k+\cdots+a_{l+1}p^{l+1} + jp^l < i \le a_k p^k + \ldots
    + a_{l+1}p^{l+1} + (j+1)p^l} i \\
  &=& \prod_{l=0}^k \prod_{j=0}^{a_l-1} (-p)^{-\frac{p^l-1}{p-1}}
  \prod_{0 < i \le p^l} (i + j p^l + a_k p^k + \cdots + a_{l+1}
  p^{l+1}) \\
  &=& \prod_{l=0}^k \prod_{j=0}^{a_l-1} (-p)^{-\mu((p^l)!)}
  \prod_{0 < i \le p^l} (i + j p^l + a_k p^k + \cdots + a_{l+1} p^{l+1}) \\
  &=& \prod_{l=0}^k \prod_{j=0}^{a_l-1} \prod_{0 < i \le p^l} \frac{i + j
    p^l + a_k p^k + \cdots + a_{l+1}p^{l+1}}{(-p)^{\mu(i)}} \\
  &=& \prod_{l=0}^k \prod_{j=0}^{a_l-1} (-1)^l (1+j + a_k p^{k-l} +
  \cdots + a_{l+1}p) \\
  &&\prod_{0 < i < p^l} \frac{i + j p^l + a_k p^k + \cdots +
    a_{l+1}p^{l+1}}{(-p)^{\mu(i)}} \\
  &=& \prod_{l=0}^k \prod_{j=0}^{a_l-1} (-1)^l (1+j + a_k p^{k-l} +
  \cdots + a_{l+1}p) \\
  &&\prod_{0 < i < p^l} (-1)^{\mu(i)} \left(\frac{i}{p^{\mu(i)}} +
  p^{l-\mu(i)} (j + a_k p^{k-l} + \cdots + a_{l+1}p\right) \\
\end{eqnarray*}

If $0 < i < p^l$ then $\mu(i) < l$, we then deduce that

\begin{eqnarray*}
  \frac{n!}{(-p)^\mu} &\equiv& \prod_{l=0}^k \prod_{j=0}^{a_l - 1}
  (-1)^l (1+j) \prod_{0<i<p^l} \frac{i}{(-p)^{\mu(i)}} \pmod p \\
  &\equiv& \prod_{l=0}^k a_l! \prod_{0<i\le p^l}
  \frac{i}{(-p)^{\mu(i)}} \pmod p \\
  &\equiv& \prod_{l=0}^k a_l! \pmod p,\ \mbox{using (*)}
\end{eqnarray*}

\newpar{15} It's easy to see that the permanent of matrix where we
multiply a column by a constant is equal to that same constant
multiplying the permanent of the initial matrix --- just by developing
the permanent according to that column.

We then deduce that the permanent of the square matrix is equal to
$n!$ multiplied by the permanent of the matrix where each column is
equal to the vector
\[ \left(
\begin{array}{c}
  1 \\
  2 \\
  \vdots \\
  n
\end{array}
\right)
\]

Let's show by induction on the size of the matrix that the permanent
of that matrix is equal to $n!^2$.  If $n=1$, it's obvious.  Suppose
that we have the property for $n-1$.  Thus by developing the permanent
according to the last row and by induction, it's equal to:
\[ n(n-1)!^2 \times n = n!^2.\]

Hence the permanent of the initial square matrix is $n!^3$.

\newpar{16} We have
\[ \lim_{n \to +\infty} \left(1 -\frac{1}{1!} + \frac{1}{2!} - \cdots
+ \frac{(-1)^n}{n!} \right) = \frac{1}{e} .\]

Hence if $n$ is negative, the general term of the series tends to
$-\infty$ when $n$ tends to $+\infty$.  Thus the series diverges.

\newpar{17} We have
\begin{eqnarray*}
  \prod_{i=1}^k
  \frac{\Gamma(1+\beta_i)}{\Gamma(1+\alpha_i)} &=&
  \prod_{i=1}^k \lim_{m\to +\infty} \frac{m^{1+\beta_i}
    m!(1+\alpha_i) \ldots (1+\alpha_i+m)}
       {m^{1+\alpha_i}m!(1+\beta_i)\ldots (1+\beta_i+m)} \\
  &=& \lim_{m\to +\infty}
  \frac{m^{\sum_{i=1}^k\beta_i}}{m^{\sum_{i=1}^k\alpha_i}}
  \prod_{i=1}^k \frac{(1+\alpha_i) \ldots (1+\alpha_i +
    m)}{(1+\beta_i) \ldots (1+\beta_i+m)} \\
  &=& \lim_{m\to+\infty} \prod_{i=1}^k \prod_{j=0}^m \frac{1+\alpha_i
    + j}{1+\beta_i+j} \\
  &=& \lim_{m\to+\infty} \prod_{j=0}^m \prod_{i=1}^k \frac{1+\alpha_i
    + j}{1+\beta_i+j} \\
  &=& \prod_{n\ge 0} \prod_{i=1}^k \frac{n+\alpha_i}{n+\beta_i}
\end{eqnarray*}

\newpar{18} We have

\begin{eqnarray*}
  \frac{\pi}{2} &=& \prod_{k\ge 1}\frac{(2k)^2}{(2k-1)(2k+1)} \\
  &=& \lim_{m\to \infty}\prod_{k=1}^m \frac{(2k)^2}{(2k-1)(2k+1)} \\
  &=& \lim_{m\to +\infty} \frac{(2^m
    m!)^2}{\prod_{k=1}^m(2k-1)(2k+1)}\\
  &=& \lim_{m\to +\infty} (2m+1)\left( \frac{2^m m!}{\prod_{k=1}^m
    (2k+1)} \right)^2 \\
  &=& 2 \lim_{m\to+\infty} \left( \frac{\sqrt{m}\,
    m!}{\prod_{k=1}^m\left(k + \frac{1}{2}\right) }\right)^2 \\
  &=& 2 \left(\frac{1}{2}\right)!,\ \mbox{from (15)}
\end{eqnarray*}

Hence $\left(\frac{1}{2}\right)! = \frac{\sqrt{\pi}}{2}$.

\newpar{19} Note if $x>0$ and $m>1$:
\[ f_m(x) = \int_0^m \left( 1 - \frac{t}{m} \right)^m t^{x-1} dt.\]

By doing the variable change $t \la mu$, we have
\[ f_m(x) = m^x\int_0^1 (1 - u)^m  u^{x-1} du.\]

If we note
\[g_m(x) = \int_0^1 (1 - t)^m  t^{x-1} dt.\]

By doing an integration by parts we have:
\begin{eqnarray*}
  g_m(x) &=& \left[ \frac{t^x}{x} (1-t)^m
    \right]_0^1 + \frac{m}{x} \int_0^1 (1-t)^{m-1} t^x dt \\
  &=& \frac{m}{x} g_{m-1}(x+1) \\
  &=& \frac{m!}{\prod_{k=0}^{m-1} (x+k)}g_0(x+m),\ \mbox{by
    induction}\\
  &=& \frac{m!}{\prod_{k=0}^{m-1}(x+k)} \int_0^1 t^{x+m-1}dt\\
  &=& \frac{m!}{\prod_{k=0}^m(x+k)}
\end{eqnarray*}

We then deduce that $f_m(x) = \Gamma_m(x).$

\newpar{20} We have the following inequality for all real $x$
\[ e^x \ge 1+x.\]
Hence if $0\le t \le m$ \[ e^{\pm t/m} \ge 1 \pm t/m.\]
and finally \[ e^{\pm t} \ge (1 \pm t/m)^m\]

On the other hand,
\begin{eqnarray*}
  (1 - t/m)^m &=& e^{-t} (1-t/m)^m e^t \\
  &\ge& e^{-t} (1-t/m)^m (1+t/m)^m \\
  &=& e^{-t} \left(1-\left(\frac{t}{m}\right)^2\right)^m \\
  &\ge& e^{-t} (1 - t^2/m)
\end{eqnarray*}

The last inequality is obtained by the inequality $(1+\alpha)^\beta
\ge 1 + \beta\alpha$ if $\beta \ge 1$ and $\alpha \ge -1$.  Thus
finally we have if $0\le t\le m$
\[ 0 \le e^{-t} - (1-t/m)^m \le t^2e^{-t}/m.\]

For sufficiently large values of $t$ we have $0 \le t^{x-1} \le
t^{x+1} \le e^{t/2}$.  Hence $t \mapsto e^{-t}t^{x-1}$ and $t\mapsto
e^{-t}t^{x+1}$ are integrable from $0$ to $+\infty$. Thus we have

\begin{eqnarray*}
  0 &\le& \int_0^m e^{-t} t^{x-1} dt - \Gamma_m(x) \\
  &=& \int_0^m (e^{-t} - (1-t/m)^m) t^{x-1} dt \\
  &\le& \frac{1}{m} \int_0^m t^{x+1} e^{-t} dt \\
  &\le& \frac{1}{m} \int_0^{+\infty} t^{x+1}e^{-t} dt
\end{eqnarray*}

We then deduce that
\[ \lim_{m\to +\infty} \Gamma_m(x) = \int_0^{+\infty} e^{-t}
t^{x-1}dt.\]

\newpar{21}  Let's prove the equality by induction for $n>0$.  We have
the equality for $n=1$.  Suppose that we also have the equality for
$n\ge 1$.  We then have by induction
\begin{eqnarray*}
  D_x^{n+1}w &=& \sum_{j=0}^n \sum_{k_1+\cdots+k_n=j \atop
    {k_1+2k_2+\cdots+nk_n = n \atop k_1,\ldots,k_n \ge 0}} D_u^{j+1} w
  \frac{n! (D_x^1 u)^{k_1+1} (D_x^2 u)^{k_2} \ldots (D_x^n
    u)^{k_n}}{k_1!(1!)^{k_1}\ldots k_n! (n!)^{k_n}} +\\
  && \sum_{j=0}^n \sum_{l=1}^n \sum_{k_1+\cdots+k_n = j \atop
    {k_1+2k_2+\cdots+nk_n = n \atop {k_1, \ldots, k_n \ge 0
  \atop k_l > 0}}} D_u^j
  w\,n! \left(\prod_{i=1\atop i\not=l}^n
  \frac{(D_x^iu)^{k_i}}{k_i!(i!)^{k_i}}\right)
  \frac{(D_x^lu)^{k_l-1}}{(k_l-1)!(l!)^{k_l}} D_x^{l+1}u
\end{eqnarray*}

Let's note $T_1$ the first term and $T_2$ the second term.  By doing
the variables change $k_1+1 \ra k_1$ and $j\ra j+1$, we have
\begin{eqnarray*}
  T_1 &=& \sum_{j=1}^{n+1} \sum_{k_1 + \cdots + k_n = j \atop {
      k_1 + 2 k_2 + \cdots + n k_n = n+1 \atop {k_1, \ldots, k_n \ge
        0 \atop k_1 > 0}}} k_1 D_u^jw \frac{n!}{k_1!(1!)^{k_1} \ldots
    k_n!(n!)^{k_n}} (D_x^1u)^{k_1} \ldots (D_x^nu)^{k_n} \\
  &=& \sum_{j=1}^{n+1} \sum_{k_1 + \cdots + k_n = j \atop {
      k_1 + 2 k_2 + \cdots + (n+1) k_{n+1} = n+1 \atop {k_1, \ldots, k_{n+1} \ge
        0 \atop k_1 > 0}}} k_1 D_u^jw \frac{n!(D_x^1u)^{k_1} \ldots
    (D_x^{n+1}u)^{k_{n+1}}}{k_1!(1!)^{k_1} \ldots
    k_{n+1}!((n+1)!)^{k_{n+1}}}
\end{eqnarray*}

We obtain the last equality since $k_1 > 0$ and $k_{n+1} > 0$ can't be
satisfied simultaneously, otherwise $k_1 + 2 k_2 \cdots +
(n+1)k_{n+1} > n+1$.  So $k_{n+1} = 0$ in the above summation.
Moreover we could extend the sum for $k_1 = 0$ and $j=0$.  So finally,

\[  T_1 = \sum_{j=0}^{n+1} \sum_{k_1 + \cdots + k_n = j \atop {
      k_1 + 2 k_2 + \cdots + (n+1) k_{n+1} = n+1 \atop {k_1, \ldots, k_{n+1} \ge
        0}}} k_1 D_u^jw \frac{n!(D_x^1u)^{k_1} \ldots
    (D_x^{n+1}u)^{k_{n+1}}}{k_1!(1!)^{k_1} \ldots
    k_{n+1}!((n+1)!)^{k_{n+1}}}.\]

By doing the variables substitution $k_l - 1 \ra k_l$ and $k_{l+1} + 1
\ra k_l$, we obtain
\begin{eqnarray*}
  T_2 &=& \sum_{j=0}^n \sum_{l=1}^n \sum_{k_1+\cdots+k_n = j \atop {
      k_1 + 2k_2 + \ldots + n k_n = n+1 \atop {
        k_1, \ldots, k_n \ge 0 \atop k_{l+1} > 0}}} k_{l+1}(l+1) D_u^j
  w \frac{n!(D_u^1)^{k_1} \ldots (D_u^n)^{k_n}}{k_1!(1!)^{k_1} \ldots
    k_n!(n!)^{k_n}} \\
  &=& \sum_{j=0}^n \sum_{l=1}^n \sum_{k_1+\cdots+k_{n+1} = j \atop {
      k_1 + \ldots + (n+1) k_{n+1} = n+1 \atop {
        k_1, \ldots, k_{n+1} \ge 0 \atop k_{l+1} > 0}}} D_u^j
  w \frac{k_{l+1}(l+1) n!(D_u^1)^{k_1} \ldots
    (D_u^{n+1})^{k_{n+1}}}{k_1!(1!)^{k_1} \ldots k_{n+1}!((n+1)!)^{k_{n+1}}} \\
\end{eqnarray*}

The last equality is obtained applying the same reasoning we used when
$k_1 > 0$.  Moreover it's also true for $k_{l+1} = 0$, and by
exchanging the two innermost summations we obtain

\begin{eqnarray*}
  &=& \sum_{j=0}^n \sum_{k_1+\cdots+k_{n+1} = j \atop {
      k_1 + 2k_2 + \ldots + n k_{n+1} = n+1 \atop {
        k_1, \ldots, k_{n+1} \ge 0}}} \left(\sum_{l=2}^{n+1}lk_l\right) D_u^j
  w \frac{n!(D_u^1)^{k_1} \ldots (D_u^{n+1})^{k_{n+1}}}{k_1!(1!)^{k_1} \ldots
    k_{n+1}!((n+1)!)^{k_{n+1}}} \\
  &=& \sum_{j=0}^n \sum_{k_1+\cdots+k_{n+1} = j \atop {
      k_1 + 2k_2 + \ldots + n k_{n+1} = n+1 \atop {
        k_1, \ldots, k_{n+1} \ge 0}}} (n+1-k_1) D_u^j
  w \frac{n!(D_u^1)^{k_1} \ldots (D_u^{n+1})^{k_{n+1}}}{k_1!(1!)^{k_1} \ldots
    k_{n+1}!((n+1)!)^{k_{n+1}}} \\
\end{eqnarray*}

If we have $k_1+\cdots+k_{n+1} = n+1$ and $k_1 + 2k_2 + \ldots + (n+1)
k_{n+1} = n+1$, by subtracting the two equalities, we have $k_2 +
\ldots + nk_{n+1} = 0$.  Hence $k_2 = \cdots = k_{n+1} = 0$ and $k_1 =
n+1$.  Thus,

\[ T_2 = \sum_{j=0}^{n+1} \sum_{k_1+\cdots+k_{n+1} = j \atop {
      k_1 + 2k_2 + \ldots + n k_{n+1} = n+1 \atop {
        k_1, \ldots, k_{n+1} \ge 0}}} (n+1-k_1) D_u^j
  w \frac{n!(D_u^1)^{k_1} \ldots (D_u^{n+1})^{k_{n+1}}}{k_1!(1!)^{k_1} \ldots
    k_{n+1}!((n+1)!)^{k_{n+1}}} .
  \]

Thus finally,

\begin{eqnarray*}
  D_x^{n+1}w &=& T_1 + T_2 \\
  &=& \sum_{j=0}^{n+1} \sum_{k_1+\cdots+k_{n+1} = j \atop {
      k_1 + 2k_2 + \ldots + n k_{n+1} = n+1 \atop {
        k_1, \ldots, k_{n+1} \ge 0}}} D_u^j
  w \frac{(n+1)!(D_u^1)^{k_1} \ldots (D_u^{n+1})^{k_{n+1}}}{k_1!(1!)^{k_1} \ldots
    k_{n+1}!((n+1)!)^{k_{n+1}}}
\end{eqnarray*}

\newpar{22} A good guess would be to approximate $(n + x)!/n!$ with
$n^x$.  If $x$ is an integer we have
\begin{eqnarray*}
  \frac{(n+x)!}{n!} &=& \prod_{k=1}^x (n+k) \\
  &=& n^x \prod_{k=1}^x\left(1 + \frac{k}{n}\right) \\
  &{n \ra +\infty \atop \sim}& n^x
\end{eqnarray*}

Moreover if we assume that $x! = x (x-1)!$ if $x>0$, we have by induction
\[ (n+x)! = x! \prod_{k=1}^n (x+k)\]

Combining with the proposed approximation we then deduce that
\[ \frac{x! \prod_{k=1}^n(x+k)}{n!} \sim n^x.\]
Thus finally
\[ x! = \lim_{n\ra +\infty} \frac{n^x n!}{\prod_{k=1}^n(x+k)}.\]

\newpar{23} We have using (15) if $z$ is not an integer
\begin{eqnarray*}
  (-z!)(z!) &=& \lim_{m\ra +\infty} \frac{m^{-z} m!}{\prod_{k=1}^m(-z
    + k)} \times \frac{m^z m!}{\prod_{k=1}^m (z+k)} \\
  &=& \lim_{m\ra +\infty} \frac{1}{\prod_{k=1}^m
    \left(1-\frac{z}{k}\right) \left(1+\frac{z}{k}\right)} \\
  &=& \frac{1}{\prod_{k=1}^{+\infty} \left(1 -
    \frac{z^2}{k^2}\right)}\\
  &=& \frac{\pi z} {\sin \pi z}
\end{eqnarray*}
Hence
\[ (-z!) \Gamma(z) = \frac{\pi}{\sin \pi z}.\]

\newpar{24} We have $1+x\le e^x$ for all real x.  Thus we have for $k >0$
\begin{eqnarray*}
  \frac{k+1}{k} &\le& e^{1/k} \\
  \frac{(k+1)^k}{k^k} &\le& e \\
  \prod_{k=1}^{n-1} \frac{(k+1)^k}{k^k} &\le& e^{n-1} \\
  \frac{\prod_{k=2}^n k^{k-1}}{\prod_{k=1}^{n-1} k^k} &\le& e^{n-1} \\
  \frac{n^{n-1}}{\prod_{k=2}^{n-1}k} &\le& e^{n-1} \\
  \frac{n^n}{e^{n-1}} &\le& n!
\end{eqnarray*}

And for $k>1$
\begin{eqnarray*}
  1 - \frac{1}{k} &\le& e^{-1/k} \\
  e &\le& \frac{k^k}{(k-1)^k} \\
  e^{n-1} &\le& \prod_{k=2}^n \frac{k^k}{(k-1)^k} \\
  e^{n-1} &\le& \frac{\prod_{k=2}^n k^k}{\prod_{k=1}^{n-1}k^{k+1}} \\
  e^{n-1} &\le& \frac{n^n}{\prod_{k=2}^{n-1}k} \\
  n! &\le& \frac{n^{n+1}}{e^{n-1}}
\end{eqnarray*}

\newpar{25} Using (21), we have
\begin{eqnarray*}
  x^{\underline{m+n}} &=& \frac{x!}{(x-m-n)!} \\
  &=& \frac{x!}{(x-m)!} \times \frac{(x-m)!}{(x-m-n)!} \\
  &=& x^{\underline{m}}(x-m)^{\underline{n}}
\end{eqnarray*}

and
\begin{eqnarray*}
  x^{\overline{m+n}} &=& \frac{\Gamma(x+m+n)}{\Gamma(x)} \\
  &=& \frac{\Gamma(x+m+n)}{\Gamma(x+m)} \times
  \frac{\Gamma(x+m)}{\Gamma(x)} \\
  &=& x^{\overline{m}}(x+m)^{\overline{n}}
\end{eqnarray*}

\end{document}
