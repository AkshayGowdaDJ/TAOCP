\documentclass[a4paper,12pt]{article}
\begin{document}

\noindent
\textbf{1.} $t \leftarrow a$, $a \leftarrow b$, $b \leftarrow c$, 
$c \leftarrow d$, $a \leftarrow t$.

\bigskip
\noindent
\textbf{2.} We have the following algorithm:
\begin{description}
\item[E1.]
[Find remainder.] Divide $m$ by $n$ and let $r$ be the remainder.
(We will have $0 \leq r < n$.)
\item[E2.]
[Is it zero?] If $r = 0$, the algorithm terimates; n is the answer.
\item[E3.]
[Interchange.] Set $m \leftarrow n$, $n \leftarrow r$, and go back
to step E1. $|$
\end{description}

\noindent
Suppose $m \ge n$\\
after \textbf{E1}, $m \ge n$\\
after \textbf{E2}, $m \ge n$

\noindent
after \textbf{E3}, we have $m > n$ because $0 \le r < n$.

Suppose $m < n$ then $r = m$, and after the execution of \textbf{E3}, 
if we get there, $m > n$.

\medskip
Thus, $m$ is always greater than n at the beginning of \textbf{E1}, 
except possibly the first time this step occurs.

\bigskip
\noindent
\textbf{3.} Here's the algorithm:
\begin{description}
\item[F1.]
Divide $m$ by $n$ and let $r$ be the remainder. $(0 \le r < n)$
\item[F2.]
If $r = 0$, the algorithm terminates; $n$ is the answer.
\item[F3.]
Divide $n$ by $r$ and let m be the remainder. $(0 \le m < r)$
\item[F4.]
If $m = 0$, the algorithm terminates; $r$ is the answer.
\item[F5.]
Divide $r$ by $m$ and let $n$ be the remainder. $(0 \le n < m)$
\item[F5.]
If $n = 0$, the algorithm terminates; $m$ is the answer; otherwise go
back to step \textbf{F1}. $|$
\end{description}

\bigskip
\noindent
\textbf{4.} $\gcd(2166, 6099) = 57$.

\bigskip
\noindent
\textbf{5.} Finiteness: the reader can work on exercises and check answers
indefinitely.\\
Output: the procedure have no output.\\
Effectiveness: relaxing and sleeping are not effective operations.

\bigskip
\noindent
\textbf{6.} $N_0 = 1; N_1 = 2; N_2 = 3; N_3 = 4; N_4 = 3;$ \\
$T_5 = \frac{1+2+3+4+3}{5} = \frac{13}{5} = 2.6$.

\medskip
If $k < 5$, the first iteration just permutes the values of $m$ and $n$.
And if $k \ge 5$, the first iteration is a division of $k$ by $5$.
So the number of steps is the same for each class modulo 5 of integers.

\bigskip
\noindent
\textbf{7.} $U_n = T_{n+1}$

\bigskip
\noindent
\textbf{8.}
\begin{quote}
\begin{tabular}{|c|c|c|c|c|l}
\cline{1-5}
&$t_j$&$f_j$&$a_j$&$b_j$\\
\cline{1-5}
$0$&$ab$&---&$2$&$1$&Suppress an occurence of $ab$\\
\cline{1-5}
$1$&---&$c$&---&$0$&Adding $c$ at the beginning of the string\\
\cline{1-5}
$2$&$a$&$b$&$3$&$2$&Replacing all $a$'s by $b$'s\\
\cline{1-5}
$3$&$c$&$a$&$4$&$3$&Replacing all $c$'s by $b$'s\\
\cline{1-5}
$4$&$b$&$b$&$5$&$0$&If b's remain, repeat\\
\cline{1-5}
\end{tabular}
\end{quote}
\end{document}
