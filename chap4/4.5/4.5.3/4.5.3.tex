\documentclass[a4paper,12pt]{article}
\newcommand{\newpar}[1]{\bigskip \noindent \textbf{#1.}}
\newcommand{\subpar}[1]{\medskip \noindent (#1)}
\newcommand{\la}{\leftarrow}
\newcommand{\ra}{\rightarrow}

\begin{document}

\newpar{2} Let's show by induction on $n > 1$ that

\begin{eqnarray*}
  P_n &=&
\left(
\begin{array}{cc}
  x_1 & 1 \\
  1 & 0
\end{array}
\right)
\left(
\begin{array}{cc}
  x_2 & 1 \\
  1 & 0
\end{array}
\right)
\cdots
\left(
\begin{array}{cc}
  x_n & 1 \\
  1 & 0
\end{array}
\right) \\
&=&
\left(
\begin{array}{cc}
  K_n(x_1, x_2, \ldots, x_n) & K_{n-1}(x_1, x_2, \ldots, x_{n-1}) \\
  K_{n-1}(x_2, x_3, \ldots, x_n) & K_{n-2}(x_2, x_3, \ldots, x_{n-1})
\end{array}
\right)
\end{eqnarray*}

We verify easily that the equality holds for $n = 2$.  Suppose that
it's true for $n > 1$.  Then by induction we have,

\begin{eqnarray*}
  P_{n+1} &=& P_n
  \left(
  \begin{array}{cc}
    x_{n+1} & 1 \\
    1 & 0
  \end{array}
  \right) \\ &=&
  \left(
  \begin{array}{cc}
    K_n(x_1, x_2, \ldots, x_n) & K_{n-1}(x_1, x_2, \ldots, x_{n-1}) \\
    K_{n-1}(x_2, x_3, \ldots, x_n) & K_{n-2}(x_2, x_3, \ldots, x_{n-1})
  \end{array}
  \right)
  \left(
  \begin{array}{cc}
    x_{n+1} & 1 \\
    1 & 0
  \end{array}
  \right) \\
  &=&
  \left(
  \begin{array}{cc}
    x_{n+1} K_n(x_1, \ldots, x_n) + K_{n-1}(x_1, \ldots, x_{n-1}) &
    K_{n-1}(x_1, \ldots, x_{n-1}) \\
    x_{n+1} K_n(x_2, \ldots, x_n) + K_{n-2}(x_1, \ldots, x_{n-1}) &
    K_{n-1}(x_2, \ldots, x_n) \\
  \end{array}
  \right) \\
  &=&
  \left(
  \begin{array}{cc}
    K_{n+1}(x_1, \ldots, x_{n+1}) & K_{n-1}(x_1, \ldots, x_{n-1}) \\
    K_n(x_2, \ldots, x_{n+1}) & K_{n-1}(x_2, \ldots, x_{n-1})
  \end{array}
  \right),\,\mbox{from (7)}
\end{eqnarray*}

\newpar{3}  Let's show that the determinant is equal to $q_1q_2\ldots
q_n$.  In order to accomplish this, we replace successively for $i$
varying from $2$ to $n$ the $i^{th}$ column $C_i$ by $m_iC_i -
C_{i-1}$ where $m_i = C_{i-1,i-1}$.

Let's show by induction that after the $i^{th}$ transformation, the
column $C_k$ where $1\le k\le i+1$ is equal to
$(C_{k,1},\ldots,C_{k,n})$ where

\[
C_{k,l} = \left\{
\begin{array}{ll}
  q_k &\mbox{if $l=k$} \\
  -q_{k-1}& \mbox{if $l=k+1$} \\
  0& \mbox{otherwise}
\end{array}
\right.
\]

After accomplishing $0$ transformation, we have $C_{1,1} = q_1$,
$C_{1, 2} = -q_0$ and $C_{1, l} = 0$ otherwise.

Suppose we have the property for $i$ such that $1\le i\le n-2$.  After
the $(i+1)^{th}$ transformation all the other columns remain the same
except $C_{i+1}$.  Thus by induction we have the property for $1\le
k\le i$.  And we have

\[
C_{i+1,l} = \left\{
\begin{array}{ll}
  x_{i+1}q_i + q_{i-1} = q_{i+1} &\mbox{if $l=i+1$} \\
  -q_i& \mbox{if $l=i+2$} \\
  0& \mbox{otherwise}
\end{array}
\right.
\]

And finally, the initial determinant is equal to the determinant of a
trigonal matrix whose diagonal elements are equal to $q_1, q_2,
\ldots, q_n$.  Hence, we have the desired result.

\newpar{4} Let's show the relation by induction on $n \ge 1$.  If
$n=1$, we have

\[ K_1(x_1)K_1(x_2) - K_2(x_1,x_2) = -1.\]

Suppose that we have the property for $n \ge 1$, we then have

\begin{eqnarray*}
  &&K_{n+1}(x_1,\ldots,x_{n+1})K_{n+1}(x_2,\ldots,x_{n+2}) \\
  &=& K_{n+1}(x_1,\ldots,x_{n+1})(x_{n+2}K_n(x_2,\ldots,x_{n+1}) +
  K_{n-1}(x_2,\ldots,x_n)) \\
  &=& x_{n+2} K_{n+1}(x_1,\ldots, x_{n+1}) K_n(x_2,\ldots,x_{n+1}) + \\
  &&K_{n+1}(x_1,\ldots,x_{n+1}) K_{n-1}(x_2,\ldots,x_n) \\
  &=& (K_{n+2}(x_1,\ldots,x_{n+2}) - K_n(x_1,\ldots, x_n))
  K_n(x_2,\ldots,x_{n+1}) + \\
  && K_{n+1}(x_1,\ldots,x_{n+1}) K_{n-1}(x_2,\ldots,x_n) \\
  &=& K_{n+2}(x_1,\ldots,x_{n+2})K_n(x_2,\ldots,x_{n+1}) +
  (-1)^{n+1},\,\mbox{by induction}
\end{eqnarray*}

Thus we have also verified that we have the property for $n+1$.

\newpar{5} We have,

\[ q_n q_{n+1} - q_{n-1} q_n = x_{n+1} q_n q_{n-1} > 0.\]

Thus the sequence $\left(\frac{1}{q_nq_{n-1}}\right)_{n>0}$ is positive
and decreasing.  If we show that the sequence tends to $0$ when $n$
tends to $+\infty$, we then deduce from (9) that $//x_1,x_2,\ldots//$
exists because it's the sum of decreasing alternate sequence that tends
to $0$.  Note $u_n$ the sequence defined by

\[
u_n = \left\{
\begin{array}{ll}
  1,&\mbox{if $n=0$} \\
  \epsilon,&\mbox{if $n=1$} \\
  \epsilon u_{n-1} + u_{n-2},&\mbox{if $n>1$}
\end{array}
\right.
.\]

We deduce easily from (7) that $q_n \ge u_n$ for $n\ge0$.Let's show by
induction on $n$ that

\[ u_n = \frac{1}{\sqrt{\epsilon^2+4}}\left(\left(\frac{\epsilon +
  \sqrt{\epsilon^2 + 4}}{2}\right)^{n+1} -
\left(\frac{\epsilon - \sqrt{\epsilon^2 + 4}}{2}\right)^{n+1}\right) .\]

We have the inequality for $n=0$ and $n=1$.  Now suppose that we have
the equality for $0\le k\le n$ where $n\ge 2$.  Then we have by
induction

\begin{eqnarray*}
  u_{n+1} &=& \epsilon u_n + u_{n-1} \\
  &=& \frac{1}{\sqrt{\epsilon^2+4}}\left(
  \left(\frac{\epsilon + \sqrt{\epsilon^2 + 4}}{2} \right)^n
  \left(\epsilon\ \frac{\epsilon + \sqrt{\epsilon^2+4}}{2} + 1 \right)
  -\right.\\
  && \left.\left(\frac{\epsilon - \sqrt{\epsilon^2 + 4}}{2} \right)^n
  \left(\epsilon\ \frac{\epsilon - \sqrt{\epsilon^2+4}}{2} + 1 \right)
  \right) \\
  &=& \frac{1}{\sqrt{\epsilon^2+4}}\left(
  \left(\frac{\epsilon + \sqrt{\epsilon^2 + 4}}{2} \right)^n
  \left(\frac{\epsilon + \sqrt{\epsilon^2+4}}{2}\right)^2
  -\right.\\
  && \left.\left(\frac{\epsilon - \sqrt{\epsilon^2 + 4}}{2} \right)^n
  \left(\frac{\epsilon - \sqrt{\epsilon^2+4}}{2}\right)^2
  \right) \\
  &=& \frac{1}{\sqrt{\epsilon^2+4}}\left(\left(\frac{\epsilon +
  \sqrt{\epsilon^2 + 4}}{2}\right)^{n+2} -
\left(\frac{\epsilon - \sqrt{\epsilon^2 + 4}}{2}\right)^{n+2}\right)
\end{eqnarray*}

We then deduce easily that

\[ \lim_{n\to +\infty}\frac{1}{q_nq_{n-1}} = 0.\]

Now consider the sequence $(x_n)_{n>0}$ defined by

\[
x_n = \left\{
\begin{array}{ll}
  1,&\mbox{if $n=1$}\\
  1,&\mbox{if $n=2$}\\
  \frac{4(n-1)}{n(n-2)(5n-7)},&\mbox{if $n>2$}
\end{array}
\right.
.\]

Let's show by induction on $n>0$ that $q_n \le \frac{5}{2} -
\frac{1}{n}$.

We have $q_1 = 1 \le \frac{3}{2}$ and

\[  q_2 = x_2 q_1 + q_0 = 2 \le \frac{5}{2} - \frac{1}{2}.\]

Now suppose that the property is true for $n \ge 2$ and $n-1$.  We
then have

\begin{eqnarray*}
  q_{n+1} &=& x_{n+1} q_n + q_{n-1} \\
  &\le& x_{n+1}\left(\frac{5}{2} - \frac{1}{n}\right) + \frac{5}{2} -
  \frac{1}{n-1} \\
  &=& \frac{5}{2} - \frac{1}{n+1}
\end{eqnarray*}

We then deduce that the sequence $(q_n)_{n\in \mathbf{N}}$ is bounded
above by $\frac{5}{2}$. Hence, $q_nq_{n-1}$ doesn't tend to $0$ when
$n$ tends to $+\infty$.  We then deduce from (9) that
$//x_1,\ldots,x_n//$ has no limit when $n$ tends to $+\infty$.

\end{document}
